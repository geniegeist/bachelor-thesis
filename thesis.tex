%% thesis.tex 2014/04/11
%
% Based on sample files of unknown authorship.
%
% The Current Maintainer of this work is Paul Vojta.

\documentclass[masters]{ucbthesis}

\usepackage{biblatex}
\usepackage{rotating} % provides sidewaystable and sidewaysfigure

\usepackage[utf8]{inputenc}
\usepackage{amsmath,amsthm,amssymb}

\usepackage{mathtools}
\usepackage{bbm}
\usepackage{marvosym}
\usepackage[hidelinks]{hyperref}
\usepackage{framed}
\usepackage{enumitem}
\usepackage{float}
\usepackage{bm}
\usepackage{tabularx}
\usepackage{booktabs}
\usepackage{url}


\newtheorem{theorem}{Theorem}[chapter]
\newtheorem{proposition}[theorem]{Proposition}
\newtheorem{lemma}[theorem]{Lemma}
\newtheorem{corollary}[theorem]{Corollary}
\newtheorem{conjecture}[theorem]{Conjecture}
\newtheorem{postulate}[theorem]{Postulate}
\theoremstyle{definition}
\newtheorem{definition}[theorem]{Definition}
\newtheorem{example}[theorem]{Example}
\newtheorem{observation}{Observation}
\newtheorem{remark}[theorem]{Remark}

\newcommand{\vertiii}[1]{{\left\vert\kern-0.25ex\left\vert\kern-0.25ex\left\vert#1 
    \right\vert\kern-0.25ex\right\vert\kern-0.25ex\right\vert}}

\newcommand*\enlarg[2]{\bar{#1}_{#2}}
\newcommand*\cohnorm[1]{\vertiii{#1}^{\mathrm{coh}}_{K,\varphi,\alpha,\gamma}}

% To compile this file, run "latex thesis", then "biber thesis"
% (or "bibtex thesis", if the output from latex asks for that instead),
% and then "latex thesis" (without the quotes in each case).

% Double spacing, if you want it.  Do not use for the final copy.
% \def\dsp{\def\baselinestretch{2.0}\large\normalsize}
% \dsp

% If the Grad. Division insists that the first paragraph of a section
% be indented (like the others), then include this line:
% \usepackage{indentfirst}

\addtolength{\abovecaptionskip}{\baselineskip}

\bibliography{references}

\hyphenation{mar-gin-al-ia}
\hyphenation{bra-va-do}

\begin{document}

% Declarations for Front Matter

\title{On the Hairer-Caravenna-Zambotti Reconstruction}
\author{Viet-Duc Nguyen}
\degreesemester{Summer}
\degreeyear{2022}
\degree{Bachelor of Science}
\chair{Professor Peter K. Friz}
\othermembers{Professor Peter Bank}
% For a co-chair who is subordinate to the \chair listed above
% \cochair{Professor Benedict Francis Pope}
% For two co-chairs of equal standing (do not use \chair with this one)
% \cochairs{Professor Richard Francis Sony}{Professor Benedict Francis Pope}
\numberofmembers{3}
% Previous degrees are no longer to be listed on the title page.
% \prevdegrees{B.A. (University of Northern South Dakota at Hoople) 1978 \\
%   M.S. (Ed's School of Quantum Mechanics and Muffler Repair) 1989}
\field{Mathematics}
% Designated Emphasis -- this is optional, and rare
% \emphasis{Colloidal Telemetry}
% This is optional, and rare
% \jointinstitution{University of Western Maryland}
% This is optional (default is Berkeley)
% \campus{Berkeley}

% For a masters thesis, replace the above \documentclass line with
% \documentclass[masters]{ucbthesis}
% This affects the title and approval pages, which by default calls this
% document a "dissertation", not a "thesis".

\maketitle
% Delete (or comment out) the \approvalpage line for the final version.
% \approvalpage
\copyrightpage

\begin{alwayssingle}
\section*{Zusammenfassung in deutscher Sprache}

Gegeben sei eine Familie von Distributionen \( (F_x)_{x \in \mathbb{R}^d} \). Gesucht ist eine Distribution, die für jeden Punkt \( x \in \mathbb{R}^d \) durch \( F_x \) lokal gut approximiert wird. Wir stellen das \emph{Reconstruction Theorem} vor, welche die Existenz solch einer Distribution sichert und die Eindeutigkeit in bestimmten Fällen. 

Das Problem ist von großer Bedeutung für die Behandlung stochastischer Differentialgleichungen; genauer ist es das zentrale Theorem der Theorie der \emph{Regularity Structures} von Martin Hairer, welche auch in diesem Zusammenhang zum ersten Mal bewiesen wurde. Wir geben einen alternativen Zugang zum Reconstruction Theorem, die ohne Regularity Structures auskommt. Stattdessen betten wir das Reconstruction Theorem in die Theorie der Distributionen ein. Mit elementaren Techniken beweisen wir das Theorem. Wir erhalten am Ende ein mächtiges Theorem der Analysis, welche nicht nur für die stochastische Analysis von Interesse ist.

Als Anwendung des Reconstruction Theorems führen wir negative Hölderräume ein und beweisen das Sewing Lemma, ein wichtiges Hilfsmittel in der Theorie der rauen Pfade. Das Sewing Lemma wird oft als das eindimensionale Analogon des Reconstruction Theorems betrachtet. Wir überprüfen diese Aussage.

\end{alwayssingle}

% (This file is included by thesis.tex; you do not latex it by itself.)

\begin{abstract}

% The text of the abstract goes here.  If you need to use a \section
% command you will need to use \section*, \subsection*, etc. so that
% you don't get any numbering.  You probably won't be using any of
% these commands in the abstract anyway.

We give a self-contained and elementary proof of Hairer's Reconstruction Theorem using only the theory of distributions. We introduce an optimal condition for the Reconstruction Theorem called coherence  

Using the Reconstruction Theorem, we characterize negative Hölder

\end{abstract}


\begin{frontmatter}

% You can delete the \clearpage lines if you don't want these to start on
% separate pages.

\tableofcontents
\clearpage
%\listoffigures
%\clearpage
%\listoftables


\end{frontmatter}

\pagestyle{headings}

% (Optional) \part{First Part}

\chapter{Introduction} 

The Reconstruction Theorem allows to construct a distribution $f$ from a family of distributions $(F_x)_{x \in \mathbb{R}^d}$ such that $f$ is locally well-approximated by $F_x$ around $x \in \mathbb{R}^d$. It may be viewed as a converse to Taylor's Theorem if $F_x$ is a Taylor polynomial. However, this view fails to capture the true importance. The Reconstruction Theorem is the most fundamental theorem in the theory of regularity structures --- a novel theory proposed by Hairer \cite{hairer2014theory} that provides a robust solution theory to many ill-posed stochastic partial differential equations. In fact, the theory of regularity structures was so groundbreaking that Hairer was awarded the Fields medal for his \emph{``creation of regularity structures''} in 2014 \cite{FieldsMedalHairer}. It is the Reconstruction Theorem, a purely deterministic tool from analysis, that enabled the theory of regularity structures.

The original proof of the Reconstruction Theorem relied heavily on wavelet analysis; since then numerous proofs were published \cite{hairer2017reconstruction, otto2019quasilinear, gubinelli2015paracontrolled, martin2020littlewood, singh2018elementary}. All of these proofs required profound knowledge of regularity structures, rough path theory or paracontrolled distributions. A concise treatment of the Reconstruction Theorem for a broader audience was not available until Caravenna and Zambotti gave a proof in 2020, which only used elementary distribution theory \cite{caravenna2021hairer}. 

The aim of this thesis is to give a self-contained proof of the Reconstruction Theorem. Hence, we mimic the proof by Caravenna and Zambotti, which requires no prior knowledge. We hope that this thesis introduces the Reconstruction Theorem to a broader audience. Since the Reconstruction Theorem is a purely analytical tool, it may find applications outside regularity structures or stochastic partial differential equations. 

The thesis is structured in the following way: Chapter \ref{chapter:notation} introduces notation. Chapter \ref{chapter:distributions} gives a concise overview of the theory of distributions. In Chapter \ref{chapter:reconstruction} we state the Reconstruction Theorem with its assumptions. This leads to the central notion of \emph{coherence}, an optimal assumption coined by Caravenna and Zambotti. In Chapter \ref{chapter:general-proof}, \ref{chapter:proof-gamma-positive} and \ref{chapter:proof-gamma-negative} we prove the Reconstruction Theorem. We split the proof into three parts because we consider two cases: $\gamma > 0$ and $\gamma \leq 0$, where $\gamma$ is a parameter occuring in the Reconstruction Theorem. A major part of the proof holds for all $\gamma \in \mathbb{R}$ and is presented in Chapter \ref{chapter:general-proof}. We continue the proof for $\gamma > 0$ in Chapter \ref{chapter:proof-gamma-positive}, and the proof for $\gamma \leq 0$ in Chapter \ref{chapter:proof-gamma-negative}. In Chapter 6 we give an application of the Reconstruction Theorem, where we extend Hölder spaces to negative exponents. In Chapter 7, we end the thesis with a discussion of the relationship between the Reconstruction Theorem and the Sewing Lemma, another analytical tool from Rough Path theory that is often considered as the one-dimensional analogue of the Reconstruction Theorem.

% ------------------------------------------

\section{Notation}\label{chapter:notation}

The \emph{open ball} $B(x_0, r)$  centered around $x_0 \in \mathbb{R}^d$ with radius $r > 0$ is defined as $B(x_0,r) = \{ x \in \mathbb{R}^d : |x - x_0| \leq r \}$, where $|\cdot|$ is the \emph{standard Euclidean norm}. We write $\enlarg{A}{\epsilon}$ to enlarge a set $A \subset \mathbb{R}^d$ by $\epsilon$: 
\begin{align*}
    \enlarg{A}{\epsilon} \coloneqq A + B(0, \epsilon) \coloneqq \{ x \in \mathbb{R}^d : |x - a| \leq \epsilon \text{ for some $a \in A$} \}.
\end{align*}

The \emph{multi-index notation} makes many theorems for functions of severable indeterminates appear as if there is only one indeterminate. A \emph{multi-index} $k = (k_1,...,k_d) \in \mathbb{N}^d_0$ is a $d$-tuple of non-negative integers. The \emph{length} of $k$ is defined as $|k| = k_1 + ... + k_d$. We define for all $x, k,l \in \mathbb{R}^d$ and $f: \mathbb{R}^d \to \mathbb{R}$
\begin{gather*}
    x^k = x_1^{k_1} \cdot ... \cdot x_d^{k_d}, \qquad
    k! = k_1! k_2! ... k_d!, \qquad
    {k \choose k} = {k_1 \choose l_1} ... {k_d \choose l_d} = \frac{k!}{l! (k - l)!}.
\end{gather*}
A polynomial $f(x)$ with real coefficients $\alpha_k \in \mathbb{R}$ of degree $m \in \mathbb{N}_0$ in $d$ indeterminates can be written as 
\begin{align*}
    f(x) = \sum\limits_{|k| \leq m} \alpha_{k}x^k  \quad \text{and} \quad
    \partial^k f(x) = \partial^{k_1}_{1} ... \partial^{k_d}_{d} f(x).
\end{align*}
We say 
\begin{itemize}
    \item $f \in C$ or $f \in C^0$ if $f$ is continuous,
    \item $f \in C^k$ if $f$ is {$k$-times continuously differentiable} for $k \in \mathbb{N}$, and
    \item $f$ is {smooth} if $f \in C^\infty$.
\end{itemize}
We define the \emph{$C^k$-norm} as $\lVert f \rVert_{C^k} = \max\limits_{|i| \leq k} \lVert \partial^i f \rVert_\infty$ where $\lVert f \rVert_{\infty} = \sup_{x \in \mathbb{R}^d} |f(x)|$. Next, we state classical results from analysis without their proofs (the proofs can be found in every standard analysis book, e.g. \cite{MR0055409}).
\begin{theorem}[Taylor's Theorem]
    Let $f \in C^{k}(B(x_0, r))$ and $k \in \mathbb{N}^d_0$. Then, we have $f(x) = \sum\limits_{|j| \leq k}\partial^{j} f(x_0) \frac{(x-x_0)^j}{j!} + R(x)$ with $\frac{R(x)}{|x-x_0|^{k}} \to 0$ as $x \to x_0$ for all $x \in B(x_0, r)$.
\end{theorem}

\begin{theorem}[Leibniz Rule]\label{theorem:leibniz}
    The Leibniz rule is a generalization of the product rule:
    \begin{align*}
        \partial^\alpha(fg) = \sum_{\beta \leq \alpha} {\alpha \choose \beta} (\partial^\beta f) (\partial^{\alpha - \beta} g).
    \end{align*}
\end{theorem}

\begin{theorem}[Mean Value Inequality]\label{mean-value-inequality}
    Let $f: G \to \mathbb{R}$ be differentiable, where $G$ is an open convex subset of $\mathbb{R}^n$. Let $a,b \in G$. Then, $|f(b) - f(a)| \leq \sup\limits_{x \in \overline{ab}} |f'(x)| \, |b-a|$, where $f'(x) = \begin{pmatrix}
        \partial_{x_1}f & ... & \partial_{x_d}f
    \end{pmatrix}$ is the gradient of $f$. 
\end{theorem}

Later, we will need certain functions to \emph{annihilate} monomials. This means the following.

\begin{definition}[Annihilation of Monomials]
    A function $g: \mathbb{R}^d \to \mathbb{R}$ is said to \emph{annihilate monomials} of degree $j \in \mathbb{N}$ if for all $n \in \mathbb{N}^d_0$ with $|n| = j$ we have
    \begin{align*}
        \int_{\mathbb{R}^d} y^n g (y) \, \mathrm{d}y = 0.
    \end{align*}
\end{definition}

For later applications of the Reconstruction Theorem, we introduce the space of \emph{locally $\alpha$-Hölder functions} denoted by $\mathcal{C}^\alpha$, $\alpha > 0$ . 

\begin{definition}[Locally $\alpha$-Hölder Functions]
    Let $\alpha > 0$. Let $\varphi: \mathbb{R}^d \to \mathbb{R}$. We say $\varphi \in \mathcal{C}^\alpha$ if
\begin{itemize}
    \item $\varphi \in C^r$ for $r = \max\left\{ n \in \mathbb{N}_0 : n < \alpha \right\}$, and
    \item there exists a constant $C < \infty$ such that $|\varphi(y) - F_x(y)| \leq C |y-x|^\alpha$ uniformly for all $x,y$ in compact sets. Here, $F_x$ is the Taylor polynomial of $\varphi$ of order $r$ at $x$.   
\end{itemize}
\end{definition}









% ------------------------------------------









\section{Theory of Distributions}\label{chapter:distributions}


In distribution theory one is interested in... 

The beauty of \cite{caravenna2021hairer}, on which this bachelor thesis is based, lies within the fact that the Reconstruction Theorem can be stated in terms of elementary distribution theory without the need of regularity structures. 

The first concept we will encounter is that of a \emph{support} of a function $\varphi: \mathbb{R}^d \to \mathbb{R}$, which is defined as $\mathrm{supp}(\varphi) = \overline{\left\{ x \in \mathbb{R}^d : \varphi(x) \neq 0 \right\}}$. 

\begin{definition}[Test Function]
    \emph{Test functions} $\varphi: \mathbb{R}^d \to \mathbb{R}$ are smooth functions that have compact support. The \emph{space of test functions} $\mathcal{D}$ is the set that contains all test functions:
    \begin{align*}
        \mathcal{D} = \mathcal{D}(\mathbb{R}^d) &= \left\{ \varphi \in C^\infty(\mathbb{R}^d) : \text{$\mathrm{supp}(\varphi)$ is compact} \right\}, \\
        \mathcal{D}(A) &= \left\{ \varphi \in \mathcal{D} : \mathrm{supp}(\varphi) \subset A \right\} \qquad \text{for any subset $A \subset \mathbb{R}^d$.}
    \end{align*}
\end{definition}

A standard example for a test function is the \emph{bump function}:
\begin{align*}
    \mathcal{B}(x) = \begin{cases}
        \exp{\left( -\frac{1}{1 - |x|^2} \right)}, \quad & |x| < 1, \\
        0, & \text{otherwise}.
    \end{cases}
\end{align*}
Clearly, the bump function has compact support in $B(0,1)$. The proof that it is smooth can be found in any standard analysis book, e.g. see (22.2) in \cite{Forster_2016}.

The main object of study of Distribution Theory is unsuprisingly a \emph{distribution}.
\begin{definition}[Distribution]
A functional $u: \mathcal{D} \to \mathbb{R}$ is called a \emph{distribution} if $u$ is linear, and if for every compact set $K \subset \mathbb{R}^d$ there exist $r \in \mathbb{N}_0$ and $C < \infty$ such that 
\begin{align*}
    |u(\varphi)| \leq C \lVert\varphi\rVert_{C^r}, \quad \forall \varphi \in \mathcal{D}(K).
\end{align*}
The \emph{space of all distributions} is denoted $\mathcal{D}' = \left\{ u: \mathcal{D} \to \mathbb{R} \, | \, \text{$u$ is a distribution} \right\}$.
\end{definition}

Next, we give \emph{convergence in $\mathcal{D}$} a meaning.

\begin{definition}[Convergence]
    Let $(\varphi_j)$ be a sequence in $\mathcal{D}$ and $\varphi \in \mathcal{D}$. We say $
        \varphi_j \to \varphi  \text{ in $\mathcal{D}$}
    $
    if 
    \begin{enumerate}[label=(\roman*)]
        \item there exists a compact set $K \in \mathbb{R}^d$ such that $\mathrm{supp}(\varphi)$ and $\mathrm{supp}(\varphi_j)$ are contained in $K$ for all $j$, and 
        \item $\lVert \varphi_j - \varphi \rVert_{C^r} \to 0$ as $j \to \infty$ for all $r \in \mathbb{N}_0$.
    \end{enumerate} 
\end{definition}

This allows us to give an alternative characterization of distributions: a distribution is a linear functional that is \emph{continuous}.

\begin{lemma}
    Let $u: \mathcal{D} \to \mathbb{R}$ be a linear functional. Then, $u$ is a {distribution} if and only if $\varphi_j \to \varphi$ in $\mathcal{D}$ implies $u(\varphi_j) \to u(\varphi)$ for all test functions $\varphi_j$ and $\varphi$.
\end{lemma}

\begin{proof}
    Let $u$ be a distribution and $\varphi_j \to \varphi$ in $\mathcal{D}$. Then, there exist $C$ and $r$ such that $|u(\varphi_j -\varphi)| \leq C \lVert \varphi_j - \varphi \rVert_{C^r} \to 0$ as $j \to \infty$ . By linearity, it follows $u(\varphi_j) \to u(\varphi)$.

    For the converse direction, we argue by contradiction. Let $\varphi_j \to \varphi$ in $\mathcal{D}$ imply $u(\varphi_j) \to u(\varphi)$ for all test functions $\varphi_j$ and $\varphi$. Assume, there is a compact set $K \subset \mathbb{R}^d$ such that for all $r \in \mathbb{N}_0$ and $C < \infty$ the inequality $|u(\varphi)| \leq C \lVert\varphi\rVert_{C^r}$ is violated for some $\varphi \in \mathcal{D}$. Then, we can find $\varphi_n$ with $|u(\varphi_n)| > n \lVert \varphi_n \rVert_{C^n}$ for all $n \in \mathbb{N}$. Next, we define $\phi_n \coloneqq \frac{\varphi_n}{n \lVert \varphi_n \rVert_{C^n}}$. So, we get $u(\phi_n) > 1$. However, $\phi_n \to 0$ in $\mathcal{D}$ as $n \to \infty$ because $\lVert \phi_n \rVert_{C^r} \leq \frac{1}{n}$ for all $n \geq r$.
\end{proof}

Quite often, we are given a test function $\varphi$, and we would like to construct a sequence $(\varphi_j)$ such that $\varphi_j \to \varphi  \text{ in $\mathcal{D}$}$. \emph{Mollifiers} are an indispensable tool for constructing such sequences, which we will use throughout the paper. First, we need to scale and translate arbitrary functions $\varphi: \mathbb{R}^d \to \mathbb{R}$:
\begin{align*}
    \varphi^\epsilon_y(x) \coloneqq \frac{1}{\epsilon^d} \, \varphi\left(\frac{x-y}{\epsilon}\right), \quad \varphi^\epsilon(x) \coloneqq \varphi^\epsilon_0(x), \quad \varphi_y(x) \coloneqq \varphi^1_y(x).
\end{align*} 

Given a compactly supported function $\rho: \mathbb{R}^d \to \mathbb{R}$ that integrates to one, a family of scaled and translated versions of $\rho$ is called a \emph{mollifier}.

\begin{definition}[Mollifier]
    Let $\rho: \mathbb{R}^d \to \mathbb{R}$ be a function with compact support and $\int_{\mathbb{R}^d} \rho(x) \, \mathrm{d}x = 1$. Then, the family of scaled functions $(\rho^\epsilon)_{\epsilon > 0}$ is called a \emph{mollifier}.
\end{definition}

Constructing a sequence $(\varphi_j)$ such that $\varphi_j \to \varphi$ in $\mathcal{D}$ becomes an easy task with the help of mollifiers and \emph{convolutions}. The \emph{convolution} of two functions $f,g \in {L}^1(\mathbb{R}^d)$ is defined as $(f*g)(x) = \int_{\mathbb{R}^d} f(x - y)g(y) \, \mathrm{d}y$. 

\begin{lemma}
    Let $f,g \in {L}^1(\mathbb{R}^d)$. Then, 
    \begin{enumerate}
        \item $f*g$ is well-defined almost everywhere, and 
        \item $f*g \in {L}^1(\mathbb{R}^d)$.
    \end{enumerate}
\end{lemma}
\begin{proof}
    We can safely assume $f$ and $g$ to be representatives of the equivalence classes, because we  treat $\int f(x) \; \mathrm{d}x$ as Lebesgue integrals and these integrals are independent of the chosen representatives.

    First, we check that $(x,y) \mapsto f(x-y)g(y) \in L^1(\mathbb{R}^d \times \mathbb{R}^d)$ in order to apply Fubini. From Tonelli's theorem and the translation invariance of the Lebesgue integral we obtain
    \begin{align*}
        \int_{\mathbb{R}^d \times \mathbb{R}^d}|f(x-y)g(y)| \, \mathrm{d}(x,y) = \int_{\mathbb{R}^d}\int_{\mathbb{R}^d} |f(x-y)g(y)| \, \mathrm{d}x \, \mathrm{d}y = \lVert f \rVert_{1} \lVert g \rVert_{1} < \infty.
    \end{align*}

    Then, by Fubini's theorem we obtain that $y \mapsto f(x-y)g(y)$ is integrable for almost every $x \in \mathbb{R}^d$. Thus, $(f*g)(x) = \int f(x-y)g(y) \, \mathrm{d}y$ is well-defined for almost every $x \in \mathbb{R}^d$. Also by Fubini, $f*g$ is integrable (if we assign zero in the points of the null set where it is not defined).
\end{proof}
Additionally, the proof also tells us that
\begin{align}
    \lVert f*g \rVert_1 \leq \lVert f \rVert_{1} \lVert g \rVert_{1}.
\end{align}

When we convolute a test function $\varphi \in \mathcal{D}$ against a mollifier $(\rho^\epsilon)$, we obtain a sequence $(\varphi * \rho^\epsilon) \subset \mathcal{D}$ that converges to $\varphi$ in $\mathcal{D}$ as $\epsilon \to 0$.  

\begin{lemma}\label{mollifier-lemma}
    Let $(\rho^\epsilon)$ be a mollifier. For all test functions $\varphi \in \mathcal{D}$, we have
    \begin{enumerate}
        \item $\varphi * \rho^{\epsilon} \in \mathcal{D}$ for all $\epsilon > 0$, and
 
        \item $\varphi * \rho^\epsilon \to \varphi  \text{ in $\mathcal{D}$ as $\epsilon \to 0$}$.
    \end{enumerate}
\end{lemma}

\begin{proof}
    We first show $\frac{\partial (\varphi * \rho^{\epsilon})}{\partial x_j} = \left(\frac{\partial \varphi}{\partial x_j}\right) * \rho^{\epsilon}$. Applying that rule inductively implies $\varphi * \rho^{\epsilon} \in C^{\infty}$. Let $e_j$ denote the $j$-th unit vector. Consider the difference quotient
    \begin{align*}
        \frac{(\varphi * \rho^\epsilon)(x + te_j) - (\varphi * \rho^\epsilon)(x)}{t} = \int \frac{\varphi(x+te_j - y) - \varphi(x-y)}{t}\rho^{\epsilon}(y) \mathrm{d}y.
    \end{align*}
    By the mean value inequality (Theorem \ref{mean-value-inequality}), we can bound
    \begin{align*}
        \frac{\varphi (x + te_j - y) - \varphi (x-y)}{t} \leq \max_{\xi \in [0,t]}\frac{\partial}{\partial x_j}\varphi(x - y + \xi e_j) \leq C
    \end{align*}
    for some $C > 0$ since $\varphi$ is continuously differentiable. Thus, we found an integrable function that dominates $\frac{\varphi(x+te_j - y) - \varphi(x-y)}{t}\rho^{\epsilon}(y)$, and we can apply Lebesgue's dominated convergence theorem to obtain $\frac{\partial (\varphi * \rho^{\epsilon})}{\partial x_j} = \left(\frac{\partial \varphi}{\partial x_j}\right) * \rho^{\epsilon}$.

    The convolution $\varphi * \rho^{\epsilon}$ has compact support because $\varphi$ and $\rho^{\epsilon}$ have also compact support. Therefore, we conclude $\varphi * \rho^{\epsilon} \in \mathcal{D}$.

    Now, we show $\varphi * \rho^\epsilon \to \varphi  \text{ in $\mathcal{D}$ as $\epsilon \to 0$}$. First, there exists a compact set $K$ that contains the support of $\varphi * \rho^\epsilon$ for all $\epsilon \in (0,1)$ because $\varphi$ and $\rho$ have compact support. Let the support of $\rho$ be contained in $B(0,r)$ for some $r > 0$. For any multi-index $k$, $\epsilon \in (0,1)$ and $x \in K$, we have
    \begin{align*}
        \partial^k(\varphi * \rho^\epsilon) (x) - \partial^k\varphi(x) = \int (\partial^k \varphi(x - y) - \partial^k \varphi(x)) \, \rho^{\epsilon}(y) \mathrm{d}y
    \end{align*}
    because $\int \rho^{\epsilon}(y) \, \mathrm{d} y = \int \rho(y) \, \mathrm{d} y = 1$. Hence, 
    \begin{align*}
        |\partial^k(\varphi * \rho^\epsilon) (x) - \partial^k\varphi(x)| 
        &\leq \int |\partial^k \varphi(x - y) - \partial^k \varphi(x)| \, |\rho^{\epsilon}(y)| \mathrm{d}y \\
        &\Downarrow \text{Mean value theorem} \\
        & \leq \max_{z \in K_r} |\partial^{k+1}\varphi(z)|  \int  |y| \, |\rho^{\epsilon}(y)| \mathrm{d}y \\
        &\Downarrow \text{Substitution: } y = \epsilon \tilde y \\
        &= \max_{z \in K_r} |\partial^{k+1}\varphi(z)| \, \epsilon  \underbrace{\int  |\tilde y| \, |\rho(\tilde y)| \mathrm{d}\tilde y}_{< \infty}.
    \end{align*}
    As $\epsilon \to 0$, we see that $\sup_{x \in K} |\partial^k(\varphi * \rho^\epsilon) (x) - \partial^k\varphi(x)| \to 0$. 
\end{proof}

Next, we can study the effect of applying a distribution $F$ on $\varphi * \rho^\epsilon$, where $\rho$ is a mollifier. We know that $\varphi * \rho^{\epsilon} \to \varphi$ in $\mathcal{D}$ as $\epsilon \to 0$. Then, $F(\varphi * \rho^{\epsilon}) \to F(\varphi)$ because $F$ is a distribution. $F(\varphi * \rho^{\epsilon})$ is also known as a \emph{mollified distribution}. We have the following result for mollified distributions.

\begin{lemma}
    Let $F \in \mathcal{D}'$. Let $\varphi \in \mathcal{D}$ and $g$ be locally integrable and compactly supported. Then,
    \begin{align}\label{lemma:mollified-distribution}
        F(\varphi * g) = \int_{\mathbb{R}^d} F(\varphi_y)g(y) \, \mathrm{d}y.
    \end{align}
\end{lemma}

\begin{proof}
    This follows from linearity of $F$ and Riemann sum approximation; see Lemma 4.12 in \cite{JanKristensenDistribution}.
    %\begin{align*}
    %    F(\varphi * g) = F\left(\int_{\mathbb{R}} \varphi(x-y) g(y) \, \mathrm{d}y\right) = F\left( 
    %        \lim_{N \to \infty} \sum^N_{k=0} \varphi(x - \xi_k)g(\xi_k)(\xi_{k+1} - \xi)
    %     \right)
    %\end{align*}
\end{proof}

If we let $g = \rho$, we obtain the crucial relationship $\int F(\varphi_y)\rho^{\epsilon}(y) \, \mathrm{d}y \to F(\varphi)$ as $\epsilon \to 0$; or equivalently
\begin{align}\label{eq:starting-point}
    \int F(\rho_y^\epsilon) \varphi(y)\, \mathrm{d}y \to F(\varphi) \quad \text{ as } \epsilon \to 0.
\end{align}
This is \emph{key} to proving the Reconstruction Theorem. 

For future reference, we state the following corollary.
\begin{corollary}\label{cor:minosokoad}
    Let $F \in \mathcal{D}'$ and $g,h, \psi \in \mathcal{D}$. Then, 
    \begin{align*}
        \int_{\mathbb{R}^d} F((g*h)_z) \psi(z)\, \mathrm{d}z
    = \iint_{\mathbb{R}^{d \times d}} F(g_y)  h(y-z) \psi(z) \, \mathrm{d}y\, \mathrm{d}z.
    \end{align*}
\end{corollary}

\begin{proof}
    Note that $(g*h)_z(x) = (g*h)(x - z) = \int g(y)h(x-z-y) \, \mathrm{d}y = (g*h_z)(x)$. Using \eqref{lemma:mollified-distribution} we get $F((g*h)_z) = F(g*h_z) = \int F(g_y) h_z(y) \, \mathrm{d}y$. This proves the corollary.
\end{proof}

% 

\chapter{Distribution Theory}

In distribution theory one is interested in... 

The beauty of \cite{caravenna2021hairer}, on which this bachelor thesis is based, lies within the fact that the Reconstruction Theorem can be stated in terms of elementary distribution theory without the need of regularity structures. 

The first concept we will encounter is that of a \emph{support} of a function \(\varphi: \mathbb{R}^d \to \mathbb{R}\), which is defined as \(\mathrm{supp}(\varphi) = \overline{\left\{ x \in \mathbb{R}^d : \varphi(x) \neq 0 \right\}}\). 

\begin{definition}[Test Function]
    \emph{Test functions} \(\varphi: \mathbb{R}^d \to \mathbb{R}\) are smooth functions that have compact support. The \emph{space of test functions} \(\mathcal{D}\) is the set that contains all test functions:
    \begin{align*}
        \mathcal{D} = \mathcal{D}(\mathbb{R}^d) &= \left\{ \varphi \in C^\infty(\mathbb{R}^d) : \text{\(\mathrm{supp}(\varphi)\) is compact} \right\}, \\
        \mathcal{D}(A) &= \left\{ \varphi \in \mathcal{D} : \mathrm{supp}(\varphi) \subset A \right\} \qquad \text{for any subset \(A \subset \mathbb{R}^d\).}
    \end{align*}
\end{definition}

A standard example for a test function is the \emph{bump function}:
\begin{align*}
    \mathcal{B}(x) = \begin{cases}
        \exp{\left( -\frac{1}{1 - |x|^2} \right)}, \quad & |x| < 1, \\
        0, & \text{otherwise}.
    \end{cases}
\end{align*}
Clearly, the bump function has compact support in \(B(0,1)\). The proof that it is smooth can be found in any standard analysis book, e.g. see (22.2) in \cite{Forster_2016}.

\emph{Distributions} are our main objects in Distribution Theory.
\begin{definition}[Distribution]
A functional \(u: \mathcal{D} \to \mathbb{R}\) is called a \emph{distribution} if \(u\) is linear, and if for every compact set \(K \subset \mathbb{R}^d\) there exist \(r \in \mathbb{N}_0\) and \(C < \infty\) such that 
\begin{align*}
    |u(\varphi)| \leq C \lVert\varphi\rVert_{C^r}, \quad \forall \varphi \in \mathcal{D}(K).
\end{align*}
The \emph{space of all distributions} is denoted \(\mathcal{D}' = \left\{ u: \mathcal{D} \to \mathbb{R} \, | \, \text{\)u\( is a distribution} \right\}\).
\end{definition}

Next, we give \emph{convergence in \(\mathcal{D}\)} a meaning.

\begin{definition}[Convergence]
    Let \((\varphi_j)\) be a sequence in \(\mathcal{D}\) and \(\varphi \in \mathcal{D}\). We say $
        \varphi_j \to \varphi  \text{ in \(\mathcal{D}\)}
    $
    if 
    \begin{enumerate}[label=(\roman*)]
        \item there exists a compact set \(K \in \mathbb{R}^d\) such that \(\mathrm{supp}(\varphi)\) and \(\mathrm{supp}(\varphi_j)\) are contained in \(K\) for all \(j\), and 
        \item \(\lVert \varphi_j - \varphi \rVert_{C^r} \to 0\) as \(j \to \infty\) for all \(r \in \mathbb{N}_0\).
    \end{enumerate} 
\end{definition}

This allows us to give an alternative characterization of distributions: a distribution is a linear functional that is \emph{continuous}.

\begin{lemma}
    Let \(u: \mathcal{D} \to \mathbb{R}\) be a linear functional. Then, \(u\) is a {distribution} if and only if \(\varphi_j \to \varphi\) in \(\mathcal{D}\) implies \(u(\varphi_j) \to u(\varphi)\) for all test functions \(\varphi_j\) and \(\varphi\).
\end{lemma}

\begin{proof}
    Let \(u\) be a distribution and \(\varphi_j \to \varphi\) in \(\mathcal{D}\). Then, there exist \(C\) and \(r\) such that \(|u(\varphi_j -\varphi)| \leq C \lVert \varphi_j - \varphi \rVert_{C^r} \to 0\) as \(j \to \infty\) . By linearity, it follows \(u(\varphi_j) \to u(\varphi)\).

    For the converse direction, we argue by contradiction. Let \(\varphi_j \to \varphi\) in \(\mathcal{D}\) imply \(u(\varphi_j) \to u(\varphi)\) for all test functions \(\varphi_j\) and \(\varphi\). Assume, there is a compact set \(K \subset \mathbb{R}^d\) such that for all \(r \in \mathbb{N}_0\) and \(C < \infty\) the inequality \(|u(\varphi)| \leq C \lVert\varphi\rVert_{C^r}\) is violated for some \(\varphi \in \mathcal{D}\). Then, we can find \(\varphi_n\) with \(|u(\varphi_n)| > n \lVert \varphi_n \rVert_{C^n}\) for all \(n \in \mathbb{N}\). Next, we define \(\phi_n \coloneqq \frac{\varphi_n}{n \lVert \varphi_n \rVert_{C^n}}\). So, we get \(u(\phi_n) > 1\). However, \(\phi_n \to 0\) in \(\mathcal{D}\) as \(n \to \infty\) because \(\lVert \phi_n \rVert_{C^r} \leq \frac{1}{n}\) for all \(n \geq r\).
\end{proof}

Quite often, we are given a test function \(\varphi\), and we would like to construct a sequence \((\varphi_j)\) such that \(\varphi_j \to \varphi  \text{ in \)\mathcal{D}\(}\). \emph{Mollifiers} are an indispensable tool for constructing such sequences, which we will use throughout the paper. First, we need to scale and translate arbitrary functions \(\varphi: \mathbb{R}^d \to \mathbb{R}\):
\begin{align*}
    \varphi^\epsilon_y(x) \coloneqq \frac{1}{\epsilon^d} \, \varphi\left(\frac{x-y}{\epsilon}\right), \quad \varphi^\epsilon(x) \coloneqq \varphi^\epsilon_0(x), \quad \varphi_y(x) \coloneqq \varphi^1_y(x).
\end{align*} 

Given a compactly supported function \(\rho: \mathbb{R}^d \to \mathbb{R}\) that integrates to one, a family of scaled and translated versions of \(\rho\) is called a \emph{mollifier}.

\begin{definition}[Mollifier]
    Let \(\rho: \mathbb{R}^d \to \mathbb{R}\) be a function with compact support and \(\int_{\mathbb{R}^d} \rho(x) \, \mathrm{d}x = 1\). Then, the family of scaled functions \((\rho^\epsilon)_{\epsilon > 0}\) is called a \emph{mollifier}.
\end{definition}

Constructing a sequence \((\varphi_j)\) such that \(\varphi_j \to \varphi\) in \(\mathcal{D}\) becomes an easy task with the help of mollifiers and \emph{convolutions}. The \emph{convolution} of two functions \(f,g \in {L}^1(\mathbb{R}^d)\) is defined as \((f*g)(x) = \int_{\mathbb{R}^d} f(x - y)g(y) \, \mathrm{d}y\). 

\begin{lemma}
    Let \(f,g \in {L}^1(\mathbb{R}^d)\). Then, 
    \begin{enumerate}
        \item \(f*g\) is well-defined almost everywhere, and 
        \item \(f*g \in {L}^1(\mathbb{R}^d)\).
    \end{enumerate}
\end{lemma}
\begin{proof}
    We can safely assume \(f\) and \(g\) to be representatives of the equivalence classes, because we  treat \(\int f(x) \; \mathrm{d}x\) as Lebesgue integrals and these integrals are independent of the chosen representatives.

    First, we check that \((x,y) \mapsto f(x-y)g(y) \in L^1(\mathbb{R}^d \times \mathbb{R}^d)\) in order to apply Fubini. From Tonelli's theorem and the translation invariance of the Lebesgue integral we obtain
    \begin{align*}
        \int_{\mathbb{R}^d \times \mathbb{R}^d}|f(x-y)g(y)| \, \mathrm{d}(x,y) = \int_{\mathbb{R}^d}\int_{\mathbb{R}^d} |f(x-y)g(y)| \, \mathrm{d}x \, \mathrm{d}y = \lVert f \rVert_{1} \lVert g \rVert_{1} < \infty.
    \end{align*}

    Then, by Fubini's theorem we obtain that \(y \mapsto f(x-y)g(y)\) is integrable for almost every \(x \in \mathbb{R}^d\). Thus, \((f*g)(x) = \int f(x-y)g(y) \, \mathrm{d}y\) is well-defined for almost every \(x \in \mathbb{R}^d\). Also by Fubini, \(f*g\) is integrable (if we assign zero in the points of the null set where it is not defined).
\end{proof}
Additionally, the proof also tells us that
\begin{align}
    \lVert f*g \rVert_1 \leq \lVert f \rVert_{1} \lVert g \rVert_{1}.
\end{align}

When we convolute a test function \(\varphi \in \mathcal{D}\) against a mollifier \((\rho^\epsilon)\), we obtain a sequence \((\varphi * \rho^\epsilon) \subset \mathcal{D}\) that converges to \(\varphi\) in \(\mathcal{D}\) as \(\epsilon \to 0\).  

\begin{lemma}\label{mollifier-lemma}
    Let \((\rho^\epsilon)\) be a mollifier. For all test functions \(\varphi \in \mathcal{D}\), we have
    \begin{enumerate}
        \item \(\varphi * \rho^{\epsilon} \in \mathcal{D}\) for all \(\epsilon > 0\), and
 
        \item \(\varphi * \rho^\epsilon \to \varphi  \text{ in \)\mathcal{D}\( as \)\epsilon \to 0\(}\).
    \end{enumerate}
\end{lemma}

\begin{proof}
    We first show \(\frac{\partial (\varphi * \rho^{\epsilon})}{\partial x_j} = \left(\frac{\partial \varphi}{\partial x_j}\right) * \rho^{\epsilon}\). Applying that rule inductively implies \(\varphi * \rho^{\epsilon} \in C^{\infty}\). Let \(e_j\) denote the \(j\)-th unit vector. Consider the difference quotient
    \begin{align*}
        \frac{(\varphi * \rho^\epsilon)(x + te_j) - (\varphi * \rho^\epsilon)(x)}{t} = \int \frac{\varphi(x+te_j - y) - \varphi(x-y)}{t}\rho^{\epsilon}(y) \mathrm{d}y.
    \end{align*}
    By the mean value inequality (Theorem \ref{mean-value-inequality}), we can bound
    \begin{align*}
        \frac{\varphi (x + te_j - y) - \varphi (x-y)}{t} \leq \max_{\xi \in [0,t]}\frac{\partial}{\partial x_j}\varphi(x - y + \xi e_j) \leq C
    \end{align*}
    for some \(C > 0\) since \(\varphi\) is continuously differentiable. Thus, we found an integrable function that dominates \(\frac{\varphi(x+te_j - y) - \varphi(x-y)}{t}\rho^{\epsilon}(y)\), and we can apply Lebesgue's dominated convergence theorem to obtain \(\frac{\partial (\varphi * \rho^{\epsilon})}{\partial x_j} = \left(\frac{\partial \varphi}{\partial x_j}\right) * \rho^{\epsilon}\).

    The convolution \(\varphi * \rho^{\epsilon}\) has compact support because \(\varphi\) and \(\rho^{\epsilon}\) have also compact support. Therefore, we conclude \(\varphi * \rho^{\epsilon} \in \mathcal{D}\).

    Now, we show \(\varphi * \rho^\epsilon \to \varphi  \text{ in \)\mathcal{D}\( as \)\epsilon \to 0\(}\). First, there exists a compact set \(K\) that contains the support of \(\varphi * \rho^\epsilon\) for all \(\epsilon \in (0,1)\) because \(\varphi\) and \(\rho\) have compact support. Let the support of \(\rho\) be contained in \(B(0,r)\) for some \(r > 0\). For any multi-index \(k\), \(\epsilon \in (0,1)\) and \(x \in K\), we have
    \begin{align*}
        \partial^k(\varphi * \rho^\epsilon) (x) - \partial^k\varphi(x) = \int (\partial^k \varphi(x - y) - \partial^k \varphi(x)) \, \rho^{\epsilon}(y) \mathrm{d}y
    \end{align*}
    because \(\int \rho^{\epsilon}(y) \, \mathrm{d} y = \int \rho(y) \, \mathrm{d} y = 1\). Hence, 
    \begin{align*}
        |\partial^k(\varphi * \rho^\epsilon) (x) - \partial^k\varphi(x)| 
        &\leq \int |\partial^k \varphi(x - y) - \partial^k \varphi(x)| \, |\rho^{\epsilon}(y)| \mathrm{d}y \\
        &\Downarrow \text{Mean value theorem} \\
        & \leq \max_{z \in K_r} |\partial^{k+1}\varphi(z)|  \int  |y| \, |\rho^{\epsilon}(y)| \mathrm{d}y \\
        &\Downarrow \text{Substitution: } y = \epsilon \tilde y \\
        &= \max_{z \in K_r} |\partial^{k+1}\varphi(z)| \, \epsilon  \underbrace{\int  |\tilde y| \, |\rho(\tilde y)| \mathrm{d}\tilde y}_{< \infty}.
    \end{align*}
    As \(\epsilon \to 0\), we see that \(\sup_{x \in K} |\partial^k(\varphi * \rho^\epsilon) (x) - \partial^k\varphi(x)| \to 0\). 
\end{proof}

Next, we can study the effect of applying a distribution \(F\) on \(\varphi * \rho^\epsilon\), where \(\rho\) is a mollifier. We know that \(\varphi * \rho^{\epsilon} \to \varphi\) in \(\mathcal{D}\) as \(\epsilon \to 0\). Then, \(F(\varphi * \rho^{\epsilon}) \to F(\varphi)\) because \(F\) is a distribution. \(F(\varphi * \rho^{\epsilon})\) is also known as a \emph{mollified distribution}. We have the following result for mollified distributions.

\begin{lemma}
    Let \(F \in \mathcal{D}'\). Let \(\varphi \in \mathcal{D}\) and \(g\) be locally integrable and compactly supported. Then,
    \begin{align}\label{lemma:mollified-distribution}
        F(\varphi * g) = \int_{\mathbb{R}^d} F(\varphi_y)g(y) \, \mathrm{d}y.
    \end{align}
\end{lemma}

\begin{proof}
    This follows from linearity of \(F\) and Riemann sum approximation; see Lemma 4.12 in \cite{JanKristensenDistribution}.
    %\begin{align*}
    %    F(\varphi * g) = F\left(\int_{\mathbb{R}} \varphi(x-y) g(y) \, \mathrm{d}y\right) = F\left( 
    %        \lim_{N \to \infty} \sum^N_{k=0} \varphi(x - \xi_k)g(\xi_k)(\xi_{k+1} - \xi)
    %     \right)
    %\end{align*}
\end{proof}

If we let \(g = \rho\), we obtain the crucial relationship \(\int F(\varphi_y)\rho^{\epsilon}(y) \, \mathrm{d}y \to F(\varphi)\) as \(\epsilon \to 0\); or equivalently
\begin{align}\label{eq:starting-point}
    \int F(\rho_y^\epsilon) \varphi(y)\, \mathrm{d}y \to F(\varphi) \quad \text{ as } \epsilon \to 0.
\end{align}
This is \emph{key} to proving the Reconstruction Theorem. 

For future reference, we state the following corollary.
\begin{corollary}\label{cor:minosokoad}
    Let \(F \in \mathcal{D}'\) and \(g,h, \psi \in \mathcal{D}\). Then, 
    \begin{align*}
        \int_{\mathbb{R}^d} F((g*h)_z) \psi(z)\, \mathrm{d}z
    = \iint_{\mathbb{R}^{d \times d}} F(g_y)  h(y-z) \psi(z) \, \mathrm{d}y\, \mathrm{d}z.
    \end{align*}
\end{corollary}

\begin{proof}
    Note that \((g*h)_z(x) = (g*h)(x - z) = \int g(y)h(x-z-y) \, \mathrm{d}y = (g*h_z)(x)\). Using \eqref{lemma:mollified-distribution} we get \(F((g*h)_z) = F(g*h_z) = \int F(g_y) h_z(y) \, \mathrm{d}y\). This proves the corollary.
\end{proof}

\chapter{Coherence and the Reconstruction Theorem}\label{chapter:reconstruction}

The Reconstruction Theorem was originally stated in the context of regularity structures by Hairer~\cite{hairer2014theory}. Later, Caravenna and Zambotti~\cite{caravenna2021hairer} revisited the Reconstruction Theorem and embedded it into the theory of distributions. In this chapter, we closely follow their approach with the advantage being that it allows for an easily accessible and self-contained treatment of the Reconstruction Theorem.

\section{A Peek at the Reconstruction Theorem}\label{chapter:first-peek-at-reconstruction}

We confront ourselves with the following problem.

\textbf{Problem:} Given a family of distributions \({(F_x)}_{x \in \mathbb{R}^d}\), find a distribution \(f \in \mathcal{D'}\) that is locally well approximated by \(F_x\) around any point \(x \in \mathbb{R}^d\).

\vspace{0.2cm}

Think of \({ F= (F_x)}_{x \in \mathbb{R}^d}\) as a family of local approximations of an unknown distribution \(f\). Our goal in this chapter is to find an assumption under which finding \(f\) is possible. Sometimes, \( f \) is denoted by \( \mathcal{R}F \), and we say \( \mathcal{R}F \) is the reconstruction of the germ \( F \). The symbol \( \mathcal{R} \) is to be understood as a reconstruction operator, mapping germs to their reconstruction.

Let's start with the first novel definition by Caravenna and Zambotti~\cite{caravenna2021hairer}.
\begin{definition}[Germ]
    A family of distributions \({(F_x)}_{x \in \mathbb{R}^d}\) is called a \emph{germ} if for all test functions \(\psi \in \mathcal{D}\) the map \(x \mapsto F_x(\psi)\) is measurable.
\end{definition}

For our purpose, a germ \({F = (F_x)}_{x \in \mathbb{R}^d}\) is \emph{locally well approximated} by a distribution \( f = \mathcal{R}F \) if for any test function \(\psi \in \mathcal{D}, \int \psi(x)\, \mathrm{d}x \neq 0\) and compact set \(K \subset \mathbb{R}^d\) we have 
\begin{align}\label{peek:well-approximated}
    \lim_{\epsilon \to 0} |(f - F_x)(\psi^\epsilon_x)| = 0 \quad \text{uniformly for \(x \in K\) }.
\end{align} 

The Reconstruction Theorem states that \emph{under some assumption} we find a reconstruction \(  f = \mathcal{R}F \) of some germ \( F = (F_x)_{x \in \mathbb{R}^d} \) such that~\eqref{peek:well-approximated} holds. This is our conjecture.

\begin{conjecture}\label{peek:conjecture}
    Let \({F = (F_x)}_{x \in \mathbb{R}^d}\) be a germ that satisfies an assumption \emph{\texttt{???}} that we do not know yet. Let \( \gamma > 0 \). Then, there exists a reconstruction \(f \in \mathcal{D}'\) such that for every test function \(\psi \in \mathcal{D}\) there exists \(C < {\infty}\) with
    \begin{gather}\label{peek:eq:conjecture}
        |(f-F_x)(\psi^\epsilon_x)| \leq C \epsilon^{\gamma} \\
        \text{uniformly for \(x\) in compact sets and \( \epsilon \in (0,1] \)} \nonumber. % chktex 9
    \end{gather}
    
\end{conjecture}

Any distribution \(f\) satisfying the above inequality approximates the germ \( F = (F_x)_{x \in \mathbb{R}^d} \) well in the sense of~\eqref{peek:well-approximated} because \(|(f-F_x)(\psi^\epsilon_x)| \leq C  \epsilon^{\gamma} \to 0\) as \(\epsilon \to 0\).

Our task is to find a suitable assumption for the conjecture. Let \({(F_x)}_{x \in \mathbb{R}^d}\) be a germ. Recall~\eqref{eq:starting-point} and set \( F = F_x \), then \( F_x(\psi * \rho^\epsilon) \to F_x(\psi) \) as \( \epsilon \to 0 \) because
\begin{align*}
    F_x(\psi * \rho^\epsilon) \overset{\eqref{lemma:mollified-distribution}}{=} \int F_x(\rho_y^\epsilon) \psi(y)\, \mathrm{d}y \to F_x(\psi) \quad \text{ as } \epsilon \to 0.
\end{align*}
This observation inspires us to replace \(F_x\) under the integral by \(F_y\) so that we obtain the map \(f_\epsilon: \psi \mapsto \int F_y(\rho_y^\epsilon) \psi(y)\, \mathrm{d}y\). The motivation for the newly constructed map \(f_{\epsilon}\) is that we hope that \( f_\epsilon \) converges to \( \mathcal{R}F \) as \( \epsilon \to 0 \).

\begin{definition}[Approximating Distribution]\label{def:approximating-distributions}
        Let \({(F_x)}_{x \in \mathbb{R}^d}\) be a germ and \( \rho \) a mollifier. Set \(\epsilon_n = 2^{-n}\) for \(n \in \mathbb{N}\). The \emph{approximating distribution} \(f_n \in \mathcal{D}'\) is defined as 
        \begin{align*}
                f_n: \psi \mapsto \int_{\mathbb{R}^d} F_y(\rho_y^{\epsilon_n}) \psi(y)\, \mathrm{d}y.
        \end{align*}
\end{definition}
Now, the assumption \(\texttt{???}\) should ensure that \(f_n\) converges. The limit \( \lim_{n \to \infty} f_n \) is then called the reconstruction of the germ \( (F_x)_{x \in \mathbb{R}^d} \); of course we must justify that \( \lim f_n \) is indeed a reconstruction. Ideally, this follows from \texttt{???}. Moving forward, we write 
\begin{align*}
    f_n = f_1 + \sum^{n-1}_{k=1}g_k, \quad \text{where } g_k \coloneqq f_{k+1} - f_k.
\end{align*}
The limit \(\lim_{n \to \infty} f_n\) exists if and only if \(\sum g_k\) converges. By definition, we have
\begin{align*}
        g_k(\psi) = \int_{\mathbb{R}^d} F_y(\rho_y^{\epsilon_{k+1}} - \rho_y^{\epsilon_k}) \psi(y)\, \mathrm{d}y.
\end{align*}
A smart choice of our mollifier \( \rho \) lets us write \( \rho^{\epsilon_{k+1}}_y - \rho^{\epsilon_k}_y = {(\hat \varphi^{\epsilon_k} * \check \varphi^{\epsilon_k})}_y\) for two nice test functions \( \hat \varphi \) and \( \check \varphi \). Finding such a mollifier with test functions \( \hat \varphi \) and \( \check \varphi \) later fills an entire chapter called \emph{Tweaking}. This is one of the most important steps in proving the Reconstruction Theorem. For now, we take the existence of \( \rho \) for granted.
By Corollary~\ref{cor:minosokoad} we write
\begin{align*}
    g_k(\psi) &= \int_{\mathbb{R}^d} F_z({(\hat \varphi^{\epsilon_k} * \check \varphi^{\epsilon_k})}_z) \psi(z)\, \mathrm{d}z \\
    &= \iint_{\mathbb{R}^{d \times d}} F_z(\hat \varphi^{\epsilon_k}_y) \check \varphi^{\epsilon_k}(y-z) \psi(z) \, \mathrm{d}y\, \mathrm{d}z \\
    &= \iint F_y(\hat \varphi^{\epsilon_k}_y) \check \varphi^{\epsilon_k}(y-z) \psi(z) \, \mathrm{d}y\, \mathrm{d}z 
    + \iint (F_z - F_y)(\hat \varphi^{\epsilon_k}_y) \check \varphi^{\epsilon_k}(y-z) \psi(z) \, \mathrm{d}y\, \mathrm{d}z.
\end{align*}
Remember that we want \(\sum g_k\) to converge; the crucial piece is \( F_z - F_y \). To find \texttt{???}, we let us guide by a closely related problem in another branch of mathematics: stochastic analysis. 

In stochastic analysis, one would like to make sense of an integral \(I_t\)  of the form \(I_t = \int^t_0 X_s \, \mathrm{d}Y_s\) where \(X_s\) and \(Y_s\) are paths of low regularity. Consider the following example: let \(G \in \mathcal{V}^p\) and \(F \in \mathcal{V}^q\) with \(\frac{1}{p} + \frac{1}{q} > 1\), where \(\mathcal{V}^j\) is the space of all functions with finite \(j\)-variation for \( j \in \left \{ p,q \right \} \). Then, there exists a canonical integration theory for this setting (the so-called \emph{Young regime}~\cite{Young1936AnIO}) such that \(I_t = \int^t_0 G \, \mathrm{d}F\) is defined. It is based on the approximation idea
\begin{align*}
    \int^t_s G \, \mathrm{d}F \approx G(s)(F(t) - F(s)) \eqqcolon A_{s,t} \quad \text{for very small \(|t-s|\)}.
\end{align*}
The \emph{Sewing Lemma}~\cite{GUBINELLI200486}, an analytical tool, which lets integrals of low regularity to be defined in a meaningful sense, allows us to sew the local approximations \(A_{s,t}\) together to obtain an integral as a Riemann-type sum
\begin{align}\label{sewing-lemma-integral}
    I_t = \int^t_0 G \, \mathrm{d}F \coloneqq \lim\limits_{|\pi| \to 0} \sum\limits_{i=0}^{\# \pi - 1} A_{t_i,t_{i+1}}
\end{align}
for arbitrary partitions\footnote{Here, a partition of \([0,t]\) is an ordered set \(\pi = \left \{ 0 = t_0 < t_1 < \cdots < t_k = t  \right \} \), \( \# \pi = k \) and \(|\pi| = \max\limits_{i=0, \ldots ,\# \pi - 1} |t_{i+1} - t_{i}|\).} \( \pi \) of \([0,t]\) with \(|\pi| \to 0\) as \(n \to 0\).
\begin{lemma}[Sewing Lemma~\cite{broux2021sewing}]\label{first-sewing-lemma}
    Let \(\gamma > 1\) and \( \Delta = \left \{ (s,t) :  0 \leq s \leq t \leq T\right \} \) for some fixed \(T > 0\). Let \(A: \Delta \to \mathbb{R}\) be a continuous function such that there exists \( C <\infty \) with
    \begin{gather}
        \delta A_{s,u,t} \coloneqq |A_{s,t} - A_{s,u} - A_{u,t}| \leq C {(\max \{|u-s|,|t-u|\})}^\gamma \label{sewing-lemma-condition}\\
        \text{uniformly for \(0 \leq s \leq u \leq t \leq T\)}. \nonumber
    \end{gather} 
    Then, there exists a unique function \(I: [0,T] \to \mathbb{R}\) and \(\tilde C < {\infty}\)  such that \(I_0 = 0\) and 
    \begin{gather*}
        |I_t - I_s - A_{s,t}| \leq \tilde C|t-s|^\gamma \\
        \text{uniformly over \(0 \leq s \leq t \leq T\).}
    \end{gather*}  
    Furthermore, \(I\) is the limit of Riemann-type sums as in~\eqref{sewing-lemma-integral}.  
\end{lemma}
The connection to the Reconstruction Theorem is established in the following way: From a distributional viewpoint, we want to give the integral \(I(\psi) = \int G \psi \, \mathrm{d}F\) a meaning for all test functions \( \psi \in \mathcal{D} \). We use the following approximation
\begin{align*}
    F_x(\psi)  \leadsto I(\psi), \quad \text{where } F_x(\psi) \coloneqq G(x)\int \psi \, \mathrm{d}F 
\end{align*}
If we allowed indicator functions as test functions \(\psi = 1_{[s,t]}\), we would get
\begin{align*}
    F_s(1_{[s,t]}) = G(s) \int^t_s \mathrm{d}F = G(s)(F(t) - F(s)) = A_{s,t}.
\end{align*}
If we further assumed that \(A_{s,t} = F_s(1_{[s,t]})\) satisfies~\eqref{sewing-lemma-condition}, we would have 
\begin{align*}
    (F_x - F_u)({(1_{[0,1]})}^{y-u}_u) = \frac{(F_x - F_u)(1_{[u,y]})}{y-u} &= \frac{(G(x) - G(u))(F(y) - F(u))}{y-u} \\
    &= \frac{\delta A_{x,u,y}}{y-u}\\
    &\leq C \frac{ {(|u-x| + |y-u|)}^\gamma}{y-u}
\end{align*} 
by the Sewing Lemma. Hence, the germ \({(F_x)}_{x \in \mathbb{R}^d}\) would satisfy
\begin{align*}
    (F_x - F_u)({(1_{[0,1]})}^{\epsilon}_u) \leq C \epsilon^{-1}{(|u-x| + \epsilon)}^\gamma
\end{align*}
for \(\epsilon = y-u\). 

Our reasoning premised around \( 1_{[s,t]} \) being a test function; clearly it is not smooth. So the argumentation fails. Nevertheless, it inspires us to define a property coined \emph{coherence}. A germ is called \emph{coherent} if \( (F_x)_{x \in \mathbb{R}^d} \) satisfies
\begin{gather}\label{pre-condition-coherence}
    |(F_z - F_y)(\varphi^\epsilon_y)| \leq C\epsilon^{a}{(|z-y| + \epsilon)}^{c - a}  \\ \text{uniformly for \(z,y\) in compact sets and \(\epsilon \in (0,1]\)} \nonumber. % chktex 9
\end{gather}
for some test function \(\varphi\), constants  \(c\) and \(a\).
The precise definition will occur in Chapter~\ref{chapter:coherence}. In our previous example \(A_{s,t} = F_s(1_{[s,t]})\), we have \(a = -1\) and \(c = \gamma - 1\). Note that our naive (and wrong) argumentation is not completely preposterous because in Chapter~\ref{section:sewing-lemma} we fix it by considering a dyadic approximation of the indicator function \( 1_{[s,t]} \).

Returning to our problem of finding an upper bound for \(g_k\)
\begin{align*}
    g_k(\psi) = \iint F_y(\hat \varphi^{\epsilon_k}_y) \check \varphi^{\epsilon_k}(y-z) \psi(z) \, \mathrm{d}y\, \mathrm{d}z 
    + \iint (F_z - F_y)(\hat \varphi^{\epsilon_k}_y) \check \varphi^{\epsilon_k}(y-z) \psi(z) \, \mathrm{d}y\, \mathrm{d}z,
\end{align*}
we now have \emph{coherence}~\eqref{pre-condition-coherence} to control the second summand of \(g_k\). By coherence 
\begin{align*}
    |(F_z - F_y)(\hat \varphi^{\epsilon_k}_y)| \leq C \epsilon_k^{a} {(|z-y| + \epsilon_k)}^{c-a}.
\end{align*}
If \(|z-y| < \epsilon_k\), then \(|(F_z - F_y)(\hat \varphi^{\epsilon_k}_y)| \leq C2^{c-y} \epsilon_k^c \to 0\) as \(k \to \infty\).

Regarding the first summand, we want \(F_x(\hat \varphi^{\epsilon}_x) \to 0\) as \(\epsilon \to 0\). One way to achieve this is by imposing a condition which we will call \emph{homogeneity}: if \(F_x(\hat \varphi^{\epsilon}_x) \leq B \epsilon^{\beta}\) for some constant \(B < \infty\), we will say that the germ \({(F_x)}_{x \in \mathbb{R}^d}\) has homogeneity bound \(\beta\). If a germ has homogeneity bound \(\beta > 0\), then   \(F_y(\hat \varphi^{\epsilon}_y) \leq B \epsilon^\beta \to 0\) as \(\epsilon \to 0\).  

It seems that we want a germ to satisfy the coherence and homogeneity condition. Fortunately, we get homogeneity for free if a germ is coherent. So, requiring a germ to be coherent is all we need to get the Reconstruction Theorem going.

\section{Coherence and Homogeneity}\label{chapter:coherence}

In this section we rigorously define \emph{coherence} and \emph{homogeneity}. We will see that coherence is an optimal condition for the Reconstruction Theorem. Moreover, homogeneity follows from coherence.

In Chapter~\ref{chapter:first-peek-at-reconstruction} we gave a naive motivation for coherence. Here comes the rigorous definition.

\begin{definition}[\(\gamma\)-coherent germs]\label{definition:coherence}
   Let \(\gamma \in \mathbb{R}\). A germ \({(F_x)}_{x \in \mathbb{R}^d}\) is called \emph{\(\gamma\)-coherent} if there exists a test function \(\varphi \in \mathcal{D}\) with \(\int \varphi(x) \, \mathrm{d}x \neq 0\) such that for every compact set \(K \subset \mathbb{R}^d\) there exists a non-positive real number \(\alpha_K \leq \min\left\{ 0, \gamma \right\}\) and a constant \(C < \infty\) with
   \begin{gather}\label{coherence}
        |(F_z - F_y)(\varphi^\lambda_y)| \leq C\lambda^{\alpha_K}{(|z-y| + \lambda)}^{\gamma - \alpha_K}  \\ \text{uniformly for \(z,y \in K\), \(|y-z| \leq 2\)  and \(\lambda \in (0,1]\)} \nonumber. % chktex 9
   \end{gather}
   We say that  \({(F_x)}_{x \in \mathbb{R}^d}\) is \((\bm{\alpha}, \gamma)\)-coherent if \(\bm \alpha = (\alpha_K)\). If \( \alpha_k \) is constant, we say \({(F_x)}_{x \in \mathbb{R}^d}\) is \(({\alpha}, \gamma)\)-coherent.
\end{definition}
\begin{remark}\label{remark:cutoff}
    Note how we require \(|y-z| \leq R\) for \( R = 2 \), which appears rather arbitrary. It turns out to be indifferent what value of \( R \) we plug in. The constraint \(|y-z| \leq R\) can even be entirely dropped, see Proposition~\ref{proposition:cutoff}. In the end, we choose \(R = 2\) because it is convenient for our goals.
\end{remark}

Fix \(K, \varphi, \alpha, \gamma\). The \emph{semi-norm \(\vertiii{\cdot}^{\mathrm{coh}}_{K,\varphi,\alpha,\gamma}\)} is the smallest constant \(C \in \mathbb{R} \cup \left\{ \infty \right\}\) such that the coherence condition~\eqref{coherence} holds for \(K, \varphi, \alpha, \gamma\). Concretely, we define
\begin{align*}
    \vertiii{F}^{\mathrm{coh}}_{K,\varphi,\alpha,\gamma} = \sup \left\{ \frac{(F_z - F_y)(\varphi^\lambda_y)}{\lambda^\alpha{(|z-y| + \lambda)}^{\gamma - \alpha}} : y,z \in K, |z-y| \leq 2, \lambda \in (0,1] \right\}. % chktex 9
\end{align*}

We briefly discuss the intuition behind coherence. For some constant \( C' < \infty \) we rewrite inequality~\eqref{coherence} to
\begin{align}\label{EspressoHouse}
    |(F_z - F_y)(\varphi^\lambda_y)| \leq C' \begin{cases}
        \lambda^\gamma \quad & \text{if \(|z-y| \leq \lambda\)} \\
        \lambda^{\alpha} |z-y|^{\gamma - \alpha} \quad & \text{otherwise}
    \end{cases}.
\end{align}
Observe that \(\lambda^{\gamma} \leq \lambda^{\alpha}\) because \(\lambda \in (0,1] \) and \(\gamma \geq \alpha\). As \(|z-y|\) decreases to \(\lambda\), the magnitude of \(|(F_z - F_y)(\varphi^\lambda_y)|\) shifts from \(\lambda^\alpha\) to \(\lambda^{\gamma}\). This change becomes very dramatic when \(\gamma > 0\) and \(\alpha < 0\) because \( \lambda^{\alpha} \) diverges whereas \(\lambda^{\gamma}\) vanishes.

We make the following important observation.
\begin{remark}[Monotonicity]
    The right-hand side of~\eqref{EspressoHouse} shrinks as \(\alpha \nearrow 0\) for fixed \(\gamma, y\) and \(z\). In other words, the larger \(\alpha\) (remember that \(\alpha < 0\)), the better the estimate becomes. Hence, without loss of generality we assume that the map \(K \mapsto \alpha_K\) is \emph{monotone}; that is for all compact sets \( K, K' \subset \mathbb{R}^d \) we have
\begin{align}\label{alpha-monotone}
    K \subset K' \implies \alpha_K \geq \alpha_{K'}.
\end{align}
This is achieved by choosing the exponents \(\alpha_K\) in the following way: for balls \(K = B(0,n)\) of radius \(n \in \mathbb{N}\) choose \(\alpha_K = \min\left\{ \alpha_{B(0,i)}  : 1\leq i \leq n\right\}\); otherwise for general compact sets \(K\) choose \(\alpha_K = \min\left\{ \alpha_{B(0,i)}  : 1\leq i \leq n\right\}\) with \(n \in \mathbb{N}\) such that  \(B(0,n) \supset K\). This ensures that the family of exponents \((\alpha_K)\) is monotone. It will play an important role in the proof of the Reconstruction Theorem in case \(\gamma < 0\).
\end{remark}

For the upcoming proofs, we need an assumption that controls the regularity of the reconstruction \( \mathcal{R}F \) for some germ \( F = (F_x)_{x \in \mathbb{R}^d} \). This assumption is called \emph{homogeneity} and directly follows from coherence.

\begin{lemma}[Homogeneity Bound]\label{lemma:homogeneity-bound}
   Let \({(F_x)}_{x \in \mathbb{R}^d}\) be a \(\gamma\)-coherent germ. Then, for every compact set \(K \subset \mathbb{R}^d\) there exists a real number \(\beta_K < \gamma\) and a constant \(B < \infty\) such that
   \begin{gather*}\label{homogeneity}
                |F_y(\varphi^\lambda_y)| \leq B\lambda^{\beta_K} \quad
                \text{uniformly for \(y \in K\) and \(\lambda \in (0,1]\)}. % chktex 9
   \end{gather*}
   We say the germ \({(F_x)}_{x \in \mathbb{R}^d}\) has \emph{local homogeneity bound} \(\bm \beta = (\beta_K)\). We say the germ \({(F_x)}_{x \in \mathbb{R}^d}\) has \emph{global homogeneity bound} \(\beta \in \mathbb{R}\) if \(\beta_K = \beta\) for all compact sets \(K \subset \mathbb{R}^d\).
\end{lemma}

\begin{proof}
    Writing
    \begin{align*}
        |F_y(\varphi^\lambda_y)| \leq |(F_y - F_z)(\varphi^\lambda_y)| + |F_z(\varphi^\lambda_y)|
    \end{align*}
    we use coherence to bound the first summand. Precisely, fix any compact set \(K \subset \mathbb{R}^d\) and \(z \in K\). By coherence, we have \(|(F_y - F_z)(\varphi^\lambda_y)| \leq  C \lambda^{\alpha}(|z-y|+\lambda)^{\gamma - \alpha} \leq \{ C (\mathrm{diam}(K) + 1)^{\gamma - \alpha} \} \cdot  \lambda^{\alpha}\) uniformly for \(y \in K\) and \(\lambda \in (0,1]\) (where \(\mathrm{diam}(K) \coloneqq \sup_{y,z \in K}|y-z| \)). % chktex 9

    To estimate \(|F_z(\varphi^\lambda_y)|\), we know there exist \(\tilde C < \infty\) and \(r \in \mathbb{N}_0\) such that \(|F_z(\varphi^\lambda_y)| \leq \tilde C \lVert \varphi^\lambda_y \rVert_{C^r}\) for all \(y \in K\) and \(\lambda \in (0,1]\) because \(F_z\) is a distribution. Also, we have \(\lVert \partial^k\varphi^\lambda_y \rVert_\infty \leq \lambda^{-|k|- d} \lVert \partial^k\varphi \rVert_\infty \leq \lambda^{-r - d} \lVert \varphi \rVert_{C^r}\). Thus, \(\lVert \varphi^\lambda \rVert_{C^r} \leq \lambda^{-r-d}\lVert \varphi \rVert_{C^r}\) follows. In the end, we obtain 
    \(|F_z(\varphi^\lambda_y)| \leq \{ \tilde C  \lVert \varphi \rVert_{C^r}  \} \cdot \lambda^{-r-d}\).

    We choose \(B = C {(\mathrm{diam}(K) + 1)}^{\gamma - \alpha} +  \tilde C  \lVert \varphi \rVert_{C^r}  \) and \(\beta \leq \min \left\{ \alpha, -r-d, \gamma \right\}\). We can further decrease \( \beta \) to ensure \( \beta < \gamma \).
\end{proof}

Similar to \((\alpha_K)\), the family \((\beta_K)\) is \emph{monotone} in the sense that 
\begin{align}\label{beta-monotone}
    K \subset K' \implies \beta_K \geq \beta_{K'}.
\end{align} 
We also introduce a semi-norm that quantifies the homogeneity of a coherent germ \( F  \)
\begin{align}\label{definition:semi-norm-homogeneity}
    \vertiii{F}^{\mathrm{\hom}}_{K, \varphi, \beta} = \sup_{\substack{x \in K \\ \lambda \in (0,1]}} \frac{|F_x(\varphi^\lambda_x)|}{\lambda^\beta} \qquad \text{compact set \( K \subset \mathbb{R}^d \)}.
\end{align}

\section{The Reconstruction Theorem in Detail}

We are ready to state the Reconstruction Theorem.

\begin{theorem}[Reconstruction Theorem~\cite{caravenna2021hairer}]\label{theorem:reconstruction-theorem}
   Let \(\gamma \in \mathbb{R}\) and \({(F_x)}_{x \in \mathbb{R}^d}\) be an \((\bm{\alpha}, \gamma)\)-coherent germ with local homogeneity bounds \(\bm \beta\). Then, there exists a distribution \(f \in \mathcal{D}'(\mathbb{R}^d)\) such that for every compact set \(K \subset \mathbb{R}^d\) and all integers \(r> \max \left\{ -\alpha_{\bar K_2}, -\beta_{\bar K_2} \right\}\), \( \alpha \coloneqq \alpha_{\enlarg{K}{2}} \) we have
   \begin{gather}\label{reconstruction-theorem}
        |(f-F_x)(\psi^\lambda_x)| \leq \{\mathrm{constant} \} \cdot \vertiii{F}^{\mathrm{coh}}_{\bar K_2, \varphi, \alpha, \gamma} \cdot \begin{cases}
            \lambda^\gamma \quad & \text{if \(\gamma \neq 0\)}\\
            1 + |\log \lambda| \quad & \text{if \(\gamma = 0\)}
        \end{cases}
        \\ \text{uniformly for \(\psi \in \mathcal{B}_r\), \(x \in K\), \(\lambda \in (0,1]\).} \nonumber
   \end{gather}
   The multiplicative constant may only depend on \( \alpha, \gamma, r, d \) and \( \varphi \). It is explicitly computed in~\eqref{constant:rec-gamma-bigger-zero},~\eqref{holy-molly} and~\eqref{holy-mollyl} for the cases \( \gamma > 0 \), \( \gamma < 0 \) and \( \gamma = 0 \) respectively.

   If \( \gamma > 0 \), the reconstruction \( f = \mathcal{R}F \) is unique. Moreover, the map \( F \mapsto \mathcal{R}F \) is linear.

   If \( \gamma \leq 0 \) is no longer unique. However, for any \( \alpha \leq 0 \) and \( \gamma \geq \alpha \) we can choose \( \mathcal{R}F \) in such a way that the map \( F \mapsto \mathcal{R}F \) is linear on the vector space of \( (\alpha, \gamma) \)-coherent germs with global homogeneity bound \( \beta \).
\end{theorem}

Remarkably, coherence is not only \emph{sufficient}; that is coherence implies the Reconstruction Theorem. Coherence is also necessary. We say that coherence is an \emph{optimal} condition.

\begin{theorem}[Coherence is necessary]\label{theorem:coherence-is-necessary}
   Fix any \(\gamma \in \mathbb{R}\).  Let \({(F_x)}_{x \in \mathbb{R}^d}\) be a germ. Let \(f \in \mathcal{D}'\) be a distribution such that for every compact set \(K \subset \mathbb{R}^d\) there exists \(C < \infty\) and \(r \in \mathbb{N}\) with
   \begin{gather}\label{thm:coh-necessary}
        |(f-F_y)(\psi^\lambda_y)| \leq C \lambda^\gamma \\
        \text{for all \(y \in K, \lambda \in (0,1]\) and \(\psi \in \mathcal{B}_r\).} \nonumber
   \end{gather}
   Then, \({(F_x)}_{x \in \mathbb{R}^d}\) is \(\gamma\)-coherent.  
\end{theorem}

\begin{proof}
   To show that \({(F_x)}_{x \in \mathbb{R}^d}\) is \(\gamma\)-coherent, we prove that there exists \(\alpha \leq \min\left\{ 0, \gamma \right\}\) and a constant \(C < \infty\) such that
   \begin{gather*}
        |(F_z - F_y)(\varphi^\lambda_y)| \leq C\lambda^\alpha{(|z-y| + \lambda)}^{\gamma - \alpha}  \\ \text{uniformly for \(z,y \in K\), \(|y-z| \leq \frac{1}{2}\)  and \(\lambda \in (0,1]\)} \nonumber. % chktex 9
   \end{gather*}  
   Note that we require \(|y-z| \leq \frac{1}{2}\), but this is equivalent to Definition~\ref{definition:coherence}, see Remark~\ref{remark:cutoff}.

   Fix a compact set \(K \subset \mathbb{R}^d\). Let \(x,y \in K\) with \(|x-y| \leq \frac{1}{2}\), \(\lambda \in (0, \frac{1}{2}]\) and \(\psi \in \mathcal{B}_r\). Then,
   \begin{align*}
    |(F_x - F_y)(\psi^\lambda_y)| \leq |(F_x - f)(\psi^\lambda_y)| + |(f - F_y)(\psi^\lambda_y)| \overset{\eqref{thm:coh-necessary}}{\leq} |(F_x - f)(\psi^\lambda_y)| + C \lambda^\gamma.
   \end{align*}
   Next, estimating \(|(F_x - f)(\psi^\lambda_y)|\) is nontrivial because \(\psi_y^\lambda\) is centered around \(y\) and not \(x\). We overcome this obstacle by substituting \(\psi^\lambda_y \leadsto \xi^{\lambda_1}_x\), where 
   \begin{gather*}
       \xi \coloneqq \psi^{\lambda_2}_w, \quad w \coloneqq \frac{y-x}{|x-y| + \lambda}, \\
    \lambda_1 \coloneqq |x-y| + \lambda, \;\text{ and } \; \lambda_2 \coloneqq \frac{\lambda}{|x-y|  + \lambda}.
   \end{gather*}
   We quickly verify the correctness of the substitution
   \begin{align*}
    \xi^{\lambda_1}_x = \frac{\psi\left(
        \lambda_2^{-1}\left(\frac{\cdot - x}{|x-y| + \lambda} - w\right)
    \right) }{\left((|x-y| + \lambda)\frac{\lambda}{|x-y| + \lambda}\right)^d}
    =
    \frac{\psi\left(
        \frac{\cdot - x - (y-x)}{\lambda}
    \right) }{\lambda^d} = \lambda^{-d}\psi\left( \frac{\cdot - y}{\lambda} \right) = \psi^\lambda_{y}.
   \end{align*}
   Hence, 
   \begin{align}
    |(F_x - f)(\psi^\lambda_y)| = |(F_x - f)\left( \xi^{\lambda_1}_x \lVert \xi \rVert_{C^{r}}^{-1} \right)| \cdot  \lVert \xi \rVert_{C^{r}} &\overset{\eqref{thm:coh-necessary}}{\leq} C\lambda_1^\gamma\lVert \xi \rVert_{C^{r}}. \label{SponsoredByGilette}
   \end{align}
   To justify \(\eqref{thm:coh-necessary}\), observe that 
   \begin{itemize}
       \item  \(\lambda_1 \in (0, 1]\), and 
       \item  \(\xi^{\lambda_1}_x \lVert \xi \rVert_{C^{r}}^{-1} \in \mathcal{B}_{r}\) because \(\lambda_2 + |w| = 1\) and \(\mathrm{supp}(\psi) \subset B(0,1)\); both imply that \(\xi = \psi^{\lambda_2}_w\) is supported in \(B(0,1)\), and the scaling factor \(\lVert \xi \rVert_{C^{r}}^{-1}\) ensures that the \(C^r\)-norm is one.
   \end{itemize}   
   Additionally, \(\lVert \xi \rVert_{C^{r}} = \max_{k \leq r}\lVert \partial^k \psi^{\lambda_2}_w \rVert_\infty = \max_{k \leq r} \lambda_2^{-d-k} \lVert \partial^k \psi \rVert_{\infty} \leq \lambda_2^{-d - r}\). So,
   \begin{align*}
    |(F_x - f)(\psi^\lambda_y)| \leq C\lambda_1^\gamma \lambda_2^{-d-r} &= C(|x-y| + \lambda)^\gamma\left(\frac{\lambda}{|x-y|  + \lambda}\right)^{-d-r} \\
    &\leq C  (|x-y| + \lambda)^{\gamma - \alpha} \lambda^\alpha,
   \end{align*}
    where we define \(\alpha = \min\left\{ -d-r , \gamma \right\}\). Then,
    \begin{align*}
        |(F_x - F_y)(\psi^\lambda_y)| &\leq C  (|x-y| + \lambda)^{\gamma - \alpha} \lambda^\alpha + C \lambda^\gamma \\
        &\Downarrow \text{where \(\lambda^\gamma = \lambda^{\gamma - \alpha} \lambda^\alpha \leq (|x-y| + \lambda)^{\gamma - \alpha} \lambda^\alpha\)} \\
        &\leq 2C  (|x-y| + \lambda)^{\gamma - \alpha} \lambda^\alpha.
    \end{align*}
\end{proof}

We slightly modify the previous proof to show that the constraint \(|z-y| \leq 2\) in the coherence condition (see Definition~\ref{definition:coherence}) can be dropped. If~\eqref{coherence} holds uniformly for any \(y,z \in K\) with \(|z-y| \leq 2\), then it also holds for any \(\tilde y, \tilde z \in K\) with \(|\tilde z- \tilde y| > 2\) (possibly with a different multiplicative constant \(C\)). Hence, coherence is equivalent to
\begin{gather}\label{better-coherence}
    |(F_z - F_y)(\varphi^\lambda_y)| \leq C\lambda^\alpha(|z-y| + \lambda)^{\gamma - \alpha}  \\ \text{uniformly for \(z,y \in K\) and \(\lambda \in (0,1]\)} \nonumber.
\end{gather}

\begin{proposition}\label{proposition:cutoff}
    Let \(F\) be a \(\gamma\)-coherent germ as in Definition~\ref{definition:coherence}. Then, it satisfies~\eqref{better-coherence} for any compact set \(K\) provided the multiplicative constant \(C\) is adjusted.
\end{proposition}

\begin{proof}
    Let \(F\) be a \(\gamma\)-coherent germ and \(\varphi\) be as in Definition~\ref{definition:coherence}. Fix a compact set \(K \subset \mathbb{R}^d\). Assume \(y,z \in K\) with \(|y-z| > 2\). Let \(A\) be a finite family of points in \(\mathbb{R}^d\) such that \(K\) is covered by \(A\) and for each point \(x \in K\) there exists \(a_x \in A\) with \(|x-a| < 2\). Such \(A\) exists because \(K\) is compact. Then, we have \(|(F_z - F_y)(\varphi^\lambda_z)| \leq |(F_z - F_{a_z})(\varphi^\lambda_z)| + |(F_{a_z} - F_y)(\varphi^\lambda_z)|\). The first summand is bounded by~\eqref{coherence}. Bounding the second summand is nontrivial since \(\varphi\) is centered around \(z\) and not \(a_z\) or \(y\). This is the same situation as in the proof of Theorem~\ref{theorem:coherence-is-necessary}. So, we write
    \begin{align*}
        |(F_{a_z} - F_y)(\varphi^\lambda_z)| \leq |(F_{a_z} - f)(\varphi^\lambda_z)| + |(f - F_y)(\varphi^\lambda_z)|,
    \end{align*}
    where \(f\) is the reconstruction of the germ \(F\); note that \(f\) exists by the Reconstruction Theorem and the Reconstruction Theorem only requires \(F\) to be coherent in the sense of Definition~\ref{definition:coherence}. Next, we use the same substitution as in~\eqref{SponsoredByGilette} to obtain an upper bound for both summands. These upper bounds only depend on \(|a_z - z|\), \(|y - z|\) and \(\lambda\), which ends the proof.  
\end{proof}

Next, we show uniqueness of the reconstruction \( \mathcal{R}F \). 

\begin{theorem}[Uniqueness]\label{theorem:uniqueness-reconstruction}
   Let \(F = {(F_x)}_{x \in \mathbb{R}^d}\) be a germ and \(\varphi \in \mathcal{D}\) be a test function with \(\int \varphi(x) \mathrm{d}x \neq 0\). Let \(K \subset \mathbb{R}^d\) be a compact set, and let \(f, g \in \mathcal{D}'\) be any two distributions such that
   \begin{align*}
       \lim_{\lambda \to 0} |(f-F_x)(\varphi^\lambda_x)| &= 0 \quad \text{uniformly for \(x\) in \(K\)} \\ 
       \quad \lim_{\lambda \to 0} |(g-F_x)(\varphi^\lambda_x)| &= 0 \quad \text{uniformly for \(x\) in \(K\)}
   \end{align*} 
    Then, \(f(\psi) = g(\psi)\) for all test functions \(\psi \in \mathcal{D}(K)\).  
\end{theorem}

\begin{proof}
    Define \(F, \varphi, K, f\) and \(g\) as in the theorem. Next, we define \(T \coloneqq f - g\), fix \(\psi \in \mathcal{D}(K)\) and show \(T(\psi) = 0\).
   
    We assume that \(\int \varphi(x) \mathrm{d} x = 1\) (otherwise we replace \(\varphi\) by \((\int \varphi(x) \mathrm{d}x)^{-1}\varphi\)). Then, the family \((\varphi^\lambda)_{\lambda \in (0,1]}\) is a mollifier, and thus \(T(\psi) = \lim_{\lambda \to 0}T(\psi * \varphi^\lambda)\). This allows us to estimate 
    \begin{align*}
        |T(\psi * \varphi^\lambda)| = \left|\int T(\varphi^\lambda_x) \psi(x) \mathrm{d}x\right| \leq \lVert \psi \rVert_{L^1} \, \sup_{x \in K}|T(\varphi^\lambda_x)|,
    \end{align*}
    where for the last inequality we recall that \(\psi\) has compact support in \(K\). Using the triangle inequality, we bound 
    \begin{align*}
        |T(\varphi^\lambda_x)| = |(f-g)(\varphi^\lambda_x)| \leq |(f-F_x)(\varphi^\lambda_x)| + |(g-F_x)(\varphi^\lambda_x)|.
    \end{align*}
    Taking the limit \(\lambda \to 0\) proves the uniqueness. 
\end{proof}


\chapter{Proof of the Reconstruction Theorem}\label{chapter:general-proof}

From chapter 1 TO-DO that given a germ \((F_x)_{x \in \mathbb{R}^d}\) we introduced the \emph{approximating distributions} 
\begin{align*}
    f_n(\psi) \coloneqq \int_{\mathbb{R}^d} F_y(\rho_y^{\epsilon_n}) \psi(y) \, \mathrm{d}y \tag{Approximating distributions},
\end{align*}
which was motivated by \(F_x(\psi * \rho^\epsilon) \to F_x(\psi)\). We \emph{guessed} that if the \emph{coherence condition}~\eqref{coherence} holds, i.e.\ the germ \((F_x)_{x \in \mathbb{R}^d}\) is \((\bm \alpha, \gamma)\)-coherent, then
\begin{enumerate}[label=(H\arabic*)] % chktex 36
    \item \(\lim_{n \to \infty}f_n\) exists, and
    \item  \(\lim_{n \to \infty}f_n\) satisfies the inequality~\eqref{reconstruction-theorem} of the reconstruction theorem
 \end{enumerate}
 Then, we \emph{decomposed} the approximating distributions \(f_n\) using a telescopic into 
 \begin{align*}
    f_n = f_1 + \sum^{n-1}_{k=1}
    g_k, \quad g_k(\psi) \coloneqq f_{k+1}(\psi) - f_k(\psi) = \int_{\mathbb{R}^d} F_y(\rho_y^{\epsilon_{k+1}} - \rho_y^{\epsilon_k}) \psi(y)\, \mathrm{d}y.
 \end{align*}
For (H1) to hold, the series must therefore be finite. When proving the reconstruction theorem, we will largely be concerned with showing (H1) and (H2). However, there is one caveat that we need to be aware of: the limit \(\lim_{n \to \infty}f_n\) of claim (H1) need not exist for \(\gamma \leq 0\), and we need to divide the proof into two cases \(\gamma >0\) and \(\gamma \leq 0\). Fortunately, claim (H1) for \(\gamma \leq 0\) can be fixed in a simple way (TO-DO), and both cases share a large part of the proof. 

In the following sections, we prove the reconstruction theorem in multiple steps. As mentioned, a large part of the proof for the case \(\gamma > 0\) holds for the case \(\gamma \leq 0\), too; so the steps we will be presenting now \emph{hold for both cases}; only in the latest steps we divide the proof. We will specifically mention in later steps when we need \(\gamma > 0\) or \(\gamma\leq 0\), but for now we do not worry about it.

\section{Step 0: Setup}\label{setup} First, we lay the foundation of the proof. We have already given an informal ansatz how we wanted to approach the proof. Specifically, the proof is centered around showing that the limit of the approximating distributions \(\lim_{n \to \infty} f_n\) do exist and satisfy the inequality~\eqref{reconstruction-theorem} of the reconstruction theorem.

Let \(\gamma \in \mathbb{R}\) and \(F = (F_x)_{x \in \mathbb{R}^d}\) be a \((\bm \alpha, \gamma)\)-coherent germ with local homogeneity bounds \(\bm \beta\) and test function \(\varphi\). Without loss of generality, we assume that \(\bm \alpha\) and \(\bm \beta\) are monotone. Let \(K \subset \mathbb{R}^d\) be a compact set. Define 
\begin{align*}
    \alpha \coloneqq \alpha_{\bar K_{3/2}} \quad \text{and} \quad  \beta \coloneqq \beta_{\bar K_{3/2}}
\end{align*}
 such that the coherence condition~\eqref{coherence} with homogeneity bound~\eqref{homogeneity} holds for the \(\frac{3}{2}\)-enlargement \(\bar K_{3/2}\), i.e.\ there exist constants \(C,B < \infty\) such that
\begin{gather}\label{starter-coherence}
    |(F_z - F_y)(\varphi^\epsilon_y)| \leq C \, \epsilon^\alpha(|z-y| + \epsilon)^{\gamma - \alpha} \quad \text{ and } \quad
    |F_y(\varphi^\epsilon_y)| \leq B \epsilon^\beta \\
    \text{for all } y,z \in \bar K_{3/2} \text{ with } |z-y| \leq 2,  \epsilon \in (0,1] \nonumber
\end{gather}

We define a sequence \((\epsilon_k)_{k \in \mathbb{N}}\) by \(\epsilon_k = 2^{-k}\). Next define \(r \in \mathbb{N}\) such that
\begin{align}\label{setup:r}
    r > \max\left\{ -\alpha, -\beta \right\} \tag{\texttt{R}}.
\end{align}
This particular choice will allow us to bound \(\sum^\infty_{k=0} g_k\). 

\section{Step 1: Tweaking}\label{chapter:step-1-tweaking}

We briefly discuss the motivation and concept of tweaking: in previous chapters we wrote \(f_n = f_1 + \sum^{n-1}_{k=1}g_k\) where
\begin{align*}
    g_k(\psi) = f_{k+1}(\psi) - f_k(\psi) = \int_{\mathbb{R}^d} F_y(\rho_y^{\epsilon_{k+1}} - \rho_y^{\epsilon_k}) \psi(y)\, \mathrm{d}y
\end{align*} 
for some mollifier \(\rho\). Finding a suitable \(\rho\) is the task that we confront ourselves with in this step, which we shall call \emph{tweaking}. 

We will see that it turns out to be quite useful if we can write \(\rho_y^{\epsilon_{k+1}} - \rho_y^{\epsilon_k}\) as a difference of two test functions \(\hat \varphi\) and \(\check \varphi\):
\begin{align*}\label{rho-dif}
    \rho_y^{\epsilon_{k+1}} - \rho_y^{\epsilon_k} = (\hat \varphi^{\epsilon_k} * \check \varphi^{\epsilon_k} )_y \tag{\texttt{MOL}}.
\end{align*}
Additionally, we want \(\hat \varphi\) and \(\check \varphi\) to possess some advantageous properties that are listed in the following table.

\begin{table}[H]
\centering
\begin{tabular}{p{50mm}|p{50mm}} % chktex 44
    \bottomrule
    \\[-0.5em]
    \(\mathrm{supp}(\hat \varphi) = B(0,\frac{1}{2})\) &  \(\mathrm{supp}(\check \varphi) = B(0,1)\)\\
    \\[-0.5em]
    \(\int \hat \varphi(x) \, \mathrm{d}x = 1\)&  \(\int \check \varphi(x) \, \mathrm{d}x = 0\)\\
    \\[-0.5em]
    \(\hat \varphi\) annihilates monomials \newline of degree from \(1\) to \(r-1\) & \(\check \varphi \) annihilates monomials  \newline of degree from \(0\) to \(r-1\)\\
    \\[-0.5em]
    \(\hat \varphi\) satisfies the coherence \\condition~\eqref{starter-coherence}& \\[0.5em]
    \toprule
\end{tabular}
\caption{Properties of \(\hat \varphi\) and \(\check \varphi\)}\label{table:properties-tweak}
\end{table}

\begin{definition}
    A function \(g \in \mathcal{D}\) is said to \emph{annihilate monomials} of degree \(j \in \mathbb{N}\) if for all \(n \in \mathbb{N}^d_0\) with \(|n| = j\) we have
    \begin{align*}
        \int_{\mathbb{R}^d} y^n g (y) \, \mathrm{d}y = 0.
    \end{align*}
\end{definition}

\emph{Tweaking} \(\varphi\) allows us to construct such nice test functions \(\hat \varphi\) and \(\check \varphi\).

\begin{lemma}[Tweaking]\label{tweaking-lemma}
    Let \(r \in \mathbb{N}\), and let \(\lambda_0, \ldots,\lambda_{r-1} \in \mathbb{R}_{>0}\) be pairwise distinct. Define 
    \begin{align*}
        c_0 = 1 \quad \text{and} \quad 
        c_i = \prod_{k \in \{0, \ldots,r-1\} \setminus\left\{ i \right\} } \frac{\lambda_k}{\lambda_k - \lambda_i}, \quad i > 0.
    \end{align*}
    Then, for every measurable and compactly supported function \(\varphi: \mathbb{R}^d \to \mathbb{R}\) and every \(a \in \mathbb{R}\) the \emph{tweaked function}  
    \begin{align*}
        \mathcal{T}_{\varphi}: x \mapsto a\sum^{r-1}_{i=0}c_i\varphi^{\lambda_i}(x)
    \end{align*}
    has integral equal to \(a \int \varphi(x) \, \mathrm{d}x\) and annihilates monomials of degree from \(1\) to \(r-1\).
\end{lemma}

\begin{proof}
    The case for \(r = 1\) is simple: \(\int \mathcal{T}_{\varphi}(x) \mathrm{d}x = a \int \varphi^{\lambda_0}(x) \mathrm{d}x = a \int \varphi(x) \mathrm{d}x\).  

    Let \(r \geq 2\). Given all \(\lambda_i\)'s we solve for the variables \(c_i\)'s such that the desired properties hold. Luckily for us, this is a simple system of linear equations. Write
    \begin{align*}
        \int y^k \mathcal{T}_{\varphi}(y) \, \mathrm{d}y = a \sum^{r-1}_{i=0} c_i \int y^k \varphi^{\lambda_i}(y) \, \mathrm{d}y =  a \sum^{r-1}_{i=0} c_i \lambda_i^{|k|} \int x^k \varphi(x) \, \mathrm{d}x, \quad \forall k \in \mathbb{N}^d
    \end{align*}
    where we substituted \(y \leadsto \lambda_i x\). Now observe that for \(k = 0\) we get
    \begin{align*}
        \int \mathcal{T}_{\varphi}(y) \, \mathrm{d}y = a \sum^{r-1}_{i=0} c_i \lambda_i^{|k|} \int \varphi(x) \, \mathrm{d}x.
    \end{align*} 
    Thus, if we find \(c_i\)'s  such that the constraint \(\sum^{r-1}_{i=0} c_i \lambda_i= 1\) holds, the tweaked function \(\mathcal{T}_{\varphi}\) has integral equal to \(a \int \varphi(x) \mathrm{d}x\). Next, if we let \(1 \leq |k| \leq r - 1\), we want \(\int y^k \mathcal{T}_{\varphi}(y) = 0\); so the constraint \(\sum^{r-1}_{i=0} c_i \lambda_i^{|k|} = 0\) needs to be satisfied.

    In the language of linear algebra, we try to solve
    \begin{align*}
        \begin{pmatrix}
            1 & \cdots & 1 \\
            \lambda_0 & \cdots & \lambda_{r-1} \\
            \lambda_0^2 & \cdots & \lambda_{r-1}^2 \\
            & & \\
            \lambda_0^{r-1} & \cdots & \lambda_{r-1}^{r-1}
        \end{pmatrix}
        \begin{pmatrix}
            c_0 \\ c_1 \\ c_2 \\ \\ c_{r-1}
        \end{pmatrix}
        = 
        \begin{pmatrix}
            1 \\ 0 \\ 0 \\ \\ 0
        \end{pmatrix}.
    \end{align*}
    The matrix on the left is a \emph{Vandermonde matrix} for which it is easy to compute the determinant: \(\det = \prod_{1 \leq i < j \leq r - 1} \lambda_j  - \lambda_i\). Therefore, a solution \(c\) exists if and only if the determinant does not vanish if and only if all \(\lambda_i\)'s are distinct. If we let \(A\) denote the left hand side matrix, the inverse of \(A\)  can be explicitly stated 
    \begin{align*}
        (A^{-1})_{i=0,\ldots,r-1}^{j=0,\ldots,r-1} = (-1)^j \frac{\sum\limits_{\substack{U \subset \left\{ 0,\ldots,r-1 \right\} \setminus \left\{ i \right\} \\ |U| = r - 1 -j}} \; \prod\limits_{u \in U} \lambda_u}{\prod\limits_{v \in \left\{ 0,\ldots,r-1 \right\} \setminus \left\{ i \right\}} (\lambda_v - \lambda_i)},
    \end{align*}
    see equation (7) in~\cite{klinger1965vandermonde} for more details. We left-multiply the linear system with this inverse
    \begin{align*}
        \begin{pmatrix}
            c_0 \\ c_1 \\ c_2 \\ \\ c_{r-1}
        \end{pmatrix}
        = A^{-1}\begin{pmatrix}
            1 \\ 0 \\ 0 \\ \\ 0
        \end{pmatrix},
    \end{align*}
    and we finally confirm that the vector \(c\) is a solution if and only if  
    \begin{align*}
        c_i = \frac{\prod\limits_{u \in \left\{ 0, \ldots, r-1 \right\} \setminus \left\{ i \right\} }\lambda_u}{\prod\limits_{v \in \left\{ 0, \ldots, r-1 \right\} \setminus \left\{ i \right\}} (\lambda_v - \lambda_i)} = \prod\limits_{k \in \left\{ 0, \ldots , r-1 \right\} \setminus \left\{ i \right\}} \frac{\lambda_k}{\lambda_k - \lambda_i}.
    \end{align*}
\end{proof}

We define \(\hat \varphi\) as
\begin{align}\label{definition:tweakedvarphi}
    \varphi = \mathcal{T}_{\varphi}
\end{align}
for \(a = \frac{1}{\int \varphi(x) \, \mathrm{d}x}\) and \(\lambda_i = \frac{2^{-(i+1)}}{1+R_\varphi}\) for all \(i = 0,\ldots,r-1\).   
    
\begin{lemma}\label{lemma:hat-phi-satisfies-table}
    \(\hat \varphi\) satisfies the properties in Table~\ref{table:properties-tweak}.
\end{lemma}
\begin{proof}
    By the tweaking lemma, \(\hat \varphi\) integrates to one and annihilates all monomials from degree \(1\) to \(r-1\). 
    We also have \(\mathrm{supp}(\hat \varphi) = B(0,\frac{1}{2})\) because the support of \(\hat \varphi\) depends on the largest \(\lambda_i\), and \(\lambda_i = \frac{1}{2^{i+1}(1+R_\varphi)} \leq \frac{1}{2R_\varphi}\). 
    
    It remains to show that the coherence inequality~\eqref{starter-coherence} holds if \(\varphi\) is replaced by \(\hat \varphi\) (possibly with different constants \(C\) and \(B\)). By definition of \(\hat \varphi = \mathcal{T}_\varphi\) we have \(|(F_z - F_y)(\hat \varphi^\epsilon_y)| = a\sum^{r-1}_{i=0} |c_i| \, |(F_z - F_y)(\varphi_y^{\lambda_i \epsilon})|\). Next, we bound
    \begin{align*}
        |(F_z - F_y)(\varphi_y^{ \epsilon \lambda_i})| &\overset{\eqref{starter-coherence}}{\leq}  C(\epsilon\lambda_i)^\alpha(|z-y| + \epsilon\lambda_i)^{\gamma-\alpha} \\
        &\Downarrow \text{because \(\alpha < 0\) and \(\lambda_i > \frac{2^{-(r+1)}}{1+R_\varphi}\)}\\
        &\leq  \left(\frac{2^{-(r+1)}}{1+R_\varphi}\right)^\alpha C\epsilon^\alpha(|z-y| + \epsilon\lambda_i)^{\gamma - \alpha} \\
        &\Downarrow \text{because \(\gamma - \alpha \geq 0\) and \(\lambda_i \leq 1\)}\\
        &\leq   \left(\frac{2^{-(r+1)}}{1+R_\varphi}\right)^\alpha C \epsilon^\alpha(|z-y| + \epsilon)^{\gamma - \alpha}.
    \end{align*}
    To estimate the constants \(c_i\), we use \(|c_i| \leq e^2\) --- this fact will be proved in the next lemma (see equation~\eqref{jsknfjkewfwhiru}), but we can already use it here. Altogether, we have  
    \begin{align*}
        |(F_z - F_y)(\hat \varphi^\epsilon_y)| &\leq \frac{e^2 r}{|\int \varphi(x)\, \mathrm{d}x|} \left(\frac{2^{-(r+1)}}{1+R_\varphi}\right)^\alpha C \, \epsilon^\alpha(|z-y| + \epsilon)^{\gamma - \alpha}.
    \end{align*}
    Analogously, 
    \begin{align*}
        |F_y(\hat \varphi^\epsilon_y)| \leq \frac{e^2 r}{|\int \varphi(x)\, \mathrm{d}x|} B \epsilon^\beta\lambda_i^\beta \leq  \frac{e^2 r}{|\int \varphi(x)\, \mathrm{d}x|}  \left(\frac{2^{-(r+1)}}{1+R_\varphi}\right)^{\min\left\{ \beta, 0 \right\} }B\epsilon^\beta
    \end{align*}
    Therefore, the coherence and homogeneity condition still hold if we do the following replacements:
    \begin{align}
        \varphi &= \hat \varphi, \nonumber
        \\
        \hat C &= C\frac{e^2 r}{|\int \varphi(x)\, \mathrm{d}x|} \left(\frac{2^{-(r+1)}}{1+R_\varphi}\right)^\alpha \label{constant:hat-c}
        \\
        \hat B &= B\frac{e^2 r}{\int \varphi(x)\, \mathrm{d}x} \left(\frac{2^{-(r+1)}}{1+R_\varphi}\right)^{\min\left\{ \beta, 0 \right\} } \nonumber.
    \end{align}
\end{proof}

What we have just proven is of elementary importance for the next steps; in other words the existence of the tweaked function \(\hat \varphi\) does a lot of heavylifting for us. The main takeaway is the following fact: if \(\varphi\) satisfies the coherence condition, so does \(\hat \varphi\):
\begin{gather}\label{coherence-hat}
    |(F_z - F_y)(\hat \varphi^\epsilon_y)| \leq \hat C \epsilon^\alpha(|z-y| + \epsilon)^{\gamma - \alpha} \quad \text{and} \quad |F_y(\hat \varphi^\epsilon_y)| \leq \hat B\epsilon^\beta \tag{\({\widehat{\texttt{COH}}}\)} \\
    \text{for all } y,z \in \bar K_{3/2} \text{ with } |z-y| \leq 2,  \epsilon \in (0,1] .\nonumber
\end{gather}

Next, we define \(\check \varphi \coloneqq \hat \varphi^{\frac{1}{2}} - \hat \varphi^2\), and quickly verify the properties in Table~\ref{table:properties-tweak}.
\begin{itemize}
    \item First, \(\mathrm{supp}(\check \varphi) \subset B(0,1)\) because \(\mathrm{supp}(\hat \varphi) \subset B(0,\frac{1}{2})\).
    \item Second, \(\int \check \varphi(x) \, \mathrm{d}x = 0\) because \(\int \hat \varphi^{\frac{1}{2}}(x) \, \mathrm{d}x = \int \hat \varphi^2(x) \, \mathrm{d}x\).
    \item Third, \(\check \varphi\) annihilates monomials of degree \(1\) to \(r-1\) because  \(\hat \varphi\) annihilates monomials of degree \(1\) to \(r-1\).
\end{itemize}

Finally, we set \(\rho = \hat \varphi^2 * \hat \varphi\). This is a mollifier because \(\int \hat \varphi^2(x) \, \mathrm{d}x = \int \hat \varphi(x)  \, \mathrm{d}x = 1\).  Then, \(\rho^{\frac{1}{2}} - \rho = (\hat \varphi^2 * \hat \varphi)^{\frac{1}{2}} - (\hat \varphi^2 * \hat \varphi) = (\hat \varphi * \hat \varphi^{\frac{1}{2}})- (\hat \varphi^2 * \hat \varphi) = \hat \varphi * (\hat \varphi^{\frac{1}{2}} - \hat \varphi^{2}) = \hat \varphi * \check \varphi\). Hence, we get \(\rho^{\epsilon_{k+1}} - \rho^{\epsilon_k} = (\rho^{\frac{1}{2}} - \rho)^{\epsilon_k} = (\hat \varphi * \check \varphi)^{\epsilon_k} = \hat \varphi^{\epsilon_k} * \check \varphi^{\epsilon_k}\) --- this is exactly what we want: \emph{we found a mollifier whose difference is the convolution of two tweaked test functions}.

The remaining chapter is devoted to three technical lemmas that involve estimating the tweaked test functions \(\hat \varphi\) and \(\check \varphi\). They are only used in some minor parts of the proofs that are about to come. You can either skip to Chapter~\ref{step2:decomposition}: Step 2 right away and come back when the lemmas are actually needed.

Many upcoming proofs require estimating the \(L^1\)-norm of \(\hat \varphi\). We even used an estimate in Lemma~\ref{lemma:hat-phi-satisfies-table}.
\begin{lemma}\label{lemma:tweaked-l1-norm}
We estimate
\begin{align}\label{tweaked-l1-norm}
    \lVert \hat \varphi \rVert_{L^1} \leq \frac{e^2 r}{|\int \varphi(x) \mathrm{d}x |} \lVert \varphi \rVert_{L^1}.
\end{align}
\end{lemma}

\begin{proof}
    We start with 
    \begin{align*}
        \lVert \hat \varphi \rVert_{L^1} = \int |\hat \varphi(x)| \mathrm{d}x = \int |\mathcal{T}_{\varphi}(x)| \; \mathrm{d}x \leq |a| \sum^{r-1}_{i=0} |c_i| \int |\varphi^{\lambda_i}(x)| \mathrm{d}x
    \end{align*}
    where \(c_i = \prod_{k \in \{0,\ldots,r-1\} \setminus\left\{ i \right\} } \frac{\lambda_k}{\lambda_k - \lambda_i}\) and \(\lambda_k = \frac{2^{-(k+1)}}{1+R_\varphi}\). So, 
    \begin{align*}
        |c_i| = \left|\prod_{k \in \{0,\ldots,r-1\} \setminus\left\{ i \right\} } \frac{2^{-(k+1)}}{2^{-(k+1)} - 2^{-(i+1)}}\right| = \prod_{k \in \{0,\ldots,r-1\} \setminus\left\{ i \right\} } \frac{1}{|1 - 2^{k - i}|}.
    \end{align*} 
    Since \(|1 - 2^{k - i}| \geq 1\) for all \(k > i\), we have   
    \begin{align*}
        |c_i| = \prod_{k \in \{0,\ldots,r-1\} \setminus\left\{ i \right\} } \frac{1}{|1 - 2^{k - i}|} \leq \prod^\infty_{k=1} \frac{1}{1 - 2^{-m}}
    \end{align*}
    Note that \((1-x)^{-1} \leq 1 + 2x \leq e^{2x}\) for \(x \in [0,\frac{1}{2}]\). So, by substituting \(2^{-m} \leadsto x\), we get 
    \begin{align}\label{jsknfjkewfwhiru}
        |c_i| \leq \prod^\infty_{m=1} 1 + 2^{-m} \leq e^2
    \end{align}
    Then, using \(a = \frac{1}{\int \varphi(x)\mathrm{d}x}\) and \( \lVert \varphi^{\lambda_i} \rVert_{L^1} =  \lVert \varphi \rVert_{L^1} \)  we end up with 
    \begin{align*}
        \lVert \hat \varphi \rVert_{L^1} \leq |a| \sum^{r-1}_{i=0} |c_i| \int |\varphi^{\lambda_i}(x)| \mathrm{d}x \leq \frac{1}{|\int \varphi(x) \mathrm{d}x|} e^2r \lVert \varphi \rVert_{L^1}.
    \end{align*}
\end{proof}

Estimating the convolution of the tweaked test function \(\check \varphi\) with some test function \(g\) is also of interest for us (especially in Step 3). 

\begin{lemma}\label{step3:lemma}
    Let \(K \subset \mathbb{R}^d\) be a compact set, and let \(g \in \mathcal{D}(K)\). For any \(\epsilon > 0\), the function \(\check \varphi^\epsilon * g\) is supported in \(\enlarg{H}{\epsilon}\), and we estimate     
    \begin{align*}
        \lVert \check \varphi^{\epsilon} * g \rVert_{L^1} \leq \mathrm{Vol}(\enlarg{H}{\epsilon})  \lVert \check \varphi \rVert_{L^1} \lVert g \rVert_{C^r} \epsilon^r.
    \end{align*}
\end{lemma}

\begin{proof}
The core idea is to utilize the annihilation property of \(\check \varphi\). Let \(\epsilon > 0\), and let \(T_y g\) be the Taylor polynomial of \(g\) of order \((r-1)\) at \(y \in \mathbb{R}^d\), i.e. \(T_y g(x) = \sum\limits_{|k| \leq r - 1}\frac{1}{k!}\partial^k g(y)(x-y)^k\). We estimate the error term of the Taylor polynomial to be bounded by 
\begin{align*}
    |g(x) - T_y g(x)| \leq \lVert g \rVert_{C^r} |x-y|^r.
\end{align*}
This follows from the fact that the error term can be explicitly given by \(\frac{1}{r!}\partial^{r}g(\xi)(x - y)^r\) for some \(\xi\) between \(x\) and \(y\). Next, by the annihilation property of \(\check \varphi\) we have 
\begin{align*}
    \int\check \varphi^{\epsilon}(x - y)T_{x} g(y)  \, \mathrm{d}y = 0.
\end{align*}
Hence, we write 
\begin{align*}
        |\check \varphi^{\epsilon}* g(x)|  = \left|  \int\check \varphi^{\epsilon}(x - y) (g(y) - T_x g(y) )  \, \mathrm{d}y \right|
        &\leq \int |\check \varphi^{ \epsilon }(x-y)| \lVert g \rVert_{C^r} |x-y|^r \, \mathrm{d}y \\
        &\Downarrow \text{\(\mathrm{supp}(\check \varphi^{\epsilon}) \subset B(0,\epsilon)\)} \\
        &\leq \lVert \check \varphi \Vert_{L^1 } \lVert g \rVert_{C^r} \epsilon^r.
\end{align*}
Since \(g\) has compact support in \(H\) and \(\check \varphi\) in \(B(0,1)\), \(\check \varphi^{\epsilon}* g\) is supported in \(\enlarg{H}{\epsilon}\). So, we integrate \(|\check \varphi^{\epsilon}* g|\) over \(\enlarg{H}{\epsilon}\), and we obtain the claim.
\end{proof}

For the last lemma, consider the problem of finding an upper bound for the convolution of any test function \(\psi\) with \(\hat \varphi\) or \(\check \varphi\) against some arbitrary function \(G\).   

\begin{lemma}\label{lemma:WALFANGER}
    Let \(\lambda, \epsilon \in (0,1]\), \(K \subset \mathbb{R}^d\) be compact and \(G: \mathbb{R}^d \to \mathbb{R}\) be measurable. Then, for any \(x \in K\) and \(\psi \in \mathcal{B}_r\) we have 
    \begin{align} 
        &\Big| \int_{\mathbb{R}^d} G(y) \left( \hat \varphi^{2\epsilon} * \psi^\lambda_x \right) (y) \, \mathrm{d}y \Big| \leq 2^d \lVert \hat \varphi \rVert_{L^1} \, \sup_{B(x, \lambda + \epsilon)}|G|, \\
        &\Big| \int_{\mathbb{R}^d} G(y) \left( \check \varphi^{\epsilon} * \psi^\lambda_x \right) (y) \, \mathrm{d}y \Big| \leq 4^d \lVert \check \varphi \rVert_{L^1} \, \min\left\{ \left(\frac{\epsilon}{\lambda}\right)^r, 1 \right\}  \sup_{B(x, \lambda + \epsilon)}|G| \label{eq:WALFANGER}
    \end{align}  
\end{lemma}

\begin{proof}
    Regarding the first inequality, note that \(\hat \varphi^{2\epsilon} * \psi^\lambda_x\) has support in \(B(x, \lambda + \epsilon)\) because \(\psi\) is supported in \(B(0,1)\) (due to \(\psi \in \mathcal{B}_r\)) and \(\hat \varphi \in B(0, \frac{1}{2})\). So, 
    \begin{align*}
        \Big| \int_{\mathbb{R}^d} G(y) \left( \hat \varphi^{2\epsilon} * \psi^\lambda_x \right) (y) \, \mathrm{d}y \Big| \leq \lVert \hat \varphi^{2\epsilon} * \psi^\lambda_x \rVert_{L^1} \sup_{B(x, \lambda + \epsilon)}|G| \overset{\eqref{inequality:l1-norm}}{\leq} \lVert \hat \varphi^{2\epsilon} \rVert_{L^1} \lVert \psi^\lambda_x \rVert_{L^1} \sup_{B(x, \lambda + \epsilon)}|G|.
    \end{align*}
    \(\lVert \psi \rVert_{L^1}\) is bounded by the volume of the unit ball in \(\mathbb{R}^d\) because \(\lVert \psi \rVert_{\infty} \leq 1\) and \(\psi\) has support in \(B(0,1)\). Hence, 
    \begin{align}\label{bound-psi-l1}
        \lVert \psi \rVert_{L^1} \leq 2^d.
    \end{align}
    With \(\lVert \hat \varphi^{2\epsilon} \rVert_{L^1} = \lVert \hat \varphi\rVert_{L^1}\), we obtain the first inequality.

    For the second inequality, we use the exact same argument as for the first inequality to obtain \(\Big| \int_{\mathbb{R}^d} G(y) \left( \check \varphi^{\epsilon} * \psi^\lambda_x \right) (y) \, \mathrm{d}y \Big| \leq 2^d \lVert \check \varphi \rVert_{L^1} \,  \sup\limits_{B(x, \lambda + \epsilon)}|G|\). This yields the case \(\lambda \leq \epsilon\). 
    
    If \(\epsilon < \lambda\), we get an even sharper bound. We use Lemma~\ref{step3:lemma} to get  
    \begin{align*}
        \lVert \check \varphi^{\epsilon} * \psi^\lambda_x \rVert_{L^1} \leq \mathrm{Vol}(B(x, \lambda + \epsilon)) \lVert \psi^\lambda_x \rVert_{C^r} \epsilon^r \lVert \check \varphi \rVert_{L^1}.
    \end{align*}
    The ball \(B(x,\lambda  + \epsilon)\) has diameter \(2(\lambda + \epsilon)\), so its volume is smaller than \((2(\lambda + \epsilon))^d\). Since \(\epsilon < \lambda\), we get \(\mathrm{Vol}(B(x,\lambda  + \epsilon)) \leq 4^d \lambda^d\). The \(C^r\)-norm of \(\psi^\lambda_x\) can be computed by
    \begin{align*}
        \lVert \psi^\lambda_x \rVert_{C^r} = \max_{|k| \leq r} \lVert \partial^k \psi^\lambda_x \rVert_{\infty} =  \max_{|k| \leq r} \frac{1}{\lambda^{d + |k|}} \lVert \psi \rVert_{\infty} \leq \frac{1}{\lambda^{d + r}}.
    \end{align*}    
    This completes the proof of the second inequality.
\end{proof}

\section{Step 2: Decomposition}\label{step2:decomposition}

Let \(\psi \in \mathcal{D}\) be any test function. With the right mollifier in our toolkit, we can decompose the approximating distribution \(f_n(\psi) = \int_{\mathbb{R}^d} F_z(\rho_z^{\epsilon_n}) \psi(z)\, \mathrm{d}z\) into \(f_n(\psi) = f_1(\psi) + \sum^{n-1}_{k=1} g_k(\psi)\), where
\begin{align*}
    g_k(\psi) \coloneqq f_{k+1}(\psi) - f_k(\psi) 
    &= \int_{\mathbb{R}^d} F_z(\rho_z^{\epsilon_{k+1}} - \rho_z^{\epsilon_k}) \psi(z)\, \mathrm{d}z \\
    &\overset{\eqref{rho-dif}}{=} \int_{\mathbb{R}^d} F_z((\hat \varphi^{\epsilon_k} * \check \varphi^{\epsilon_k} )_z) \psi(z)\, \mathrm{d}z \\
    &= \int_{\mathbb{R}^d} \int_{\mathbb{R}^{d}} F_z(\hat{\varphi}^{\epsilon_k}_{y+z}) \check \varphi^
    {\epsilon_k}(y) \psi(z) \, \mathrm{d}y \, \mathrm{d}z \\
    \\[-1.2em]
    &\Downarrow \text{substitute \(y+z\) by a new variable} \\
    \\[-1.2em]
    &= \int_{\mathbb{R}^d} \int_{\mathbb{R}^{d}} F_z(\hat{\varphi}^{\epsilon_k}_{y}) \check \varphi^{\epsilon_k}(y-z) \psi(z) \, \mathrm{d}y \, \mathrm{d}z
\end{align*}
Note that \( \hat \varphi \) is centered around \( y \) and not \( z \). So, we use the triangle inequality and write 
\begin{align*}
    g_k(\psi) &= \iint F_z(\hat{\varphi}^{\epsilon_k}_{y}) \check \varphi^{\epsilon_k}(y-z) \psi(z) \, \mathrm{d}y \, \mathrm{d}z \\
    &=  \iint F_y(\hat{\varphi}^{\epsilon_k}_{y}) \check \varphi^{\epsilon_k}(y-z) \psi(z) \, \mathrm{d}y \, \mathrm{d}z +  \iint (F_z - F_y)(\hat{\varphi}^{\epsilon_k}_{y}) \check \varphi^{\epsilon_k}(y-z) \psi(z) \, \mathrm{d}y \, \mathrm{d}z.
\end{align*}
Thus, we have decomposed \(g_k\) into \(g_k = g_k' + g_k''\) where 
\begin{align*}
    g_k'(\psi) &\coloneqq \iint F_y(\hat{\varphi}^{\epsilon_k}_{y}) \check \varphi^{\epsilon_k}(y-z) \psi(z) \, \mathrm{d}y \, \mathrm{d}z \\
    g_k''(\psi) & \coloneqq \iint (F_z - F_y)(\hat{\varphi}^{\epsilon_k}_{y}) \check \varphi^{\epsilon_k}(y-z) \psi(z) \, \mathrm{d}y \, \mathrm{d}z.
\end{align*}
This allows us to use the coherence condition for \( g''_k \) and the homogeneity bound for \( g'_k \). Finally, we write
\begin{align}\label{approximating-distributions-alternative}
    f_n(\psi)= f_1(\psi) + \sum^{n-1}_{k=1} g_k(\psi) = f_1 + \sum^{n-1}_{k=1} g_k'(\psi) + \sum^{n-1}_{k=1} g_k''(\psi).
\end{align}
Our goal is to show that \(\lim\limits_{n \to \infty}f_n(\psi)\) exists for all \(\psi \in \mathcal{D}\). Hence, our goal for the subsequent chapters is to prove that the series \( \sum g_k'(\psi)\) and \( \sum g_k''(\psi)\) converge.

\begin{remark}\label{remark:fk-depends}
    The choice of our mollifier \(\rho\) depends on \(K\) since \(\rho \coloneqq \hat \varphi^2 * \hat \varphi\), and \(\hat \varphi \coloneqq \). Thus, the approximating distributions depend on TO-DO
\end{remark}

We will give a brief overview over the next steps because now we must be careful about \(\gamma\). Remember that we have fixed a compact set \(K\). Furthermore, notice that the approximating distributions \(f_n\) depend on \(K\) since our mollifier \(\rho\) depends on \(\hat \varphi\), which depends on \(r\), and the value of \(r\) depends on \(\alpha = \alpha_{\bar K_{3/2}}\) and \(\beta = \beta_{\bar K_{3/2}}\).
\begin{itemize}
    \item In step 3, we show that for every \(\gamma \in \mathbb{R}\) the series \(\sum^{\infty}_{k=1} g_k'(\psi)\) converges for all test functions \(\psi \in \mathcal{D}(\bar K_1)\).
    \item In step 4, we show that for \(\gamma > 0\) the series \(\sum^{\infty}_{k=1} g_k''(\psi)\) converges for all test functions \(\psi \in \mathcal{D}(\bar K_1)\). Thus, \(\lim_{n \to \infty} f_n(\psi)\) is well defined for \(\gamma > 0\), and we define 
    \begin{align*}
        f^K \coloneqq \lim_{n \to \infty}f_n.
    \end{align*}
    \item In step 5, we show that for all \(\gamma > 0\) the function \(f^K\) is a distribution on \(\bar K_1\) which satisfies~\eqref{reconstruction-theorem}
    \begin{gather*}
        |(f^K - F_x)(\psi^\epsilon_x)| \leq C \epsilon^\gamma \\ 
        \text{uniformly for \(\psi \in \mathcal{B}_r\), \(x \in K\), \(\epsilon \in (0,1]\)},
    \end{gather*}
    where \(C \coloneqq \mathrm{const}(\alpha, \gamma, r, d, \varphi) \cdot \vertiii{F}^{\mathrm{coh}}_{\bar K_{3/2}, \varphi, \alpha, \gamma}\).
    \item In step 6, we show that for all \(\gamma > 0\), the distributions \(f^K\) are consistent, i.e.
    \begin{align*}
        \text{for } K \subset K': f^{K}(\psi) = f^{K'}(\psi) \quad \forall \psi \in \mathcal{D}(\bar K_1).
    \end{align*}
    This will allow us to build a global distribution \(f \in \mathcal{D}'\).
\end{itemize}
We see that we need to divide the proof in two cases \(\gamma > 0\) and \(\gamma \leq 0\) when we reach step 4. Fortunately, step 4 can be fixed for \(\gamma \leq 0\) and thus the proof for \(\gamma \leq 0\) resembles that of \(\gamma > 0\).


\section{Step 3: \texorpdfstring{\(\sum |g_k'|\) converges for all \(\gamma \in \mathbb{R}\)}{Sum gk converges for all real gamma}}\label{chapter:step-3}

In this step, we want to show that \(\sum |g_k'(\psi)|\) converges for all test functions \(\psi \in \mathcal{D}(\bar K_1)\). To show this, we estimate \(g_k'(\psi) \coloneqq \iint F_y(\hat{\varphi}^{\epsilon_k}_{y}) \check \varphi^{\epsilon_k}(y-z) \psi(z) \, \mathrm{d}y \, \mathrm{d}z\). We make use of the following three facts: 
\begin{enumerate}
    \item a \emph{compact support} argument for \(\check \varphi^{\epsilon_k}\) and \(\psi\), 
    \item a \emph{local homogeneity bound} argument~\eqref{coherence-hat}, and
    \item an \emph{annihilation} argument due to \(\check \varphi\).
\end{enumerate}

Let, \(\psi \in \mathcal{D}(\bar K_1)\). First, we rewrite \(g'_k\) as 
\begin{align}\label{gk-formular}
    g_k'(\psi) = \iint F_y(\hat{\varphi}^{\epsilon_k}_{y}) \check \varphi^{\epsilon_k}(y-z) \psi(z) \, \mathrm{d}y \, \mathrm{d}z = \int F_y(\hat \varphi^{\epsilon_k}_y) \, (\check \varphi^{\epsilon_k} * \psi)(y) \, \mathrm{d}y.
\end{align}
The compact support argument for \(\check \varphi^{\epsilon_k}\) and \(\psi\) works as follows: 
\begin{itemize}
    \item the tweaked test function \(\check \varphi^{\epsilon_k}\) has a compact support in \(B(0, \epsilon_k)\) because \(\check \varphi\) has a compact support in the unit ball \(B(0,1)\); since \(\epsilon_k = 2^{-k} \leq \frac{1}{2}\) for all \(k \in \mathbb{N}\), we have \(\mathrm{supp} \left( \check \varphi^{\epsilon_k} \right) \subset B(0,\frac{1}{2})\);
    \item by assumption \(\psi \in \mathcal{D}(\bar K_1)\), the compact support of \(\psi\) lies in \(\bar K_1\).
\end{itemize}
Thus by Lemma~\ref{step3:lemma}, we have \(\mathrm{supp} \left( \check \varphi^{\epsilon_k} * \psi \right) \subset \bar K_{3/2}\). We therefore obtain 
\begin{align*}
    |g_k'(\psi)| = \left|\int F_y(\hat \varphi^{\epsilon_k}_y) \, (\check \varphi^{\epsilon_k} * \psi)(y) \, \mathrm{d}y\right| 
    \leq \sup_{y \in \overline K_{3/2}} \left(|F_y(\hat \varphi^{\epsilon_k}_y)| \right) \lVert \check \varphi^{\epsilon_k} * \psi \rVert_{L^1}.
\end{align*}
Next, by {local homogeneity bound} argument we have \(
    \sup_{y \in \overline K_{3/2}} |F_y(\hat \varphi^{\epsilon_k}_y)|  \overset{\eqref{coherence-hat}}{\leq}  \hat B \epsilon_k^\beta
\). 
So, we have 
\begin{align*}
    |g_k'(\psi)| \leq \hat B \epsilon_k^\beta \, \lVert \check \varphi^{\epsilon_k} * \psi \rVert_{L^1}
\end{align*}
With Lemma~\ref{step3:lemma}, we finalize step 3 as follows 
\begin{align}
    |g_k'(\psi)| \leq \hat B \epsilon_k^\beta \, \lVert \check \varphi^{\epsilon_k} * \psi \rVert_{L^1} &\leq \hat B \epsilon_k^\beta \,  \mathrm{Vol}(\overline K_{3/2})  \lVert \check \varphi \rVert_{L^1} \lVert \psi \rVert_{C^r} \epsilon^r_k \nonumber \\
    &= \left( \hat B \, \mathrm{Vol}(\overline K_{3/2})  \lVert \check \varphi \rVert_{L^1} \lVert \psi \rVert_{C^r} \right)\epsilon^{\beta + r}_k. \label{Mustermkatze}
\end{align}
In \emph{Step 0: Setup}, we chose \(r\) such that  \(\beta + r > 0\); hence \(\sum_{k=1}^\infty \epsilon_k^{\beta + r} < \infty\). Thus, \(\sum^\infty_{k=1} g_k'(\psi)\) is finite for all \(\gamma \in \mathbb{R}\) and all \(\psi \in \mathcal{D}(\enlarg{K}{1})\).

\begin{remark}
One might wonder why we even needed Lemma~\ref{step3:lemma} if we can just estimate \(\lVert \check \varphi^{\epsilon_k} * \psi \rVert_{L^1} \leq \lVert \check \varphi^{\epsilon_k} \rVert_{L^1} \lVert \psi \rVert_{L^1}\). However, this estimate does not make use of the scaling of \(\check \varphi^{\epsilon_k}\). If we had used that estimate, we would have ended up with \(|g_k'(\psi)| \leq \mathrm{constant} \cdot \epsilon_k^{\beta}\). Since \(\beta\) could be negative, the series need not converge.
\end{remark}

It is worth noting that Step 3 works for \(\gamma > 0\) and \(\gamma \leq 0\). This is not the case for the subsequent steps.  


\chapter{Proof Continued for \texorpdfstring{\(\gamma > 0\)}{gamma > 0}}\label{chapter:proof-gamma-positive}

\section{Step 4: \texorpdfstring{\(\sum |g_k''|\) converges for \(\gamma > 0\)}{sum g''k converges for gamma > 0}}\label{step4}

In Step 3, we showed that the first sum \(\sum_{k=1}^\infty g_k'\) in \(f_n = f_1 + \sum_{k=1}^\infty g_k' + \sum_{k=1}^\infty g_k''\) is finite, but what about the other sum? The answer is simple: it is indeed finite \emph{if} \(\gamma > 0\). The proof is short, and uses a \emph{compact support} argument and the \emph{coherence condition}~\eqref{coherence-hat}.

Let \(\psi \in \mathcal{D}(\overline K_1)\). We start with \(g_k''(\psi) \coloneqq \iint (F_z - F_y)(\hat{\varphi}^{\epsilon_k}_{y}) \check \varphi^{\epsilon_k}(y-z) \psi(z) \, \mathrm{d}y \, \mathrm{d}z\). The tweaked test function \(\check \varphi^{\epsilon_k}\) has compact support in \(B(0,\epsilon_k)\), and \(\psi\) has compact support in \(\overline K_1\). So, we have 
\begin{align*}
    |g_k''(\psi)| 
    &\leq \sup_{\substack{z \in \overline{K}_1 \\ |y-z|\leq \epsilon_k}}
    \left( |(F_z -F_y)(\hat \varphi^{\epsilon_k}_y)| \right) \lVert \check \varphi^{\epsilon_k} * \psi \rVert_{L^1}\\
    &\leq \sup_{\substack{z \in \overline{K}_1 \\ |y-z|\leq \epsilon_k}}
    \left( |(F_z -F_y)(\hat \varphi^{\epsilon_k}_y)| \right) \lVert \check \varphi^{\epsilon_k} \rVert_{L^1} \lVert \psi \rVert_{L^1}.
\end{align*}
Using the coherence condition~\eqref{coherence-hat}, we get
\begin{align}
    |g_k''(\psi)| 
    \leq \hat C \epsilon_k^\alpha (|y-z| + \epsilon_k)^{\gamma - \alpha} \lVert \check \varphi^{\epsilon_k} \rVert_{L^1} \lVert \psi \rVert_{L^1}
    &\leq \hat C \epsilon_k^\alpha (2\epsilon_k)^{\gamma - \alpha} \lVert \check \varphi^{\epsilon_k} \rVert_{L^1} \lVert \psi \rVert_{L^1}\nonumber\\
    &= \left(\hat C  2^{\gamma - \alpha} \lVert \check \varphi \rVert_{L^1} \lVert \psi \rVert_{L^1}\right) \epsilon_k^\gamma. \label{Mustermaus}
\end{align}
Thus, \(\sum |g_k''|\) converges if \(\gamma > 0\).

\section{Step 5: \texorpdfstring{\(f^K\) is a local reconstruction}{fK satisfies the reconstruction theorem}}

In Step 3 and Step 4 we showed that \(\lim\limits_{n \to \infty}f_n(\psi)\) exists for all \(\psi \in \mathcal{D}(\overline{K}_1)\). Hence, we define
\begin{align*}
    f^K \coloneqq \lim\limits_{n \to \infty}f_n \overset{\eqref{approximating-distributions-alternative}}{=} f_1 + \sum^{n-1}_{k=1} g_k' + \sum^{n-1}_{k=1} g_k''.
\end{align*}
The notation \(f^K\) explicitly emphasizes that \(\lim\limits_{n \to \infty}f_n\) depends on the compact set \(K\) that we fixed in \emph{Step 0: Setup}; recall that the construction of the mollifier \(\rho\)  depends on \(K\) (see Remark \ref{remark:fk-depends}).
 
In this step, we prove for \(\gamma > 0\) that \(f^K\) is a reconstruction in the following sense: (1) \(f^K\) is a distribution, and (2) there exists a constant \(C < \infty\) such that 
\begin{gather*}
    |(f^K - F_x)(\psi^\lambda_x)| \leq C \lambda^\gamma \tag{\ref{reconstruction-theorem}
    }\\ 
    \text{uniformly for \(\psi \in \mathcal{B}_r\), \(x \in K\), \(\lambda \in (0,1]\)}.
\end{gather*}
The constant \(C\) will be explicitly computed in~\eqref{constant:rec-gamma-bigger-zero}. 

\subsection*{\(f^K\) is a distribution} 

We want to show that \(f^K \in \mathcal{D}'(\bar K_1)\). \(f_1\) is a distribution. So, we find a constant such that \(|f_1(\psi)| \leq \mathrm{constant} \cdot \lVert \psi \rVert_{C^r}\) for all \(\psi \in \mathcal{D}(\bar K_1)\). Then, we use the upper bounds for \(\sum |g_k'|\) and \(\sum |g_k''|\) established in Step 3 and Step 4 (see~\eqref{Mustermkatze} and~\eqref{Mustermaus}) to find a constant such that  
\begin{align*}
    |f^K(\psi)| \leq \mathrm{constant} \cdot \left\{ \lVert \psi\rVert_{L^1} + \lVert \psi\rVert_{C^r} \right\} \leq  \mathrm{constant} \cdot \left\{  \mathrm{Vol}(\bar K_{3/2}) + 1 \right\} \lVert \psi \rVert_{C^r}.
\end{align*}
This proves that \(f^K\) is a distribution. 


\subsection*{Setup} 

We fix \(\psi \in \mathcal{B}_r, x \in K\) and \(\lambda \in (0,1]\). We define a function \(\tilde f\) that measures the error of the reconstruction \(f^K\) to \(F_x\):  
\begin{align*}
    \tilde f(\phi) &= f^K(\phi) - F_x(\phi), \quad \qquad \phi \in \mathcal{D}(\overline{K}_1)\\
    \tilde f_n(\phi) &= f_n(\phi) - F_x(\phi * \rho^{\epsilon_n}), \quad n \in \mathbb{N}.
\end{align*}
By Lemma \ref{mollifier-lemma}, \(\tilde f_n(\phi)\) converges to \(\tilde f(\phi)\) as \(n \to \infty\).

Let \(N \in \mathbb{N}\) be the smallest index such that \(\epsilon_N \leq \lambda\), i.e. \(N = \min \left\{ k \in \mathbb{N} : \epsilon_k \leq \lambda  \right\}\). Using the triangle inequality, we then write 
\begin{align*}
    |\tilde f (\psi^\lambda_x)| \leq  |\tilde f_N (\psi^\lambda_x)| + |(\tilde f - \tilde f_N) (\psi^\lambda_x)|.
\end{align*}

\subsection*{Bounding \(|\tilde f_N (\psi^\lambda_x)|\)} 

We start with
\begin{align*}
    |\tilde f_N (\psi^\lambda_x) |
    = |f_N(\psi^\lambda_x) - F_x(\psi^\lambda_x * \rho^{\epsilon_N}) |
    &= |\int_{\mathbb{R}^d} F_y(\rho_y^{\epsilon_N}) \psi^{\lambda}_x(y) \, \mathrm{dy} - F_x(\psi^\lambda_x * \rho^{\epsilon_N})| \\
    &\Downarrow \text{Lemma \ref{lemma:mollified-distribution}} \\
    &= |\int_{\mathbb{R}^d} F_y(\rho_y^{\epsilon_N}) \psi^{\lambda}_x(y) \, \mathrm{dy} - \int_{\mathbb{R}^d} F_x(\rho_y^{\epsilon_N}) \psi^{\lambda}_x(y) \, \mathrm{dy}| \\
    &= |\int_{\mathbb{R}^d} (F_y - F_x)((\hat \varphi^{2\epsilon_N} * \hat \varphi^{\epsilon_N})(\cdot - y)) \psi^{\lambda}_x(y) \, \mathrm{dy}|\\
    &\Downarrow \text{Lemma \ref{lemma:mollified-distribution}} \\
    &= |\iint (F_y - F_x)(\hat \varphi^{\epsilon_N}_z) \hat \varphi^{2 \epsilon_N}(z-y)\psi^\lambda_x(y) \, \mathrm{d}z \, \mathrm{d}y| \\
    &\leq\sup_{\substack{y \in \overline K_1,\\|z-y|\leq\epsilon_N}} |(F_y - F_x)(\hat \varphi^{\epsilon_N}_z)| \cdot \lVert \hat \varphi^{2\epsilon_N}\rVert_{L^1} \, \lVert \psi^{\lambda}_x\rVert_{L^1}.
\end{align*}
In the last line, we used a \emph{compact support} argument as in step 4. Note that we have the term \(\sup|(F_y - F_x)(\hat \varphi^{\epsilon_N}_z)|\), which subtly indicates that we need to use~\eqref{coherence-hat}. To do so, we write for all \(y \in  B(x,\epsilon),|z-y|\leq\epsilon_N\)
\begin{align*}
    |(F_y - F_x)(\hat \varphi^{\epsilon_N}_z)| &\leq |(F_y - F_z)(\hat \varphi^{\epsilon_N}_z)| + |(F_z - F_x)(\hat \varphi^{\epsilon_N}_z)| \\
    &\leq \hat C \epsilon_N^\alpha(|z-y| + \epsilon_N)^{\gamma - \alpha} + \hat C \epsilon_N^\alpha(|z-x| + \epsilon_N)^{\gamma - \alpha} \\
    &\leq \hat C \epsilon_N^\alpha (2\epsilon_N)^{\gamma - \alpha} + \hat C \epsilon_N^\alpha (|z-y| + |y-x| + \epsilon_N)^{\gamma - \alpha} \\
    &\leq \hat C  2^{\gamma - \alpha} \lambda^{\gamma} +  \hat C  \lambda^\alpha(3\lambda)^{\gamma - \alpha} \\
    &= 2 \hat C 3^{\gamma - \alpha}\lambda^\gamma.
\end{align*}
So, we obtain
\begin{align*}
    |\tilde f_N (\psi^\lambda_x) | 
    &\leq 2 \hat C 3^{\gamma - \alpha}\lambda^\gamma \lVert \hat \varphi^{2\epsilon_N}\rVert_{L^1} \, \lVert \psi^{\lambda}_x\rVert_{L^1}\\
     &= 2 \hat C 3^{\gamma - \alpha}\lambda^\gamma \lVert \hat \varphi\rVert_{L^1} \, \lVert \psi^{\lambda}_x\rVert_{L^1} \\
     &\Downarrow \text{\(\lVert \psi^{\lambda}_x\rVert_{L^1} \leq 2^d\), see~\eqref{bound-psi-l1}} \\
     &\leq \left \{ 3^{\gamma - \alpha} 2^{d+1} \hat C  \lVert \hat \varphi\rVert_{L^1} \right \} \lambda^\gamma.
\end{align*}

\subsection*{Bounding \(|(\tilde f - \tilde f_N) (\psi^\lambda_x)|\)} 

We begin with
\begin{align*}
    |(\tilde f - \tilde f_N) (\psi^\lambda_x)| = |\lim_{n \to \infty} (\tilde f_n - \tilde f_N) (\psi^\lambda_x)| \leq \sum_{k \geq N}|(\tilde f_{k+1} - \tilde f_k)(\psi^\lambda_x)|.
\end{align*}
By definition \(\tilde f_k(\psi) = f_k(\psi) - F_x(\psi * \rho^{\epsilon_k})\), we then obtain 
\begin{align*}
    |(\tilde f - \tilde f_N) (\psi^\lambda_x)| 
    &\leq \sum_{k \geq N} |(f_{k+1} - f_k)(\psi^\lambda_x) - F_x(\psi^\lambda_x * (\rho^{\epsilon_{k+1}} - \rho^{\epsilon_k}))| \\
    &\Downarrow \text{by definition \(f_k = f_1 + \sum^{k-1}_{j=1} g_j' + \sum^{k-1}_{j=1}g_j''\)}\\
    &\leq
    \sum_{k \geq N} |(g_k' + g_k'')(\psi^{\lambda}_x) - F_x(\psi^\lambda_x * (\rho^{\epsilon_{k+1}} - \rho^{\epsilon_k}))| \\
    &\leq \sum_{k \geq N} \underbrace{|g_k''(\psi^{\lambda}_x)|}_{\text{(A)}} + \underbrace{ |g_k'(\psi^\lambda_x) - F_x(\psi^\lambda_x * (\rho^{\epsilon_{k+1}} - \rho^{\epsilon_k}))|}_{\text{(B)}}.
\end{align*}
We are going to bound (A) and (B) separately. Our short-term goal is to find a constant such that \(|(\tilde f - \tilde f_N) (\psi^\lambda_x)| \leq \{\mathrm{constant}\} \cdot \lambda^{\gamma}\). 

\vspace{0.5em}

\begin{itemize}
    \item (A): We start with (A) that follows almost immediately from Step 4. In Chapter \ref{step4}: Step 4 we showed that \(|g_k''(\psi^{\lambda}_x)| 
    {\leq}  \left(\hat C  2^{\gamma - \alpha} \lVert \check \varphi \rVert_{L^1} \lVert \psi^{\lambda}_x \rVert_{L^1}\right) \epsilon_k^\gamma\). We also know that \(\lVert \psi^\lambda_x \rVert_{L^1} = \lVert \psi \rVert_{L^1} \leq 2^d\) by~\eqref{bound-psi-l1}. Then, we have
    \begin{align*}
        \sum_{k \geq N} |g_k''(\psi^{\lambda}_x)|  \leq 
        \left(\hat C  2^{\gamma - \alpha} \lVert \check \varphi \rVert_{L^1} \lVert \psi^{\lambda}_x \rVert_{L^1}\right) \sum_{k \geq N}  \epsilon_k^\gamma 
        \leq \left(\hat C  2^{\gamma - \alpha + d} \lVert \check \varphi \rVert_{L^1} \right) \sum_{k \geq N}  \epsilon_k^\gamma.
    \end{align*}
    The geometric series \(\sum\limits_{k \geq N}  \epsilon_k^\gamma\) converges because \(\gamma > 0\), and therefore
    \begin{align*}
        \sum_{k \geq N} |g_k''(\psi^{\lambda}_x)| \leq 
        \frac{\hat C  2^{\gamma - \alpha + d} \lVert \check \varphi \rVert_{L^1}}{1-2^{-\gamma}} \epsilon_N^{\gamma}
        \leq
        \left\{\frac{\hat C  2^{\gamma - \alpha + d} \lVert \check \varphi \rVert_{L^1}}{1-2^{-\gamma}}\right\} \lambda^{\gamma}.
    \end{align*}
 
    \item (B): Next, we bound \(|g_k'(\psi^\epsilon_x) - F_x(\psi^\epsilon_x * (\rho^{\epsilon_{k+1}} - \rho^{\epsilon_k}))|\). For that we need a technical lemma that also turns out to be useful in the case \(\gamma \leq 0\). Hence, we state it as a lemma here, so we can refer to it in later chapters.
    

    \begin{lemma}\label{technical-lemma-2}
        For any \(\gamma \in \mathbb{R}\) we have 
        \begin{align*}
            |g_k'(\psi^\lambda_x) - F_x(\psi^\lambda_x * (\rho^{\epsilon_{k+1}} - \rho^{\epsilon_k}))| \leq 4^{d + \gamma - \alpha} \hat C \lVert \check \varphi \rVert_{L^1} \begin{cases}
                \lambda^{\gamma - \alpha - r} \epsilon_k^{\alpha + r}  \quad &\text{if \(\epsilon_k < \lambda\) }\\
                \epsilon_k^\gamma & \text{if \(\epsilon_k \geq \lambda\) }
            \end{cases}
        \end{align*}
        and 
        \begin{align*}
            \sum_{k \geq N} |g_k'(\psi^\lambda_x) - F_x(\psi^\lambda_x * (\rho^{\epsilon_{k+1}} - \rho^{\epsilon_k}))|
            \leq
            \left \{ \frac{\hat C 4^{\gamma - \alpha + d} \lVert \check \varphi \rVert_{L^1} }{1-2^{-\alpha - r}} \right \} \lambda^{\gamma}.
        \end{align*}
        Recall that \(\epsilon_k = 2^{-k}\) and \(N = \min\left\{ k \in \mathbb{N}: \epsilon_k \leq \epsilon \right\}\).
    \end{lemma}

    \begin{proof}
        By~\eqref{rho-dif} and Corollary \ref{cor:minosokoad}, we obtain 
        \begin{align*}
            F_x(\psi^\lambda_x * (\rho^{\epsilon_{k+1}} - \rho^{\epsilon_k})) &= \iint F_x(\hat \varphi^{\epsilon_k}_y) \check \varphi^{\epsilon_k}(y-z) \psi^{\lambda}_x(z) \, \mathrm{d}y \mathrm{d}z\\
             &= \int F_x(\hat \varphi^{\epsilon_k}_y) ( \check\varphi^{\epsilon_k} * \psi^{\lambda}_x)(y) \, \mathrm{d}y.
        \end{align*}
        By definition of \(g'_k\) (see~\eqref{gk-formular}), we have 
        \begin{align*}
            g_k'(\psi^\lambda_x) = \int F_y(\hat \varphi^{\epsilon_k}_y) (\check \varphi^{\epsilon_k}* \psi^\lambda_x)(y) \, \mathrm{d}y.
        \end{align*}
        Hence, we get
        \begin{align}\label{temp:klwmrlkewmrw}
            |g_k'(\psi^\lambda_x) - F_x(\psi^\lambda_x * (\rho^{\epsilon_{k+1}} - \rho^{\epsilon_k}))| 
            &= \int (F_y-F_x)(\hat \varphi^{\epsilon_k}_y) (\check \varphi^{\epsilon_k}* \psi^\lambda_x)(y) \, \mathrm{d}y,
        \end{align}
        for which we find an upper bound using the second inequality of Lemma \ref{lemma:WALFANGER}
        \begin{align*}
           ~\eqref{temp:klwmrlkewmrw} \leq 4^d \lVert \check \varphi \rVert_{L^1} \min \left\{ \frac{\epsilon_k}{\lambda}, 1 \right\}^r \sup_{y \in B(x, \lambda + \epsilon_k)} |(F_y - F_x)(\hat \varphi_y^{\epsilon_k})|.
        \end{align*}
        The supremum of \(|F_y-F_x|\) is estimated with the coherence condition
        \begin{align*}
            \sup_{y \in B(x, \lambda + \epsilon_k)} |(F_y - F_x)(\hat \varphi^{\epsilon_k}_y)| &\overset{\eqref{coherence-hat}}{\leq} \hat C \epsilon_k^{\alpha} \sup_{y \in B(x, \lambda + \epsilon_k)} (|x - y| + \epsilon_k )^{\gamma - \alpha} \\
            &\;\leq \hat C \epsilon^\alpha_k(\lambda + 2\epsilon_k)^{\gamma - \alpha} \\
            &\leq \hat C \epsilon^\alpha_k 3^{\gamma - \alpha} \begin{cases}
                \epsilon_k^{\gamma - \alpha} &\text{if \(\epsilon_k > \lambda\) } \\
                \lambda^{\gamma - \alpha} &\text{if \(\epsilon_k \leq \lambda\) }
            \end{cases}
        \end{align*}
        This proves the first claim of the lemma.

        To prove the last line, note that \(\sum\limits_{k \geq N} \epsilon_k^{\alpha + r} = \frac{\epsilon_N^{\alpha+ r}}{1 - 2^{-\alpha - r}} \leq \frac{\lambda^{\alpha + r}}{1 - 2^{- \alpha - r}}\)  (regarding the last inequality, we chose \(r \in \mathbb{N}\) such that  \(\alpha + r > 0\)). Since \(k \geq N\), we have \(\epsilon_k < \lambda\). So, we use the first claim of the lemma together with \(\sum\limits_{k \geq N} \epsilon_k^{\alpha + r} \leq \frac{\lambda^{\alpha + r}}{1 - 2^{- \alpha - r}}\) to obtain
        \begin{align*}
            \sum_{k \geq N} |g_k'(\psi^\lambda_x) - F_x(\psi^\lambda_x * (\rho^{\epsilon_{k+1}} - \rho^{\epsilon_k}))| \leq \left\{\frac{4^{d + \gamma - \alpha}}{1-2^{-\alpha - r}} \lVert \check \varphi \rVert_{L^1} \hat C \right\} \lambda^{\gamma}. 
        \end{align*} 
    \end{proof}
    With this lemma proved, we use the second line to find an upper bound for (B).
\end{itemize}

Finally, we use (A) and (B) to estimate
\begin{align*}
    |(\tilde f - \tilde f_N) (\psi^\lambda_x)| 
    &\leq
    \left \{ \frac{\hat C 4^{\gamma - \alpha + d} \lVert \check \varphi \rVert_{L^1} }{1-2^{-\alpha - r}} \right \} \lambda^{\gamma} + \left\{\frac{\hat C  2^{\gamma - \alpha + d} \lVert \check \varphi \rVert_{L^1}}{1-2^{-\gamma}}\right\} \lambda^{\gamma} \\
    &\leq \left\{ 2 \frac{\hat C 4^{\gamma - \alpha + d} \lVert \check \varphi \rVert_{L^1} }{1-2^{-\min\{\alpha + r, \gamma\}}} \right\}\cdot \lambda^{\gamma} \\
    &\Downarrow \text{\(\lVert \check \varphi \rVert_{L^1} \leq 2\lVert \hat \varphi \rVert_{L^1} \) by definition of \(\check \varphi\)} \\
    &\leq \left\{ \frac{\hat C 4^{\gamma - \alpha + d + 1} \lVert \hat \varphi \rVert_{L^1} }{1-2^{-\min\{\alpha + r, \gamma\}}} \right\}\cdot \lambda^{\gamma}.
\end{align*}

\subsection*{Finish}

Now, we are finally able to prove that \(f^K\) is a local reconstruction. We show that \(f^K\) satisfies~\eqref{reconstruction-theorem} uniformly for all \(\psi \in \mathcal{B}_r, x \in K\) and \(\lambda \in (0,1]\). We have
\begin{align*}
    |f^K(\psi^\lambda_x) - F_x(\psi^\lambda_x)| = |\tilde f (\psi^\lambda_x)| 
    &\leq  |\tilde f_N (\psi^\lambda_x)| + |(\tilde f - \tilde f_N) (\psi^\lambda_x)| \\
    &\leq \left \{ 3^{\gamma - \alpha} 2^{d+1} \hat C  \lVert \hat \varphi\rVert_{L^1} \right \} \lambda^\gamma +  \left\{ \frac{ 4^{\gamma - \alpha + d + 1} \hat C \lVert \hat \varphi \rVert_{L^1} }{1-2^{-\min\{\alpha + r, \gamma\}}} \right\}\cdot \lambda^{\gamma}  \\
    &\leq  \left\{2\frac{  4^{\gamma - \alpha + d + 1} \hat C \lVert \hat \varphi \rVert_{L^1}}{1-2^{-\min\{\alpha + r, \gamma\}}} \right\} \lambda^{\gamma}.
\end{align*}
This proves that \(f^K\) is a local reconstruction of \((F_x)_{x \in \mathbb{R}^d}\).

For completeness, we estimate \(\hat C \lVert \hat \varphi \rVert_{L^1}\). Lemma \ref{lemma:tweaked-l1-norm} gives us \(\lVert \hat \varphi \rVert_{L^1} \leq \frac{e^2 r \lVert \varphi \rVert_{L^1}}{|\int \varphi(x) \mathrm{d}x |} \). The bound for \(\hat C\) in~\eqref{constant:hat-c} then yields 
\begin{align}\label{chatnorml1}
    \hat C \lVert \hat \varphi \rVert_{L^1} \leq \left(C\frac{e^2 r}{|\int \varphi(x)\, \mathrm{d}x|} \left(\frac{2^{-(r+1)}}{1+R_\varphi}\right)^\alpha\right)\left( \frac{e^2 r}{|\int \varphi(x) \mathrm{d}x |} \lVert \varphi \rVert_{L^1}\right).
\end{align}
So, the constant factor in front of \(\lambda^\gamma\) reads 
\begin{align*}
    \left\{2\frac{  4^{\gamma - \alpha + d + 1}}{1-2^{-\min\{\alpha + r, \gamma\}}} \left(C\frac{e^2 r}{|\int \varphi(x)\, \mathrm{d}x|} \left(\frac{2^{-(r+1)}}{1+R_\varphi}\right)^\alpha\right) \left(\frac{e^2 r}{|\int \varphi(x) \mathrm{d}x |} \lVert \varphi \rVert_{L^1}\right) \right\}.
\end{align*}
Finally, we bound \(e^2 \leq 4\) to obtain the constant
\begin{align}\label{constant:rec-gamma-bigger-zero}
    \left\{C\frac{  4^{\gamma - \alpha + d + 6}r^2}{1-2^{-\min\{\alpha + r, \gamma\}}} \frac{\lVert \varphi \rVert_{L^1}}{2^{\alpha(r+1)} (1+R_\varphi)^{\alpha}|\int \varphi(x) \mathrm{d}x |^2}  \right\},
\end{align} 
where \(R_\varphi\) is the radius of the ball such that \(\mathrm{supp}(\varphi) \subset B(0, R_\varphi)\).  


\section{Step 6: \texorpdfstring{\((f^K)\) are consistent}{fK's are consistent}}

In this final step, we construct a global reconstruction \(f = \mathcal{R}F\) of \((F_x)_{x \in \mathbb{R}^d}\)  using the local reconstructions \(f^K\). First, we prove that the family of local reconstructions \((f^K)_{K \subset \mathbb{R}^d, \text{\(K\) compact}}\) is consistent in the sense of 
\begin{align*}
    \forall \text{compact sets \(K\) and \(H\) with } K \subset H: \psi \in \mathcal{D}(\bar K_1) \implies f^K(\psi) = f^H(\psi).
\end{align*}
To prove this claim, we use the Uniqueness Theorem \ref{theorem:uniqueness-reconstruction}. We have \((f^K - F_x)(\varphi^\lambda_x) \to 0\) and \((g^K - F_x)(\varphi^\lambda_x) \to 0\) as \(\lambda \to 0\) uniformly for \(x \in \bar K_1\) because of \(\gamma > 0\) and~\eqref{reconstruction-theorem}. Hence, we apply the Uniqueness Theorem and conclude that \(f^K(\psi) = f^H(\psi)\) for all \(\psi \in \mathcal{D}(\bar K_1)\).  

With the consistency property being shown, we move on to construct a global reconstruction. Let \(\psi \in \mathcal{D}\). Then, we define \(\mathcal{R}F: \psi \mapsto f^K(\psi)\) where \(K \subset \mathbb{R}^d\) is a compact set large enough such that \(\psi\) is supported in \(\bar K_1\). The map \(\mathcal{R}F\) is well-defined because we showed the consistency property. Moreover, \(\mathcal{R}F\) is a reconstruction in the sense of~\eqref{reconstruction-theorem} because \(f^K\) is a local reconstruction. This ends the proof of the Reconstruction Theorem in the case \(\gamma > 0\). 




\chapter{Proof Continued for \texorpdfstring{\(\gamma \leq 0\)}{gamma <= 0}}\label{chapter:proof-gamma-negative}

The main idea of the proof for \(\gamma > 0\) was that \(f_n\) converges to some reconstruction \(f = \mathcal{R}f\) if \(\gamma > 0\), where \(f_n = f_1 + \sum\limits^{n-1}_{k=1} g_k'  + \sum\limits^{n-1}_{k=1} g_k''\), see~\eqref{approximating-distributions-alternative}.   
If however \(\gamma \leq 0\), the series \(\sum g_k''(\psi)\) need not converge. We fix this by ignoring \(\sum g_k''(\psi)\); the approximating distribution then reads \(f_1 + \sum\limits^{n-1}_{k=1} g_k'(\psi)\). We set 
\begin{align*}
    f^K(\psi) = f_1(\psi) + \lim_{n\to \infty}\sum^{n-1}_{k=1} g_k'(\psi),
\end{align*}
which is well-defined because \(\sum g_k'(\psi)\) converges for all \(\gamma \in \mathbb{R}\) (see Chapter~\ref{chapter:step-3}). Next, \(f^K\) is a distribution on \(\bar K_1\) because TO-DO

In the next steps, we will show that \(f^K\) satisfies~\eqref{reconstruction-theorem}, i.e.
\begin{gather*}
    |(f^K - F_x)(\psi_x^\lambda)| \leq \mathfrak{C} \vertiii{F}^{\mathrm{coh}}_{\bar K_{2},\varphi,\alpha,\gamma} \begin{cases}
        \lambda^\gamma  &\text{if \(\gamma < 0\)}\\
        1 + |\log(\lambda)| \quad &\text{if \(\gamma = 0\) } 
    \end{cases}
    \\
    \text{uniformly for \(\psi \in \mathcal{B}_r\), \(x \in K\) and \(\lambda \in(0,1]\).   }
\end{gather*}
where the constant \(\mathfrak{C}\) is given by TO-DO and TO-Do. We will then spend another chapter to build a \emph{global} distribution \(f \in \mathcal{D}'\) out of the local distributions \(f^K \in \mathcal{D}(\bar K_1)\) such that \(f\) satisfies~\eqref{reconstruction-theorem}, as well.

\section{Step 5: \texorpdfstring{\(f^K\) is a local reconstruction}{fK is a local reconstruction}}
 
We have the same setup as in Chapter~\ref{setup}: Step 0. Let \(K \subset \mathbb{R}^d\) be a compact set, \(x \in K\), \(\psi \in \mathcal{B}_r\) and \(\lambda \in (0,1]\). Then, we have
\begin{align*}
    |(f^K - F_x)(\psi_x^\lambda)|
    &= |((f_1 + \lim_{n\to \infty}\sum_{k=1}^{n-1}g_k') - F_x)(\psi_x^\lambda)| \\
    &= |(f_1 + \lim_{n\to \infty}\sum_{k=1}^{n-1}g_k')(\psi_x^\lambda) - \lim_{n\to \infty} F_x(\psi_x^\lambda* \rho^{\epsilon_n})| \\
    &= \lim_{n \to \infty} |\underbrace{f_1(\psi_x^\lambda) + \left\{ \sum_{k=1}^{n-1}g_k'(\psi_x^\lambda) \right\} - F_x(\psi_x^\lambda* \rho^{\epsilon_n})}_{\coloneqq \bar f_n(\psi_x^\lambda)}|.
\end{align*}  
Next, we write the above expression as a telescopic sum 
\begin{align*}
    |(f^K - F_x)(\psi_x^\lambda)| = 
    \lim_{n \to \infty} |\bar f_n(\psi_x^\lambda)| \leq 
    | \left ( \lim_{n \to \infty} \bar f_n(\psi_x^\lambda) \right ) - \bar f_N(\psi_x^\lambda)|
    + |\bar f_N(\psi_x^\lambda)|
\end{align*}
where \(N\) is chosen such that \(\epsilon_N \leq \lambda < \epsilon_{N - 1}\). The first summand is estimated by Lemma~\ref{technical-lemma-2}
\begin{align*}
    | \left( \lim_{n \to \infty} \bar f_n(\psi_x^\lambda)  \right) - \bar f_N(\psi_x^\lambda)| &\leq
    \sum_{k \geq N} |(\bar f_{k+1} - \bar f_{k})(\psi_x^\lambda)|\\ &= 
    \sum_{k \geq N} |g_k'(\psi^\lambda_x) - F_x(\psi^\lambda_x*(\rho^{\epsilon_{k+1}}-\rho^{\epsilon_k}))| \\
    &\Downarrow \text{Lemma~\ref{technical-lemma-2}} \\
    &\leq \left \{ \frac{\hat C 4^{\gamma - \alpha + d} \lVert \check \varphi \rVert_{L^1} }{1-2^{-\alpha - r}} \right \} \lambda^{\gamma}.
\end{align*}
The second summand \(|\bar f_N(\psi_x^\lambda)|\) is also bounded by Lemma~\ref{technical-lemma-2}
\begin{align*}
    |\bar f_N(\psi_x^\lambda)| &\leq |\bar f_1(\psi_x^\lambda)| + \sum^{N-1}_{k=1} |(\bar f_{k+1}  - \bar f_k)(\psi_x^\lambda)|\\
    &= |\bar f_1(\psi_x^\lambda)| + \sum^{N-1}_{k=1} |g_k'(\psi^\lambda_x) - F_x(\psi^\lambda_x*(\rho^{\epsilon_{k+1}}-\rho^{\epsilon_k}))| \\
    &\Downarrow \parbox{25em}{Lemma~\ref{technical-lemma-2} (use the case \(\epsilon_k \geq \epsilon_{N-1} > \lambda\))}\\
    &\leq  |\bar f_1(\psi_x^\lambda)| + \sum^{N-1}_{k=1} 4^{d+\gamma-\alpha}\hat C \lVert \check \varphi \rVert_{L^1} \epsilon^\gamma_k.
\end{align*}
Next, we observe
\begin{align*}
    |\bar f_1(\psi_x^\lambda)| &= |f_1(\psi_x^\lambda) - F_x(\psi_x^\lambda* \rho^{\epsilon_1})| \\
    &\Downarrow \text{use~\eqref{lemma:mollified-distribution}} \\
    &= \left | \int_{\mathbb{R}^d} F_z(\rho_z^{\epsilon_1}) \psi^\lambda_x(z)  \mathrm{d}z -  \int F_x(\rho_z^{\epsilon_1}) \psi^\lambda_x(z) \mathrm{d} z \right| \\
    &= \left | \int_{\mathbb{R}^d} (F_z - F_x)(\rho_z^{\epsilon_1}) \psi^\lambda_x(z)  \mathrm{d}z\right| \\
    &\Downarrow  \text{Recall \(\rho = \hat \varphi^2 * \hat \varphi\) and use~\eqref{lemma:mollified-distribution}} \\
    &= \left| \iint (F_z - F_x)(\hat \varphi^{\epsilon_1}_y) \hat \varphi^{2\epsilon_1}(y-z)\psi^\lambda_x(z) \; \mathrm{d}y \, \mathrm{d}z \right|.
\end{align*}
The tweaked test function \(\hat \varphi\) has a compact support in \(B(0, \frac{1}{2})\); hence \(\hat \varphi^{2\epsilon}\) is supported in \(B(0, \epsilon_1)\). Thus, the integral is nonzero if \(|y-z| \leq \epsilon_1 = \frac{1}{2}\). Additionally, we have \(|x-z|\leq \lambda\) because \(\psi^\lambda_x(z)\). So, we estimate
\begin{align*}
    |\bar f_1(\psi_x^\lambda)| &\leq  \sup_{\substack{z \in B(x, \lambda) \\ |y-z| \leq \frac{1}{2}}} \left| (F_z - F_x)(\hat \varphi^{\epsilon_1}_y) \right|  \cdot \lVert \hat \varphi^{2\epsilon_1} \rVert_{L^1} \lVert  \psi^{\lambda}_x \rVert_{L^1}.
\end{align*}
Moreover, \(z \in \bar K_1\) (recall that \(x \in K\) and \(\lambda \in (0,1]\)), \(y \in \bar K_{\frac{3}{2}}\) and \(|x-y| \leq |x-z| + |z-y| \leq \frac{3}{2}\). Hence, we use the triangle inequality and the coherence condition to obatin 
\begin{align*}
    \sup_{\substack{z \in B(x, \lambda) \\ |y-z| \leq
     \frac{1}{2}}} \left| (F_z - F_x)(\hat \varphi^{\epsilon_1}_y) \right| 
     &\leq \sup_{\substack{y, z \in \bar K_{3/2} \\ |y-z| \leq \frac{1}{2}}} \left| (F_z - F_y)(\hat \varphi^{\epsilon_1}_y) \right| + \sup_{\substack{x,y \in \bar K_{3/2} \\ |x-y| \leq \frac{3}{2}}} \left| (F_y - F_x)(\hat \varphi^{\epsilon_1}_y) \right| \\
     &\overset{\eqref{coherence-hat}}{\leq} \hat C \epsilon_1^\alpha(|z-y| + \epsilon_1)^{\gamma - \alpha} + \hat C \epsilon_1^\alpha(|y-x| + \epsilon_1)^{\gamma - \alpha}\\
     &\leq \hat C \left(\frac{3}{2}\right)^{\gamma - \alpha } + \hat C \left(\frac{5}{2}\right)^{\gamma - \alpha } \\
     &\leq 2\hat C \cdot  3^{\gamma - \alpha}.
\end{align*}
We bound
\begin{align*}
    |\bar f_1(\psi_x^\lambda)| \leq \left\{ 2\hat C \cdot  3^{\gamma - \alpha} \right\} \lVert \hat \varphi^{2\epsilon_1} \rVert_{L^1} \lVert  \psi^{\lambda}_x \rVert_{L^1} &= \left\{ 2\hat C \cdot  3^{\gamma - \alpha} \right\} \lVert \hat \varphi \rVert_{L^1} \lVert  \psi \rVert_{L^1} \\
    &\leq \left\{ 2\hat C \cdot  3^{\gamma - \alpha} \right\}  \lVert \hat \varphi \rVert_{L^1} \cdot \sup_{\psi \in \mathcal{B}_r} \lVert  \psi \rVert_{L^1} \\
    &\leq \left\{ 2\hat C \cdot  3^{\gamma - \alpha} \right\}  \lVert \hat \varphi \rVert_{L^1} \cdot 2^d \cdot \underbrace{\sup_{\psi \in \mathcal{B}_r} \lVert  \psi \rVert_{\infty}}_{\leq 1} \\
    &\leq 2^{d+1}\hat C \cdot  3^{\gamma - \alpha}  \lVert \hat \varphi \rVert_{L^1}.
\end{align*}
Then, we have \(|\bar f_N(\psi_x^\lambda)| \leq 2^{d+1}\hat C \cdot  3^{\gamma - \alpha}  \lVert \hat \varphi \rVert_{L^1} + \sum^{N-1}_{k=1} 4^{d+\gamma-\alpha}\hat C \lVert \check \varphi \rVert_{L^1} \epsilon^\gamma_k\). Also observe that 
\(\lVert \check \varphi \rVert_{L^1} = \int |\check \varphi(x)| \mathrm{d}x \leq \int|\hat \varphi^{\frac{1}{2}}(x)| + |\hat \varphi^{2}(x)| \mathrm{d}x = 2\int |\hat \varphi(x)| \mathrm{d}x = 2 \lVert \hat \varphi \rVert_{L^1}\). So, we get 
\begin{align*}
|\bar f_N(\psi_x^\lambda)| \leq 4^{d + \gamma - \alpha + 1} \hat C \lVert \hat \varphi \rVert_{L^1} \sum^{N-1}_{k=0}\epsilon^\gamma_k.
\end{align*}
Note that \(\sum^{N-1}_{k=0}\epsilon^\gamma_k\) is a geometric sum which we explicitly compute
\begin{align*}
    \sum^{N-1}_{k=0}\epsilon^\gamma_k = 
    \sum^{N-1}_{k=0}2^{-\gamma k} \leq \begin{dcases}
        \frac{\lambda^\gamma}{1-2^\gamma} \quad &\text{if \(\gamma < 0\) } \\
        \frac{\log(\frac{2}{\lambda})}{\log 2} &\text{if \(\gamma = 0\) }
    \end{dcases}.
\end{align*} 
Finally,
\begin{align*}
    |(f^K - F_x)(\psi_x^\lambda)| &\leq  \frac{\hat C 4^{\gamma - \alpha + d} \lVert \check \varphi \rVert_{L^1} }{1-2^{-\alpha - r}} \lambda^{\gamma} + 4^{d + \gamma - \alpha + 1} \hat C \lVert \hat \varphi \rVert_{L^1}\begin{dcases}
        \frac{\lambda^\gamma}{1-2^\gamma} \quad &\text{if \(\gamma < 0\) } \\
        \frac{\log(\frac{2}{\lambda})}{\log 2} &\text{if \(\gamma = 0\) }
    \end{dcases}.
\end{align*}
If \(\gamma < 0\), then 
\begin{align}
    |(f^K - F_x)(\psi_x^\lambda)| &\;\leq \left\{  \hat C\lVert \hat \varphi \rVert_{L^1}  \frac{4^{\gamma - \alpha + d + 1}}{1 - 2^{-\min\left\{ \alpha + r, - \gamma \right\}}} \right\} \lambda^\gamma \nonumber \\
    &\overset{\eqref{chatnorml1}}{\leq} \left\{ 
        \frac{r^2 2^{\alpha(-r-1)} 4^{d+ \gamma -\alpha + 6} \lVert \varphi \rVert_{L^1}}{1-2^{- \min \left\{ \alpha + r, |\gamma| \right\}} (1+R_{\varphi})^\alpha |\int \varphi(x) \, \mathrm{d}x|^2 }
        \vertiii{F}^{\mathrm{coh}}_{\bar K_{3/2}, \varphi, \alpha, \gamma}.
     \right\}
     \lambda^\gamma. \label{MacbookAirGammaNegative}
\end{align} 
Otherwise if \(\gamma = 0\), then we know \(\log(\frac{2}{\lambda}) (\log{2})^{-1} \leq 2(1 + |\log \lambda|)\). Thus
\begin{align*}
    |(f^K - F_x)(\psi_x^\lambda)| \leq \left\{ 
        \frac{r^2 2^{\alpha(-r-1)} 4^{d -\alpha + 6} \lVert \varphi \rVert_{L^1}}{1-2^{- \min \left\{ \alpha + r, |\gamma| \right\}} (1+R_{\varphi})^\alpha |\int \varphi(x) \, \mathrm{d}x|^2 }
        \vertiii{F}^{\mathrm{coh}}_{\bar K_{3/2}, \varphi, \alpha, \gamma}
     \right\}
     (1 + |\log \lambda|).
\end{align*} 
This shows that \(f^K\) is a local reconstruction. 



% ----------------------------------------



\section{Step 6: \texorpdfstring{Localization}{Localization}}\label{chapter:step6gammaNegative}


Similar to the case \(\gamma > 0\), we need to build a global distribution \(f\) from the local reconstructions \(f^K\). For that, we make use of a localization argument. First, we construct a partition of unity. Fix some test function \(\eta \in \mathcal{D}(B(0, \frac{1}{4}))\) such that \(\eta \geq 0\) on \(B(0, \frac{1}{4})\) and \(\eta \geq 1\) on \(B(0, \frac{1}{8})\). Define 
\begin{align*}
    \xi(x) = \frac{\eta(x)}{\sum\limits_{z \in E} \eta_z(x)} \quad \text{ where } E = \frac{1}{4\sqrt{d}} \mathbb{Z}^d.
\end{align*}
Note that \(\xi_y \in \mathcal{D}(B(y, \frac{1}{4}))\) for every \(y \in \mathbb{R}^d\) and \(\sum\limits_{y \in E} \xi_y \equiv 1\). We call \((\xi_y)_{y \in E}\) a \emph{partition of unity subordinated to the cover \(B(y, \frac{1}{4})_{y \in E}\)}. Define \(B_y = B(y, \frac{1}{4})\). Note that \(B_y\) has diameter \(\frac{1}{2}\). The global reconstruction \(f\) is then defined as
\begin{align*}
    f(\psi) = \sum_{y \in E} f^{B_y}(\xi_y \psi), \quad \forall \psi \in \mathcal{D}.
\end{align*}

Now, we show that \(f\) satisfies the Reconstruction Theorem. Fix a compact set \(K \subset \mathbb{R}^d\), and define 
\begin{align}\label{tropicalGeometryie}
    \alpha = \alpha_{\bar K_2}, \quad \beta = \beta_{\bar K_2}, \quad r > \max\left\{ -\alpha_{\bar K_2}, -\beta_{\bar K_2} \right\}.
\end{align}
Let \(\psi \in \mathcal{B}_r\), \(x \in K\) and \(\lambda \in (0,1]\). For \(\gamma < 0\) we have
\begin{align}\label{AbbaHunter}
    |(f-F_x)(\varphi^\lambda_x)| = |\sum_{y \in E} (f^{B_y} - F_x)(\xi_y \varphi^\lambda_x)| \leq \sum_{y \in E} | (f^{B_y} - F_x)(\xi_y \psi^\lambda_x) |.
\end{align} 
To justify the first equality, note that \(F_x(\varphi^\lambda_x) = F_x(\sum_{y \in E} \xi_y \varphi^\lambda_x) = \sum_{y \in E}F_x(\xi_y \varphi^\lambda_x)\) because \(\sum_{y \in E}\xi_y \equiv 1\).

Next, we make sure that we only sum over a finite number of \(y \in E\). Note that \(\xi_y\) has compact support in \(B(y, \frac{1}{4})\) and \(\psi^\lambda_x\) has compact support in \(B(x, \lambda)\). So, \(\xi_y \psi^\lambda_x \not \equiv 0\) only if \(|y-x| \leq |y - z| + |z-x| \leq \frac{1}{4} + \lambda \leq \frac{5}{4}\). There are at most \((2 \cdot \frac{5}{4} \cdot 4 \sqrt{d} + 1)^d \leq (11 \sqrt{d})^d\) many points \(y \in E\) that satisfy this. 

Then, we write 
\begin{align*}
    | (f^{B_y} - F_x)(\xi_y \psi^\lambda_x) | = | (f^{B_y} - F_x)(\zeta_x^\lambda) |
\end{align*}
for \(\zeta(z) = \xi_y(x + \lambda z) \psi(z)\). We would like to apply~\eqref{MacbookAirGammaNegative} for the compact set \(B_y\) and \(\zeta / \lVert \zeta \rVert_{C^r} \in \mathcal{B}_r\). 
Here, we need to be careful about \(\alpha\), \(\beta\) and \(r\) because we must check that~\eqref{MacbookAirGammaNegative} still holds if we choose \(\alpha\), \(\beta\) and \(r\) as in~\eqref{tropicalGeometryie}. 
Let \(\Gamma = \left\{ y_1,\ldots,y_n \right\} \subset \mathbb{R}^d\) such that \(y_i \cap K \not = \emptyset\) for all \(1 \leq i \leq n\). We have \(\bigcup_{y \in \Gamma}B_y \subset \enlarg{K}{\frac{1}{2}}\) because each ball \(B_y\) has diameter \(\frac{1}{2}\). So, the \(\frac{3}{2}\)-enlargement of \(\bigcup_{y \in \Gamma}B_y\) is contained in \(\enlarg{K}{2}\).
By~\eqref{alpha-monotone} and \(\eqref{beta-monotone}\), we know that the maps \(K \mapsto \alpha_K\) and \(K \mapsto \beta_K\) are monotone. So, we have \(\alpha_{\enlarg{K}{2}} \leq \alpha_{\overline{(\bigcup_{y \in \Gamma}B_y)}_{3/2}}\) and \(\beta_{\enlarg{K}{2}} \leq \beta_{\overline{(\bigcup_{y \in \Gamma}B_y)}_{3/2}}\). This, together with Step 5 (\(\gamma > 0\)), shows that~\eqref{MacbookAirGammaNegative} remains true for \(\alpha, \beta\) and \(r\). Applying~\eqref{MacbookAirGammaNegative} then yields
\begin{align*}
    | (f^{B_y} - F_x)(\zeta_x^\lambda) | = | (f^{B_y} - F_x)(\zeta_x^\lambda / \lVert \zeta \rVert_{C^r}) | \lVert \zeta \rVert_{C^r} &\leq \{ \mathrm{constant} \} \cdot \lVert \zeta \rVert_{C^r} \vertiii{F}^{\mathrm{coh}}_{\enlarg{K}{2}, \varphi, \alpha, \gamma} \lambda^\gamma \\
    &\Downarrow \text{Leibniz Rule and \(\sum_{k=0}^r \binom{r}{k} = 2^r\) }\\
    &\leq  2^r \{ \mathrm{constant} \}\lVert \xi \rVert_{C^r} \lVert \psi \rVert_{C^r} \vertiii{F}^{\mathrm{coh}}_{\enlarg{K}{2}, \varphi, \alpha, \gamma} \lambda^\gamma.
\end{align*} 
Continuing the estimate~\eqref{AbbaHunter} 
\begin{align*}
    |(f-F_x)(\varphi^\lambda_x)| &\overset{\eqref{AbbaHunter}}{\leq} \sum_{y \in E} | (f^{B_y} - F_x)(\xi_y \psi^\lambda_x) | \\ 
    &\;\leq \left\{ (11\sqrt{d})^d \left\{ \mathrm{constant} \right\} \lVert \xi \rVert_{C^r}\lVert \psi \rVert_{C^r}\vertiii{F}^{\mathrm{coh}}_{\enlarg{K}{2}, \varphi, \alpha, \gamma} \right\}\lambda^\gamma.
\end{align*}
The constant before \(\lambda^\gamma\)  then reads
\begin{align*}
   \left\{ 2^r \lVert \xi \rVert_{C^r} (11 \sqrt d)^d \frac{r^2 2^{-(r+1) \alpha} 4^{d + \gamma - \alpha + 6}}{1-2^{- \min\left\{ \alpha + r, -\gamma \right\}} |\int \varphi(x) \, \mathrm{d}x|^2  (1 + R_\varphi)^{\alpha}} \lVert \varphi \rVert_{L^1} \right\} \quad \text{in the case \(\gamma < 0\) }.
\end{align*}
The proof for the case \(\gamma = 0\) is done similarly and gives the constant 
\begin{align*}
    \left\{ 2^r \lVert \xi \rVert_{C^r} (11 \sqrt d)^d \frac{r^2 2^{-(r+1) \alpha} 4^{d - \alpha + 6}}{1-2^{- \alpha - r} |\int \varphi(x) \, \mathrm{d}x|^2  (1 + R_\varphi)^{\alpha}} \lVert \varphi \rVert_{L^1} \right\} \quad \text{in the case \(\gamma = 0\) }.
\end{align*}
This ends the proof of the Reconstruction Theorem for \(\gamma \leq 0\). 
\chapter{Applications}

\section{Negative Hölder Spaces}

In Chapter~\ref{chapter:notation} we introduced the space of locally \( \alpha  \)-Hölder functions \( \mathcal{C}^\alpha \) for positive exponents \( \alpha > 0 \). Using very similar techniques as in the proof of the Reconstruction Theorem, we extend the space \( \mathcal{C}^\alpha \) to non-positive exponents \( \alpha \leq 0 \) in this chapter. In the case of \( \alpha \leq 0 \), \( \mathcal{C}^k \) is not a space of continuously differentiable functions but a space of \emph{distributions}. We will also see that the reconstruction \( \mathcal{R}F \) of any germ \( F \) with non-positive homogeneity bound \( \beta \leq 0 \) lies in \( \mathcal{C}^\beta \).

\begin{definition}[Hölder Space]
  Let \( \alpha \leq 0 \) and let \( r_\alpha = \min\left\{ n \in \mathbb{N} : r > -\alpha \right\} \). We define the Hölder space \( C^\alpha \) as the space of all distributions \( T \) such that for any compact set \( K \subset \mathbb{R}^d \) there exists \( C < \infty \) with 
  \begin{gather}
    |T(\psi^\epsilon_x)| \leq C \epsilon^\alpha \label{eq:negative-hoelder-spaces} \\
    \text{for all \(x \in K,  \epsilon \in (0,1] \) and \( \psi \in \mathcal{B}_{r_\alpha} \)}. \nonumber 
  \end{gather}
  The semi-norm \( \lVert \cdot \rVert_{\mathcal{C}^\alpha(K)} \) is defined as 
  \begin{align*}
    \lVert T \rVert_{\mathcal{C}^\alpha(K)} 
    = \sup_{
      \substack{
        x \in K,\\
        \lambda \in (0,1],\\
        \psi \in \mathcal{B}_{r_\alpha}
      }
    } \frac{|T(\psi^\lambda_x)|}{\lambda^\alpha} \hspace{4em} \forall T \in \mathcal{D}'.
  \end{align*}
  Clearly, \( T \in \mathcal{C}^\alpha \iff \lVert T \rVert_{\mathcal{C}^\alpha(K)} < \infty \) for all compact sets \( K \subset \mathbb{R}^d \).
\end{definition}

We present our main result. Let \( \alpha \leq 0 \) and \( T \in \mathcal{D}' \). Then, \( T \) already lies in \( \mathcal{C}^\alpha \) if inequality~\eqref{eq:negative-hoelder-spaces} holds for a \emph{single} arbitrary test function \( \varphi \in \mathcal{D} \) with \( \int \varphi(x) \, \mathrm{d}x \) rather for all \( \psi \in \mathcal{B}_{r_\alpha} \). This characterization of negative Hölder spaces is obtained from the following proposition, which is proved in a similar fashion as the Reconstruction Theorem.

\begin{proposition}\label{proposition:spiderman}
  Let \( T \in \mathcal{D}' \). If there exists a set \( K \subset \mathbb{R}^d \) and a test function \( \varphi \in \mathcal{D}, \int \varphi(x) \, \mathrm{d}x \neq 0 \) such that 
  \begin{gather}\label{equation:spiderman}
    \forall x \in K, \epsilon \in \left\{ 2^{k} \right\}_{k \in \mathbb{N}}: \quad |T(\varphi^\epsilon_x)| \leq \epsilon^\alpha f(\epsilon, x) \\
    \text{for some \( \alpha \leq 0 \) and \( f: (0,1] \times \enlarg{K}{2} \to [0, \infty) \)}, \nonumber
  \end{gather}
  then the above inequality~\eqref{equation:spiderman} also holds for \emph{every} test function \( \psi \in \mathcal{B}_r \) and integer \( r > -\alpha \) in the following sense: for every \( \psi \in \mathcal{B}_r \) and \(r > -\alpha\) we have
  \begin{align*}
    \forall x \in K, \epsilon \in (0,1]: \quad |T(\psi^\epsilon_x)| \leq \{ \mathrm{constant} \} \cdot \epsilon^\alpha \sup_{\substack{\epsilon' \in (0, \epsilon]\\x' \in B(x, 2\epsilon)}} f(\epsilon', x').
  \end{align*}
\end{proposition}

\begin{proof}
  Let \( T, \varphi, K, \alpha \) and \( f \) be as above. Fix an integer \( r > -\alpha \). As in the proof of the Reconstruction Theorem (see~\eqref{definition:tweakedvarphi}), we define \( \hat \varphi = T_\varphi \) for \(a = \frac{1}{\int \varphi(x) \, \mathrm{d}x}\) and \(\lambda_i = \frac{2^{-(i+1)}}{1+R_\varphi}\), \(i = 0,\ldots,r-1\). 
  
  We claim that the tweaked test function \( \hat \varphi \) satisfies~\eqref{equation:spiderman}, just like in the proof of the Reconstruction Theorem (see Lemma~\ref{lemma:hat-phi-satisfies-table}).
\end{proof}

As a corollary, we get the following characterization of negative Hölder spaces.

\begin{corollary}[Characterization of Negative Hölder Spaces]
  Let \( \alpha \leq 0 \) and \( T \in \mathcal{D}' \). Then, the following conditions are equivalent 
  \begin{enumerate}
    \item \( T \in \mathcal{C}^\alpha \);
    \item There exists an integer \( r > - \alpha \) such that~\eqref{eq:negative-hoelder-spaces} holds for all test functions \( \psi \in \mathcal{B}_r \);
    \item There exists \( \varphi \in \mathcal{D} \) with \( \int \varphi(x) \, \mathrm{d}x  \neq 0 \) such that for any compact set \( K \subset \mathbb{R}^d \) there exists a constant \( \tilde C < \infty \) with 
    \begin{gather*}
      |T(\varphi^\epsilon_x)| \leq \tilde C \epsilon^\alpha \\
      \text{for all \( x \in K \)and \( \epsilon \in \left\{ 2^{-k} \right\}_{k \in \mathbb{N}} \)}.
    \end{gather*}
  \end{enumerate}
\end{corollary}

\begin{proof}\(  \)

  \begin{itemize}
    \item \( 1. \implies 2. \) This holds because \( \mathcal{B}_r \subset \mathcal{B}_{r_\alpha} \) for \( r \geq r_\alpha \).
    \item \( 2. \implies 3. \) Choose any \( \varphi \in \mathcal{B}_{r} \) with \( \int \varphi(x) \, \mathrm{d}x \).
    \item \( 3. \implies 1. \) Apply Proposition~\ref{proposition:spiderman} with \( f \equiv \tilde C \).
  \end{itemize}
\end{proof}

\printbibliography

% \appendix
% \chapter{More Monticello Candidates}

\end{document}
