
\chapter{Proof Continued for \texorpdfstring{\(\gamma > 0\)}{gamma > 0}}\label{chapter:proof-gamma-positive}

\section{Step 4: \texorpdfstring{\(\sum |g_k''|\) converges for \(\gamma > 0\)}{sum g''k converges for gamma > 0}}\label{step4}

In Step 3, we showed that the partial sum \( \sum g_k' \) converges. Recall that \(f_n = f_1 + \sum g_k' + \sum g_k''\). What about the other partial sum \( \sum g_k'' \)?

Let \(\psi \in \mathcal{D}(\overline K_1)\). We start with \(g_k''(\psi) = \iint (F_z - F_y)(\hat{\varphi}^{\epsilon_k}_{y}) \check \varphi^{\epsilon_k}(y-z) \psi(z) \, \mathrm{d}y \, \mathrm{d}z\). The tweaked test function \(\check \varphi^{\epsilon_k}\) has compact support in \(B(0,\epsilon_k)\), and \(\psi\) has compact support in \(\overline K_1\). So, we have 
\begin{align*}
    |g_k''(\psi)| 
    &\leq \sup_{\substack{z \in \overline{K}_1 \\ |y-z|\leq \epsilon_k}}
    \left( |(F_z -F_y)(\hat \varphi^{\epsilon_k}_y)| \right) \lVert \check \varphi^{\epsilon_k} * \psi \rVert_{L^1}\\
    &\leq \sup_{\substack{z \in \overline{K}_1 \\ |y-z|\leq \epsilon_k}}
    \left( |(F_z -F_y)(\hat \varphi^{\epsilon_k}_y)| \right) \lVert \check \varphi^{\epsilon_k} \rVert_{L^1} \lVert \psi \rVert_{L^1}.
\end{align*}
Using the coherence condition~\eqref{coherence-hat}, we get
\begin{align}
    |g_k''(\psi)| 
    \leq \hat C \epsilon_k^\alpha (|y-z| + \epsilon_k)^{\gamma - \alpha} \lVert \check \varphi^{\epsilon_k} \rVert_{L^1} \lVert \psi \rVert_{L^1}
    &\leq \hat C \epsilon_k^\alpha (2\epsilon_k)^{\gamma - \alpha} \lVert \check \varphi^{\epsilon_k} \rVert_{L^1} \lVert \psi \rVert_{L^1}\nonumber\\
    &= \left(\hat C  2^{\gamma - \alpha} \lVert \check \varphi \rVert_{L^1} \lVert \psi \rVert_{L^1}\right) \epsilon_k^\gamma. \label{Mustermaus}
\end{align}
Thus, \(\sum |g_k''|\) converges if \(\gamma > 0\).

\section{Step 5: \texorpdfstring{\(f^K\) is a local reconstruction}{fK satisfies the reconstruction theorem}}

In Step 3 and Step 4 we showed that \(\lim\limits_{n \to \infty}f_n(\psi)\) exists for all \(\psi \in \mathcal{D}(\overline{K}_1)\). Hence, we define
\begin{align*}
    f^K \coloneqq \lim\limits_{n \to \infty}f_n \overset{\eqref{approximating-distributions-alternative}}{=} f_1 + \sum^{n-1}_{k=1} g_k' + \sum^{n-1}_{k=1} g_k''.
\end{align*}

\begin{remark}
    The notation \(f^K\) explicitly emphasizes that \(\lim\limits_{n \to \infty}f_n\) depends on the compact set \(K\) that we fixed in \emph{Step 0: Setup} (recall that \(\rho\) depends on \(K\)).
\end{remark}
 
In this step, we prove for \(\gamma > 0\) that \(f^K\) is a distribution, and there exists a constant \(C < \infty\) such that 
\begin{gather*}
    |(f^K - F_x)(\psi^\lambda_x)| \leq C \lambda^\gamma \tag{\ref{reconstruction-theorem}
    }\\ 
    \text{uniformly for \(\psi \in \mathcal{B}_r\), \(x \in K\), \(\lambda \in (0,1]\)}.
\end{gather*}
The constant \(C\) will be explicitly computed in~\eqref{constant:rec-gamma-bigger-zero}. 

\subsection*{\(f^K\) is a distribution} 

We want to show that \(f^K \in \mathcal{D}'(\bar K_1)\). \(f_1\) is a distribution. So, we find a constant such that \(|f_1(\psi)| \leq \mathrm{constant} \cdot \lVert \psi \rVert_{C^r}\) for all \(\psi \in \mathcal{D}(\bar K_1)\). Then, we use the upper bounds for \(\sum |g_k'|\) and \(\sum |g_k''|\) established in Step 3 and Step 4 (see~\eqref{Mustermkatze} and~\eqref{Mustermaus}) to find a constant such that  
\begin{align*}
    |f^K(\psi)| \leq \mathrm{constant} \cdot \left\{ \lVert \psi\rVert_{L^1} + \lVert \psi\rVert_{C^r} \right\} \leq  \mathrm{constant} \cdot \left\{  \mathrm{Vol}(\bar K_{3/2}) + 1 \right\} \lVert \psi \rVert_{C^r}.
\end{align*}
This proves that \(f^K\) is a distribution. 


\subsection*{Setup} 

We fix \(\psi \in \mathcal{B}_r, x \in K\) and \(\lambda \in (0,1]\). We define a function \(\tilde f\) that measures the error of the reconstruction \(f^K\) with respect to \(F_x\):  
\begin{align*}
    \tilde f(\phi) &= f^K(\phi) - F_x(\phi), \quad \qquad \phi \in \mathcal{D}(\overline{K}_1)\\
    \tilde f_n(\phi) &= f_n(\phi) - F_x(\phi * \rho^{\epsilon_n}), \quad n \in \mathbb{N}.
\end{align*}
By Lemma~\ref{mollifier-lemma}, \(\tilde f_n(\phi)\) converges to \(\tilde f(\phi)\) as \(n \to \infty\).

Let \(N \in \mathbb{N}\) be the smallest index such that \(\epsilon_N \leq \lambda\); that is \(N = \min \left\{ k \in \mathbb{N} : \epsilon_k \leq \lambda  \right\}\). Using the triangle inequality, we then write 
\begin{align*}
    |\tilde f (\psi^\lambda_x)| \leq  |\tilde f_N (\psi^\lambda_x)| + |(\tilde f - \tilde f_N) (\psi^\lambda_x)|.
\end{align*}

\subsection*{Bounding \(|\tilde f_N (\psi^\lambda_x)|\)} 

We start with
\begin{align*}
    |\tilde f_N (\psi^\lambda_x) |
    = |f_N(\psi^\lambda_x) - F_x(\psi^\lambda_x * \rho^{\epsilon_N}) |
    &= |\int_{\mathbb{R}^d} F_y(\rho_y^{\epsilon_N}) \psi^{\lambda}_x(y) \, \mathrm{dy} - F_x(\psi^\lambda_x * \rho^{\epsilon_N})| \\
    &\Downarrow \text{Lemma~\ref{lemma:mollified-distribution}} \\
    &= |\int_{\mathbb{R}^d} F_y(\rho_y^{\epsilon_N}) \psi^{\lambda}_x(y) \, \mathrm{dy} - \int_{\mathbb{R}^d} F_x(\rho_y^{\epsilon_N}) \psi^{\lambda}_x(y) \, \mathrm{dy}| \\
    &= |\int_{\mathbb{R}^d} (F_y - F_x)((\hat \varphi^{2\epsilon_N} * \hat \varphi^{\epsilon_N})(\cdot - y)) \psi^{\lambda}_x(y) \, \mathrm{dy}|\\
    &\Downarrow \text{Lemma~\ref{lemma:mollified-distribution}} \\
    &= |\iint (F_y - F_x)(\hat \varphi^{\epsilon_N}_z) \hat \varphi^{2 \epsilon_N}(z-y)\psi^\lambda_x(y) \, \mathrm{d}z \, \mathrm{d}y| \\
    &\leq\sup_{\substack{y \in \overline K_1,\\|z-y|\leq\epsilon_N}} |(F_y - F_x)(\hat \varphi^{\epsilon_N}_z)| \cdot \lVert \hat \varphi^{2\epsilon_N}\rVert_{L^1} \, \lVert \psi^{\lambda}_x\rVert_{L^1}.
\end{align*}
Note that we have the term \(\sup|(F_y - F_x)(\hat \varphi^{\epsilon_N}_z)|\), which subtly indicates that we need to use~\eqref{coherence-hat}. To do so, we write for all \(y \in  B(x,\epsilon),|z-y|\leq\epsilon_N\)
\begin{align*}
    |(F_y - F_x)(\hat \varphi^{\epsilon_N}_z)| &\leq |(F_y - F_z)(\hat \varphi^{\epsilon_N}_z)| + |(F_z - F_x)(\hat \varphi^{\epsilon_N}_z)| \\
    &\leq \hat C \epsilon_N^\alpha(|z-y| + \epsilon_N)^{\gamma - \alpha} + \hat C \epsilon_N^\alpha(|z-x| + \epsilon_N)^{\gamma - \alpha} \\
    &\leq \hat C \epsilon_N^\alpha (2\epsilon_N)^{\gamma - \alpha} + \hat C \epsilon_N^\alpha (|z-y| + |y-x| + \epsilon_N)^{\gamma - \alpha} \\
    &\leq \hat C  2^{\gamma - \alpha} \lambda^{\gamma} +  \hat C  \lambda^\alpha(3\lambda)^{\gamma - \alpha} \\
    &= 2 \hat C 3^{\gamma - \alpha}\lambda^\gamma.
\end{align*}
So, we obtain
\begin{align*}
    |\tilde f_N (\psi^\lambda_x) | 
    &\leq 2 \hat C 3^{\gamma - \alpha}\lambda^\gamma \lVert \hat \varphi^{2\epsilon_N}\rVert_{L^1} \, \lVert \psi^{\lambda}_x\rVert_{L^1}\\
     &= 2 \hat C 3^{\gamma - \alpha}\lambda^\gamma \lVert \hat \varphi\rVert_{L^1} \, \lVert \psi^{\lambda}_x\rVert_{L^1} \\
     &\Downarrow \text{\(\lVert \psi^{\lambda}_x\rVert_{L^1} \leq 2^d\), see~\eqref{bound-psi-l1}} \\
     &\leq \left \{ 3^{\gamma - \alpha} 2^{d+1} \hat C  \lVert \hat \varphi\rVert_{L^1} \right \} \lambda^\gamma.
\end{align*}

\subsection*{Bounding \(|(\tilde f - \tilde f_N) (\psi^\lambda_x)|\)} 

We begin with
\begin{align*}
    |(\tilde f - \tilde f_N) (\psi^\lambda_x)| = |\lim_{n \to \infty} (\tilde f_n - \tilde f_N) (\psi^\lambda_x)| \leq \sum_{k \geq N}|(\tilde f_{k+1} - \tilde f_k)(\psi^\lambda_x)|.
\end{align*}
By definition \(\tilde f_k(\psi) = f_k(\psi) - F_x(\psi * \rho^{\epsilon_k})\), we then obtain 
\begin{align*}
    |(\tilde f - \tilde f_N) (\psi^\lambda_x)| 
    &\leq \sum_{k \geq N} |(f_{k+1} - f_k)(\psi^\lambda_x) - F_x(\psi^\lambda_x * (\rho^{\epsilon_{k+1}} - \rho^{\epsilon_k}))| \\
    &\Downarrow \text{by definition \(f_k = f_1 + \sum^{k-1}_{j=1} g_j' + \sum^{k-1}_{j=1}g_j''\)}\\
    &\leq
    \sum_{k \geq N} |(g_k' + g_k'')(\psi^{\lambda}_x) - F_x(\psi^\lambda_x * (\rho^{\epsilon_{k+1}} - \rho^{\epsilon_k}))| \\
    &\leq \sum_{k \geq N} \underbrace{|g_k''(\psi^{\lambda}_x)|}_{\text{(A)}} + \underbrace{ |g_k'(\psi^\lambda_x) - F_x(\psi^\lambda_x * (\rho^{\epsilon_{k+1}} - \rho^{\epsilon_k}))|}_{\text{(B)}}.
\end{align*}
We are going to bound (A) and (B) separately. Our short-term goal is to find a constant such that \(|(\tilde f - \tilde f_N) (\psi^\lambda_x)| \leq \{\mathrm{constant}\} \cdot \lambda^{\gamma}\). 

\vspace{0.5em}

\begin{itemize}
    \item (A): We start with (A) that follows almost immediately from Step 4. In Chapter~\ref{step4}: Step 4 we showed that \(|g_k''(\psi^{\lambda}_x)| 
    {\leq}  \left(\hat C  2^{\gamma - \alpha} \lVert \check \varphi \rVert_{L^1} \lVert \psi^{\lambda}_x \rVert_{L^1}\right) \epsilon_k^\gamma\). We also know that \(\lVert \psi^\lambda_x \rVert_{L^1} = \lVert \psi \rVert_{L^1} \leq 2^d\) by~\eqref{bound-psi-l1}. Then, we have
    \begin{align*}
        \sum_{k \geq N} |g_k''(\psi^{\lambda}_x)|  \leq 
        \left(\hat C  2^{\gamma - \alpha} \lVert \check \varphi \rVert_{L^1} \lVert \psi^{\lambda}_x \rVert_{L^1}\right) \sum_{k \geq N}  \epsilon_k^\gamma 
        \leq \left(\hat C  2^{\gamma - \alpha + d} \lVert \check \varphi \rVert_{L^1} \right) \sum_{k \geq N}  \epsilon_k^\gamma.
    \end{align*}
    The geometric series \(\sum\limits_{k \geq N}  \epsilon_k^\gamma\) converges because \(\gamma > 0\), and therefore
    \begin{align*}
        \sum_{k \geq N} |g_k''(\psi^{\lambda}_x)| \leq 
        \frac{\hat C  2^{\gamma - \alpha + d} \lVert \check \varphi \rVert_{L^1}}{1-2^{-\gamma}} \epsilon_N^{\gamma}
        \leq
        \left\{\frac{\hat C  2^{\gamma - \alpha + d} \lVert \check \varphi \rVert_{L^1}}{1-2^{-\gamma}}\right\} \lambda^{\gamma}.
    \end{align*}
 
    \item (B): Next, we bound \(|g_k'(\psi^\epsilon_x) - F_x(\psi^\epsilon_x * (\rho^{\epsilon_{k+1}} - \rho^{\epsilon_k}))|\). For that we need a technical lemma that also turns out to be useful in the case \(\gamma \leq 0\). Hence, we state it as a lemma here, so we can refer to it in later chapters.
    

    \begin{lemma}\label{technical-lemma-2}
        For any \(\gamma \in \mathbb{R}\) we have 
        \begin{align*}
            |g_k'(\psi^\lambda_x) - F_x(\psi^\lambda_x * (\rho^{\epsilon_{k+1}} - \rho^{\epsilon_k}))| \leq 4^{d + \gamma - \alpha} \hat C \lVert \check \varphi \rVert_{L^1} \begin{cases}
                \lambda^{\gamma - \alpha - r} \epsilon_k^{\alpha + r}  \quad &\text{if \(\epsilon_k < \lambda\) }\\
                \epsilon_k^\gamma & \text{if \(\epsilon_k \geq \lambda\) }
            \end{cases}
        \end{align*}
        and 
        \begin{align*}
            \sum_{k \geq N} |g_k'(\psi^\lambda_x) - F_x(\psi^\lambda_x * (\rho^{\epsilon_{k+1}} - \rho^{\epsilon_k}))|
            \leq
            \left \{ \frac{\hat C 4^{\gamma - \alpha + d} \lVert \check \varphi \rVert_{L^1} }{1-2^{-\alpha - r}} \right \} \lambda^{\gamma}.
        \end{align*}
        Recall that \(\epsilon_k = 2^{-k}\) and \(N = \min\left\{ k \in \mathbb{N}: \epsilon_k \leq \epsilon \right\}\).
    \end{lemma}

    \begin{proof}
        By~\eqref{rho-dif} and Corollary~\ref{cor:minosokoad}, we obtain 
        \begin{align*}
            F_x(\psi^\lambda_x * (\rho^{\epsilon_{k+1}} - \rho^{\epsilon_k})) &= \iint F_x(\hat \varphi^{\epsilon_k}_y) \check \varphi^{\epsilon_k}(y-z) \psi^{\lambda}_x(z) \, \mathrm{d}y \mathrm{d}z\\
             &= \int F_x(\hat \varphi^{\epsilon_k}_y) ( \check\varphi^{\epsilon_k} * \psi^{\lambda}_x)(y) \, \mathrm{d}y.
        \end{align*}
        By definition of \(g'_k\) (see~\eqref{gk-formular}), we have 
        \begin{align*}
            g_k'(\psi^\lambda_x) = \int F_y(\hat \varphi^{\epsilon_k}_y) (\check \varphi^{\epsilon_k}* \psi^\lambda_x)(y) \, \mathrm{d}y.
        \end{align*}
        Hence, we get
        \begin{align}\label{temp:klwmrlkewmrw}
            |g_k'(\psi^\lambda_x) - F_x(\psi^\lambda_x * (\rho^{\epsilon_{k+1}} - \rho^{\epsilon_k}))| 
            &= \int (F_y-F_x)(\hat \varphi^{\epsilon_k}_y) (\check \varphi^{\epsilon_k}* \psi^\lambda_x)(y) \, \mathrm{d}y,
        \end{align}
        for which we find an upper bound using the second inequality of Lemma~\ref{lemma:WALFANGER}
        \begin{align*}
           ~\eqref{temp:klwmrlkewmrw} \leq 4^d \lVert \check \varphi \rVert_{L^1} \min \left\{ \frac{\epsilon_k}{\lambda}, 1 \right\}^r \sup_{y \in B(x, \lambda + \epsilon_k)} |(F_y - F_x)(\hat \varphi_y^{\epsilon_k})|.
        \end{align*}
        The supremum of \(|F_y-F_x|\) is estimated with the coherence condition
        \begin{align*}
            \sup_{y \in B(x, \lambda + \epsilon_k)} |(F_y - F_x)(\hat \varphi^{\epsilon_k}_y)| &\overset{\eqref{coherence-hat}}{\leq} \hat C \epsilon_k^{\alpha} \sup_{y \in B(x, \lambda + \epsilon_k)} (|x - y| + \epsilon_k )^{\gamma - \alpha} \\
            &\;\leq \hat C \epsilon^\alpha_k(\lambda + 2\epsilon_k)^{\gamma - \alpha} \\
            &\leq \hat C \epsilon^\alpha_k 3^{\gamma - \alpha} \begin{cases}
                \epsilon_k^{\gamma - \alpha} &\text{if \(\epsilon_k > \lambda\) } \\
                \lambda^{\gamma - \alpha} &\text{if \(\epsilon_k \leq \lambda\) }
            \end{cases}
        \end{align*}
        This proves the first claim of the lemma.

        To prove the last line, note that \(\sum\limits_{k \geq N} \epsilon_k^{\alpha + r} = \frac{\epsilon_N^{\alpha+ r}}{1 - 2^{-\alpha - r}} \leq \frac{\lambda^{\alpha + r}}{1 - 2^{- \alpha - r}}\)  (regarding the last inequality, we chose \(r \in \mathbb{N}\) such that  \(\alpha + r > 0\)). Since \(k \geq N\), we have \(\epsilon_k < \lambda\). So, we use the first claim of the lemma together with \(\sum\limits_{k \geq N} \epsilon_k^{\alpha + r} \leq \frac{\lambda^{\alpha + r}}{1 - 2^{- \alpha - r}}\) to obtain
        \begin{align*}
            \sum_{k \geq N} |g_k'(\psi^\lambda_x) - F_x(\psi^\lambda_x * (\rho^{\epsilon_{k+1}} - \rho^{\epsilon_k}))| \leq \left\{\frac{4^{d + \gamma - \alpha}}{1-2^{-\alpha - r}} \lVert \check \varphi \rVert_{L^1} \hat C \right\} \lambda^{\gamma}. 
        \end{align*} 
    \end{proof}
    With this lemma proved, we use the second line to find an upper bound for (B).
\end{itemize}

Finally, we use (A) and (B) to estimate
\begin{align*}
    |(\tilde f - \tilde f_N) (\psi^\lambda_x)| 
    &\leq
    \left \{ \frac{\hat C 4^{\gamma - \alpha + d} \lVert \check \varphi \rVert_{L^1} }{1-2^{-\alpha - r}} \right \} \lambda^{\gamma} + \left\{\frac{\hat C  2^{\gamma - \alpha + d} \lVert \check \varphi \rVert_{L^1}}{1-2^{-\gamma}}\right\} \lambda^{\gamma} \\
    &\leq \left\{ 2 \frac{\hat C 4^{\gamma - \alpha + d} \lVert \check \varphi \rVert_{L^1} }{1-2^{-\min\{\alpha + r, \gamma\}}} \right\}\cdot \lambda^{\gamma} \\
    &\Downarrow \text{\(\lVert \check \varphi \rVert_{L^1} \leq 2\lVert \hat \varphi \rVert_{L^1} \) by definition of \(\check \varphi\)} \\
    &\leq \left\{ \frac{\hat C 4^{\gamma - \alpha + d + 1} \lVert \hat \varphi \rVert_{L^1} }{1-2^{-\min\{\alpha + r, \gamma\}}} \right\}\cdot \lambda^{\gamma}.
\end{align*}

\subsection*{Finish}

Now, we are finally able to prove that \(f^K\) is a local reconstruction. We show that \(f^K\) satisfies~\eqref{reconstruction-theorem} uniformly for all \(\psi \in \mathcal{B}_r, x \in K\) and \(\lambda \in (0,1]\). We have
\begin{align*}
    |f^K(\psi^\lambda_x) - F_x(\psi^\lambda_x)| = |\tilde f (\psi^\lambda_x)| 
    &\leq  |\tilde f_N (\psi^\lambda_x)| + |(\tilde f - \tilde f_N) (\psi^\lambda_x)| \\
    &\leq \left \{ 3^{\gamma - \alpha} 2^{d+1} \hat C  \lVert \hat \varphi\rVert_{L^1} \right \} \lambda^\gamma +  \left\{ \frac{ 4^{\gamma - \alpha + d + 1} \hat C \lVert \hat \varphi \rVert_{L^1} }{1-2^{-\min\{\alpha + r, \gamma\}}} \right\}\cdot \lambda^{\gamma}  \\
    &\leq  \left\{2\frac{  4^{\gamma - \alpha + d + 1} \hat C \lVert \hat \varphi \rVert_{L^1}}{1-2^{-\min\{\alpha + r, \gamma\}}} \right\} \lambda^{\gamma}.
\end{align*}
This proves that \(f^K\) is a local reconstruction of \((F_x)_{x \in \mathbb{R}^d}\).

For completeness, we estimate \(\hat C \lVert \hat \varphi \rVert_{L^1}\). Lemma~\ref{lemma:tweaked-l1-norm} gives us \(\lVert \hat \varphi \rVert_{L^1} \leq \frac{e^2 r \lVert \varphi \rVert_{L^1}}{|\int \varphi(x) \mathrm{d}x |} \). The bound for \(\hat C\) in~\eqref{constant:hat-c} then yields 
\begin{align}\label{chatnorml1}
    \hat C \lVert \hat \varphi \rVert_{L^1} \leq \left(C\frac{e^2 r}{|\int \varphi(x)\, \mathrm{d}x|} \left(\frac{2^{-(r+1)}}{1+R_\varphi}\right)^\alpha\right)\left( \frac{e^2 r}{|\int \varphi(x) \mathrm{d}x |} \lVert \varphi \rVert_{L^1}\right).
\end{align}
So, the constant factor in front of \(\lambda^\gamma\) reads 
\begin{align*}
    \left\{2\frac{  4^{\gamma - \alpha + d + 1}}{1-2^{-\min\{\alpha + r, \gamma\}}} \left(C\frac{e^2 r}{|\int \varphi(x)\, \mathrm{d}x|} \left(\frac{2^{-(r+1)}}{1+R_\varphi}\right)^\alpha\right) \left(\frac{e^2 r}{|\int \varphi(x) \mathrm{d}x |} \lVert \varphi \rVert_{L^1}\right) \right\}.
\end{align*}
Finally, we bound \(e^2 \leq 4\) to obtain the constant
\begin{align}\label{constant:rec-gamma-bigger-zero}
    \left\{C\frac{  4^{\gamma - \alpha + d + 6}r^2}{1-2^{-\min\{\alpha + r, \gamma\}}} \frac{\lVert \varphi \rVert_{L^1}}{2^{\alpha(r+1)} (1+R_\varphi)^{\alpha}|\int \varphi(x) \mathrm{d}x |^2}  \right\},
\end{align} 
where \(R_\varphi\) is the radius of the ball such that \(\mathrm{supp}(\varphi) \subset B(0, R_\varphi)\).  


\section{Step 6: \texorpdfstring{\((f^K)\) are consistent}{fK's are consistent}}

In this final step, we construct a global reconstruction \(f = \mathcal{R}F\) of \((F_x)_{x \in \mathbb{R}^d}\)  using the local reconstructions \(f^K\). First, we prove that the family of local reconstructions \((f^K)_{K \subset \mathbb{R}^d, \text{\(K\) compact}}\) is consistent in the sense of 
\begin{align*}
    \forall \text{compact sets \(K\) and \(H\) with } K \subset H: \psi \in \mathcal{D}(\bar K_1) \implies f^K(\psi) = f^H(\psi).
\end{align*}
To prove this claim, we use the Uniqueness Theorem~\ref{theorem:uniqueness-reconstruction}. We have \((f^K - F_x)(\varphi^\lambda_x) \to 0\) and \((g^K - F_x)(\varphi^\lambda_x) \to 0\) as \(\lambda \to 0\) uniformly for \(x \in \bar K_1\) because of \(\gamma > 0\) and~\eqref{reconstruction-theorem}. Hence, we apply the Uniqueness Theorem and conclude that \(f^K(\psi) = f^H(\psi)\) for all \(\psi \in \mathcal{D}(\bar K_1)\).  

With the consistency property being shown, we move on to construct a global reconstruction. Let \(\psi \in \mathcal{D}\). Then, we define \(\mathcal{R}F: \psi \mapsto f^K(\psi)\) where \(K \subset \mathbb{R}^d\) is a compact set large enough such that \(\psi\) is supported in \(\bar K_1\). The map \(\mathcal{R}F\) is well-defined because we showed the consistency property. Moreover, \(\mathcal{R}F\) is a reconstruction in the sense of~\eqref{reconstruction-theorem} because \(f^K\) is a local reconstruction. This ends the proof of the Reconstruction Theorem in the case \(\gamma > 0\). 
